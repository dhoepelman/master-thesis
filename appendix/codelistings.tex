\lstset{style=sharpc}
\begin{lstlisting}[float,caption={XLParser Print method (simplified)}, label={lst:xlparserprint}]
public static string Print(this ParseTreeNode node) {
	// Print token values
	if(node is Terminal) return node.Token.Text;
	
	// Select is C#'s map function
	var ch = node.ChildNodes.Select(Print).ToList();
	
	switch(node.Type()) {
		case "ArrayFormula":
			return "{=" + ch[0] + "}";
		case "FunctionCall":
			if(node.IsBinaryOperation()) {
				return ch[0] + " " + ch[1] + " " + ch[2];
			}
			if(node.isNamedFunction()) {
				return String.Join("", ch) + ")";
			}
			// some more conditions
			break;
		// More cases for every node type
	}
}
\end{lstlisting}

\lstset{style=sharpc}
\begin{lstlisting}[float,caption={Formula AST replacement (simplified)}, label={lst:astreplace}]

/* Context contains the workbook and worksheet of a node */

public static ParseTreeNode Replace(ParseTreeNode subject, ParseTreeNode search, ParseTreeNode replace, Context csub, Context csearch, Context crepl) {
	// Check if the subject matches search
	if(Equals(subject, search, csub, csearch)) {
		// We can return the replacement.
		// Moveto handles changing reference prefixes if necesarry 
		return MoveTo(replace, crepl, csub);
	}
	
	// No match, if we are at a leaf node, simply return the leaf node
	if (subject.ChildNodes.Count == 0) return subject;
	
	// Otherwise continue the replacement on the child nodes
	return new ParseTreeNode() {
		Type = subject.Type(),
		// Select is C#'s map
		ChildNodes = subject.ChildNodes.Select(child => Replace(child, search, replace, csub, csearch, crepl))
	};
}

public static bool Equals(ParseTreeNode p1, ParseTreeNode p2, Context c1, Context c2) {
	
	// RemoveNonEqualityAffectingNodes removes things like brackets,
	// which do not affect the equality of nodes
	p1 = RemoveNonEqualityAffectingNodes(p1);
	p2 = RemoveNonEqualityAffectingNodes(p2);
	
	// Qualify adds workbook and worksheet prefix to all references, so that
	// equality isn't affected by whether or not these are supplied in the original formula
	p1 = c1.Qualify(p1);
	p2 = c2.Qualify(p2);
	
	return p1.Type() == p2.Type()
		// Compare the token values if these are tokens
	    && (p1 is Terminal && p1.Token.ValueString == p2.Token.ValueString)
	    // Compare child count
	    && p1.ChildNodes.Count == p2.ChildNodes.Count
	    // Check if all children are equal
	    && p1.ChildNodes.Zip(p2.ChildNodes).All((ch1, ch2) => Equals(ch1, ch2, c1, c2));
}
\end{lstlisting}

\lstset{style=sharpc}
\begin{lstlisting}[float,caption={Extract formula refactoring (simplified)}, label={lst:extractformula}]
public void ExtractFormula(Range applyto, Location moveto, ParseTreeNode subformula) {
	
	/** Check if all applyto cells contain subformula **/
	
	/** Check if target cell is empty **/
	
	/** Check if subformula contains any non-absolute references **/
	
	// Set the target cell to the subformula
	moveto.Formula = subformula.Print();
	// and get its parsed address reference
	var replacementAST = moveto.Address().Parse();
	
	// Apply the refactoring once per unique R1C1 formula
	foreach (var uniqueR1C1 in applyto.Cells.GroupBy()(c => c.FormulaR1C1)) {
		var prototype = uniqueR1C1.First();
		var AST_or = prototype.Parse();
		
		prototype.Formula = Replace(AST_or, subformula, replacementAST, /*...*/).Print();
		
		foreach(var cell in uniqueR1C1) {
			cell.FormulaR1C1 = prototype.FormulaR1C1;
		}
	}
}

public void ExtractFormula(Range applyto, Direction dir, ParseTreeNode subformula) {
	
	/** Check if all applyto cells contain subformula **/
	
	/** Insert new cells in the appropriate direction **/
	
	/** Set all new cells to contain the subformula formula **/
	
	// Apply the refactoring once per unique R1C1 formula
	foreach (var uniqueR1C1 in applyto.Cells.GroupBy()(c => c.FormulaR1C1)) {
		var prototype = uniqueR1C1.First();
		var AST_or = prototype.Parse();
		var replacementAST = prototype.Offset[dir].Address().Parse();
		
		prototype.Formula = Replace(AST_or, subformula, replacementAST, /*...*/).Print();

		foreach(var cell in uniqueR1C1) {
			cell.FormulaR1C1 = prototype.FormulaR1C1;
		}
	}
}
\end{lstlisting}

\lstset{style=sharpc}
\begin{lstlisting}[float,caption={Inline Formula Refactoring (simplified)}, label={lst:inlineformula}]
public void InlineFormula(Cell toInline) {
	// Dependendants are gotten from Excel
	var dependents = toInline.dependents;
	if(dependents.Count == 0) {
		// Abort, no dependants
	}
	
	var AstToRepl = Parse(toInline.Address);
	var AstReplacement = Parse(toInline.Formula);
	
	// Check if this cell is part of any named ranges with more than one cell
	if(toInline.Names.Any(name => name.Cells.Count > 1)) {
		// Abort
	}
	
	// AST representation of the names
	var names = toInline.Names.Select(Parse);
	
	foreach(var dependent in dependents) {
		var AstOriginal = Parse(dependent.Formula);
		
		// Abort if to be inlined cell is references as part of a range
		if(CellContainedInRanges(toInlineAddress, AstOriginal)) {
			// Abort
		}
		
		// Replace references to the cell with the value
		var AstNew = Replace(AstOriginal, AstToRepl, AstReplacement, /*...*/);
		foreach(var name in names) {
			AstNew = Replace(AstNew, name, AstReplacement, /*...*/);
		}
		
		dependent.Formula = AstNew.Print();
	}
	
	toInline.Delete();
	
}
\end{lstlisting}

\lstset{style=sharpc}
\begin{lstlisting}[float,caption={Group References Refactoring (simplified)}, label={lst:groupreferences}]
public ParseTreeNode GroupReferences(ParseTreeNode formula) {
	// Get all nodes representing varargs functions
	var targets = formula.AllNodes().Where(IsVarargsFunction);
	
	foreach(var function in targets) {
		// split arguments that can be grouped from those than can't
		var toGroupArguments = function.arguments
			.Where(node => node.IsCellOrRange());
		var ignoredArguments = function.arguments
			.Where(node => !node.isCellOrRange());
		
		var grouped = new List<ParseTreeNode>();
		
		// Several characteristics define whether references can be grouped
		foreach(var sheetGroup in SplitByWorksheet(toGroupArguments)) {
			foreach(var toGroup in SplitByAbsoluteMarkers(sheetGroup)) {
				grouped.Add(GroupUsingExcel(toGroup));
			}
		}
		
		function.arguments = ignoredArguments.Concat(grouped);
	}
	
	return formula;
}

\end{lstlisting}