\chapter{Code listings}

\lstset{style=sharpc}
\begin{lstlisting}[float,caption={XLParser Print method (simplified)}, label={lst:xlparserprint}]
public static string Print(this Node node) {
	// Print token values
	if(node is Terminal) return node.Token.Text;
	
	// Select is C#'s map function
	var ch = node.ChildNodes.Select(Print).ToList();
	
	switch(node.Type()) {
		case "ArrayFormula":
			return "{=" + ch[0] + "}";
		case "FunctionCall":
			if(node.IsBinaryOperation()) {
				return ch[0] + " " + ch[1] + " " + ch[2];
			}
			if(node.isNamedFunction()) {
				return String.Join("", ch) + ")";
			}
			// some more conditions
			break;
		// More cases for every node type
	}
}
\end{lstlisting}

\lstset{style=sharpc}
\begin{lstlisting}[float,caption={Formula AST replacement (simplified)}, label={lst:astreplace}]

/* Context contains the workbook and worksheet of a node */

public static Node Replace(Node subject, Node search, Node replace, Context csub, Context csearch, Context crepl) {
	// Check if the subject matches search
	if(Equals(subject, search, csub, csearch)) {
		// We can return the replacement.
		// Moveto handles changing reference prefixes if necesarry 
		return MoveTo(replace, crepl, csub);
	}
	
	// No match, if we are at a leaf node, simply return the leaf node
	if (subject.ChildNodes.Count == 0) return subject;
	
	// Otherwise continue the replacement on the child nodes
	return new Node() {
		Type = subject.Type(),
		// Select is C#'s map
		ChildNodes = subject.ChildNodes.Select(child => Replace(child, search, replace, csub, csearch, crepl))
	};
}

public static bool Equals(Node p1, Node p2, Context c1, Context c2) {
	
	// RemoveNonEqualityAffectingNodes removes things like brackets,
	// which do not affect the equality of nodes
	p1 = RemoveNonEqualityAffectingNodes(p1);
	p2 = RemoveNonEqualityAffectingNodes(p2);
	
	// Qualify adds workbook and worksheet prefix to all references, so that
	// equality isn't affected by whether or not these are supplied in the original formula
	p1 = c1.Qualify(p1);
	p2 = c2.Qualify(p2);
	
	return p1.Type() == p2.Type()
		// Compare the token values if these are tokens
	    && (p1 is Terminal && p1.Token.ValueString == p2.Token.ValueString)
	    // Compare child count
	    && p1.ChildNodes.Count == p2.ChildNodes.Count
	    // Check if all children are equal
	    && p1.ChildNodes.Zip(p2.ChildNodes).All((ch1, ch2) => Equals(ch1, ch2, c1, c2));
}
\end{lstlisting}

\lstset{style=sharpc}
\begin{lstlisting}[float,caption={Extract formula refactoring (simplified)}, label={lst:extractformula}]
public void ExtractFormula(Range applyto, Location moveto, Node subformula) {
	
	/** Check if all applyto cells contain subformula **/
	
	/** Check if target cell is empty **/
	
	/** Check if subformula contains any non-absolute references **/
	
	// Set the target cell to the subformula
	moveto.Formula = subformula.Print();
	// and get its parsed address reference
	var replacementAST = moveto.Address().Parse();
	
	// Apply the refactoring once per unique R1C1 formula
	foreach (var uniqueR1C1 in applyto.Cells.GroupBy()(c => c.FormulaR1C1)) {
		var prototype = uniqueR1C1.First();
		var AST_or = prototype.Parse();
		
		prototype.Formula = Replace(AST_or, subformula, replacementAST, /*...*/).Print();
		
		foreach(var cell in uniqueR1C1) {
			cell.FormulaR1C1 = prototype.FormulaR1C1;
		}
	}
}

public void ExtractFormula(Range applyto, Direction dir, Node subformula) {
	
	/** Check if all applyto cells contain subformula **/
	
	/** Insert new cells in the appropriate direction **/
	
	/** Set all new cells to contain the subformula formula **/
	
	// Apply the refactoring once per unique R1C1 formula
	foreach (var uniqueR1C1 in applyto.Cells.GroupBy()(c => c.FormulaR1C1)) {
		Cell prototype = uniqueR1C1.First();
		Node AST_or = prototype.Parse();
		Node replacementAST = prototype.Offset[dir].Address().Parse();
		
		prototype.Formula = Replace(AST_or, subformula, replacementAST, /*...*/).Print();

		foreach(var cell in uniqueR1C1) {
			cell.FormulaR1C1 = prototype.FormulaR1C1;
		}
	}
}
\end{lstlisting}

\lstset{style=sharpc}
\begin{lstlisting}[float,caption={Inline Formula Refactoring (simplified)}, label={lst:inlineformula}]
public void InlineFormula(Cell toInline) {
	// Dependendants are gotten from Excel
	var dependents = toInline.dependents;
	if(dependents.Count == 0) {
		// Abort, no dependants
	}
	
	Node AstToRepl = Parse(toInline.Address);
	Node AstReplacement = Parse(toInline.Formula);
	
	// Check if this cell is part of any named ranges with more than one cell
	if(toInline.Names.Any(name => name.Cells.Count > 1)) {
		// Abort
	}
	
	// AST representation of the names
	var names = toInline.Names.Select(Parse);
	
	foreach(var dependent in dependents) {
		Node AstOriginal = Parse(dependent.Formula);
		
		// Abort if to be inlined cell is references as part of a range
		if(CellContainedInRanges(toInlineAddress, AstOriginal)) {
			// Abort
		}
		
		// Replace references to the cell with the value
		var AstNew = Replace(AstOriginal, AstToRepl, AstReplacement, /*...*/);
		foreach(var name in names) {
			AstNew = Replace(AstNew, name, AstReplacement, /*...*/);
		}
		
		dependent.Formula = AstNew.Print();
	}
	
	toInline.Delete();
	
}
\end{lstlisting}

\lstset{style=sharpc}
\begin{lstlisting}[float,caption={Introduce Cell Name (simplified)}, label={lst:introducecellname}]
public void IntroduceName(Range toName, string name) {
		// Check if name already exists
		if(toName.Workbook.Names.Contains(name)) {
			// Abort
		}
		
		toName.Name = name;
		
		var dependents = toInline.dependents;
		if(dependents.Count == 0) {
			// Abort, no dependants
		}
		
		var AstToRepl = Parse(toName.Address);
		var AstReplacement = Parse(name);
		
		foreach(var dependent in dependents) {
			var AstOriginal = Parse(dependent.Formula);
			
			var AstNew = Replace(AstOriginal, AstToRepl, AstReplacement, /*...*/);
			
			dependent.Formula = AstNew.Print();
		}
		
}
\end{lstlisting}

\lstset{style=sharpc}
\begin{lstlisting}[float,caption={Group References Refactoring (simplified)}, label={lst:groupreferences}]
public Node GroupReferences(Node formula) {
	// Get all nodes representing varargs functions
	var targets = formula.AllNodes().Where(IsVarargsFunction);
	
	foreach(Node function in targets) {
		// split arguments that can be grouped from those than can't
		var toGroupArguments = function.arguments
			.Where(node => node.IsCellOrRange());
		var ignoredArguments = function.arguments
			.Where(node => !node.isCellOrRange());
		
		var grouped = new List<Node>();
		
		// Several characteristics define whether references can be grouped
		foreach(var sheetGroup in SplitByWorksheet(toGroupArguments)) {
			foreach(var toGroup in SplitByAbsoluteMarkers(sheetGroup)) {
				grouped.Add(GroupUsingExcel(toGroup));
			}
		}
		
		function.arguments = ignoredArguments.Concat(grouped);
	}
	
	return formula;
}

\end{lstlisting}

\lstset{style=sharpc}
\begin{lstlisting}[float,caption={Introduce Aggregate (simplified)}, label={lst:introduceaggregate}]
public Node IntroduceAggregate(Node formula) {

	// Precondition: formula is a function
	if(!formula.IsFunction()) {
		// Abort
	}
	
	// Precondition: formula is an operator that has an aggregate equivalent
	var op = formula.GetFunctionName();
	if(!AggregateEquivalents.ContainsKey(op)) {
		// Abort
	}
	
	var arguments = new List<Node>();
	var current = formula;
	
	// Gather arguments while the right subtree remains the same operator
	while(current.GetFunctionName() == op){
		arguments.Add(current.LeftArgument);
		current = current.RightArgument;
	}
	arguments.Add(current.RightArgument);
	
	return new Function(AggregateEquivalents[op], arguments);
}

private static Dictionary<string,string> AggregateEquivalents =
	new Dictionary<string,string>() {
		{ "+", "SUM" },
		{ "*", "PRODUCT" },
		{ "&", "CONCATENATE" }
	};

\end{lstlisting}

\lstset{style=sharpc}
\begin{lstlisting}[float,caption={Introduce Conditional Aggregate (simplified)}, label={lst:introduceconditionalaggregate}]
public void IntroduceConditionalAggregate(Cell subject) {

	Node function = Parse(subject.Formula);
	
	// Check if we can perform the refactoring
	if(!IsTargetFunction(function)) {
		if(function.IsFunction() && function.FunctionName == "+") {
			// Rewrite + to SUM
			function = IntroduceAggregate(Ast);
		} else {
			// Abort
		}
	}
	
	// Check if all arguments are references to a single cell
	var arguments = function.arguments;
	if(argument.Any(arg => !IsSingleCellReference(arg))) {
		// Abort
	}
	
	// Logic for all argument in a single row is ommited for brevity,
	// it is identical but transposed.
	
	// Check if all arguments are in a single column
	var summedColumn = arguments.First().Select(cell => cell.Column);
	if(arguments.Select(arg => arg.Column).Any(col => col != summedColumn)) {
		// Abort
	}
	
	var summedRows = argument.Select(cell => cell.Row).ToList();
	
	// Get all non-empty columns in the worksheet
	var columns = GetNonEmptyColumns(subject.Worksheet);
	
	// Find a functional determiner
	// See the listing on the next page for the code of FindDeterminerColumn
	var determiner = FindDeterminerColumn(columns, summedRows);

	if(determiner == null) {
		// Abort, there are no candidate columns
	}
	
	// Create the SUMIF(Column, Value, SummedColumn)
	var AstNew = new Function(function.FunctionName + "IF", new List<Node>() {
			new ColumnRange(determiner.Item1),
			new AstString(determiner.Item2),
			new ColumnRange(summedColumn)
		}
	);
	
	subject.Formula = AstNew.Print();
}
\end{lstlisting}

\begin{lstlisting}[float,caption={FindDeterminerColumn method}, label={lst:finddeterminercolumn}]
private static Tuple<string,string> FindDeterminerColumn(IEnumerable<Column> columns, IEnumerable<Row> summedRows) {
	// Return the first column that is a determiner of the summed column
	return columns.Select(column => {
		// Check if there is a value that is the same in all
		// the corresponding rows of the column
		string candidateValue = column[summedRows.First()].Value;
		
		if(summedRows.Any(row => column[row.ID].Value != candidateValue)) {
			return null;
		}
		
		// Value is the same in all corresponding rows
		// Now check if it is different in all other rows
		if(column.rows.Except(summedRows)
			.Any(cell => cell.Value == candidateValue)) {
			return null;
		}
		
		// We found a candidate column
		return Tuple.Create(column.ID, candidateValue);
	})
	.Where(found => found != null)
	.FirstOrDefault();
}


\end{lstlisting}


%// Find a functional determiner column if there is one
%foreach(var column in columns) {
%	string candidateValue = null;
%	// Check if there is a value that is the same in all rows
%	foreach(var row in summedRows) {
%		if(candidateValue == null) {
%			candidateValue = column[row]; 
%		} elseif (column[row] != candidateValue) {
%		candidateValue = null;
%		break;
%	}
%}
%if(candidateValue != null) {
%	// Check if that value does not occur in another row
%	if(!column.rows.Except(summedRows).Any(val => val == candidateValue)) {
%		// We found a column that can be used
%		FDcolumn = column.ID;
%		FDvalue = candidateValue;
%		break;
%	}
%}
%}