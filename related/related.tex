% !TeX root = ../thesis.tex
% !TeX spellcheck = en_US

\chapter{Previous and related work}
\label{chapter:previouswork}

\section{Refactoring}

Refactoring is ``the process of changing a software system in such a way that it does not alter the external behaviour of the code, yet improves its internal structure'' \cite{opdyke1992refactoring}.
The process is probably as old as programming itself, and was known since at least 1986 as ``Restructuring'' \cite{arnold1986introduction}.
The term ``Refactoring'' was coined by Opdyke in 1992 \cite{opdyke1992refactoring} and originally specifically meant the restructuring of Object Oriented Programs, but nowadays the term is also used in order paradigms.
Over time the popularity of both the practice and the term ``Refactoring'' increased, greatly helped by Fowler's 1999 book ``Refactoring: improving the design of existing code'' \cite{fowler1999refactoring}.
Currently, all major programming Integrated Development Environments like Visual Studio, Eclipse, Netbeans and IntelliJ include support for automated software refactorings.

\section{Refactorings provided by excel}

Several useful refactorings are already provided by Excel, although Excel does not call them refactorings.

The best example of this is the cut, copy and paste functionality of Excel.
If a user cuts a selection of cells and pastes it elsewhere, all references to those cells made in other cells will be moved as well.
For example if a user cut-pastes cell \f{A1} to location \f{C3}, the formula \f{=A1} will be changed to \f{=C3} in other cells, even though they were not selected by the user.
This is very similar to the \rf{Move Method} \cite{fowler1999refactoring} refactoring, because not only are the cells (method contents) themselves moved, references to them (call sites) are a adjusted for the new location as well.
Another option Excel provides when copy-pasting is ``Paste Values'', with which a formula is replaced by its evaluated value and remains constant from that point on.

Excel having built-in support for these refactorings shows that there is a need among spreadsheet users to change their spreadsheets without altering functionality, to refactor them, and that advanced spreadsheet users are likely already comfortable with the concept of refactoring, albeit it not by name.

\section{Spreadsheet refactoring}

In addition to Excel, two commercial Add-Ins are known to the author that implement refactorings.
The Power Utility Pak \cite{PuPv7} contains a \rf{Unapply Names} refactoring, which changes a named range to its location, e.g. \f{=TAX_RATE*100} to \f{=\$A\$1*100}.
This is the exact inverse of the \nameref{refac:introducecellname} refactoring defined in Section \ref{refac:introducecellname}.
It also contains a ``Error Condition Wizard'', which is very similar in purpose to the \rf{Guard Call} refactoring defined by Badame and Dig \cite{badame2012refactoring}.
The ASAP Utilities for Excel \cite{ASAP5} contains a ``Change formula reference style'' function, which overlaps with the \rf{Make Cell Constant Refactoring} defined by Badame and Dig \cite{badame2012refactoring}, and also contains its own version of \rf{Unapply Names}.
The algorithms used and popularity of these Add-Ins is not know, but their existence is additional indication that advanced spreadsheets users recognize the usefulness of refactoring spreadsheets.

The first Excel Add-In specifically developed for refactoring, called RefBook, was presented by Badame and Dig \cite{badame2012refactoring}.
In this tool, seven refactorings are presented, four of which have been re-implemented and improved in this thesis in Chapter \ref{chapter:implementingrefactorings}.

Hermans et. al \cite{hermans2014detecting} present four refactorings designed to each solve a specific spreadsheet smell, which is an indicator that something might be wrong with the spreadsheet.
\rf{Extract (Common) Subformula} is very similar to Refbook's \rf{Extract Row or Column}, and is implemented in this thesis as \nameref{refac:extractformula}.
The proposed \nameref{refac:groupreferences} refactoring is implemented in this thesis, as is the proposed \rf{Merge Formulas} refactoring under the name \nameref{refac:inlineformula}.

Hermans and Dig \cite{hermans2014bumblebee} developed a refactoring tool called ``BumbleBee'' for Excel which operates using a different principle.
Instead of implementing a fixed set of refactorings, it works on the basis of so-called ``transformation rules'' which consist of a two formulas with extended syntax introducing placeholders for certain constructs such as formulas, cells or ranges.
For example the transformation rule \f{=IF($F_1$>$F_2$, $F_1$, $F_2$)} $\leftrightarrow$ \f{MIN($F_1$,$F_2$)} indicates that these two formulas are equivalent to each other and can be transformed into each other.
While this approach can implement a number of refactorings as long as they are contained within one cell, it can only work with the information from within one formula and can thus only be used to implement intra-formula refactorings, refactorings contained within one formula.
This thesis was started as an attempt to introduce BumbleBee with inter-formula refactorings, refactorings which interact with more than one formula.

\section{Parsing Spreadsheet Formulas}

All of the above mentioned tools above process Excel Formulas in one way or another.
However, all of them suffer from one or more of the following problems, which was the motivation to improve the previously proprietary parser used by BumbleBee and release this as open-source, see Chapter \ref{chapter:parsing}.

\subsubsection{Parser is proprietary}

The most obvious case from the Microsoft Excel formula parser itself, which is not available for usages by external programs or Add-Ins.
However, several research projects also do not make available their parser or grammar used, which is the case with Baryowy's et. al CheckCell \cite{barowy2014checkcell} and all of the work by Cunha et. al \cite{cunha2010automatically, cunha2012towards, cunha2012mdsheet}.

\subsubsection{Parser is not advanced enough for refactoring purposes}

While several grammars are available online \cite{ewbi,fishbrain}, and a grammar is published as part of RefBook \cite{badame2012refactoring}, all where found to contain some errors, especially in the areas of operator precedence, reference expressions or references to sheets or files.
These errors were deemed crucial to solve by the thesis Author, as making errors in parsing has a high chance of resulting in errors, and thus violating the user expectation that a refactoring will not introduce errors.

\subsubsection{Parser does integrate well within target environment}

Several open-source programs \cite{libreoffice,calligra} can process Excel formulas in one way or another, however, none of these were written in a .NET language, a requirement to use the parser in an Excel add-in.
These parsers are often deeply tied into the product, which makes using them standalone difficult.
Furthermore these programs do not make their grammar available separately, which would require the grammar to be manually extracted.
While this has not been attempted, this would likely result in grammar suffering from the same deficiencies as the official grammar provided by Microsoft, which is unsuitable for reasons described in Section \label{sec:motivation}.

While it would be possible to decouple the parser from an existing product and port it to a .NET language this likely would have been more difficult than improving the existing parser, which was done for this thesis.