% !TeX root = ../thesis.tex

\chapter{Previous and related work}
\label{chapter:previouswork}

\section{Refactoring}

Refactoring is ``the process of changing a software system in such a way that it does not alter the external behaviour of the code, yet improves its internal structure'' \ref{opdyke1992refactoring}.
The process is probably as old as programming itself, and was known since at least 1986 as ``Restructuring'' \ref{arnold1986introduction}.
The term ``Refactoring'' was coined by Opdyke in 1992 \ref{opdyke1992refactoring} and originally specifically meant the restructuring of Object Oriented Programs. 
Over time the specific the popularity of both the practice and the term ``Refactoring'' increased, greatly helped by Fowler's 1999 ``Refactoring: improving the design of existing code'' \cite{fowler1999refactoring} and the rise of more flexible ``Agile'' software development.
Currently, all major programming Integrated Development Environments like Visual Studio, Eclipse, Netbeans and IntelliJ include support for automated software refactorings.

\section{Refactorings provided by excel}

Several useful refactorings one could think of are already provided by Excel, although Excel does not call them refactorings.

Firstly the cut, copy and paste functionality of Excel is very interesting.
If a user cuts a selection of cells and pastes it elsewhere, all references to those cells made in unselected will be moved as well.
For example if a user cut-pastes cell \f{A1} to location \f{C3}, the formula \f{=A1} will be changed to \f{=C3} in other cells, even though they were not selected by the user.
This is very similar to the \rf{Move Method} \cite{fowler1999refactoring} refactoring, because not only are the cells (method contents) themselves moved, references to them (call sites) are a adjusted for the new location as well.
Another option Excel provides when copy-pasting is ``Paste Values'', with which a formula is replaced by its evaluated value and remains constant from that point on.

Excel having built-in support for these refactorings shows that there is a need among spreadsheet users to refactor their spreadsheets, and that advanced spreadsheet users are likely already comfortable with the concept of refactoring, albeit it not by name.

\section{Spreadsheet refactoring}

\section{The BumbleBee spreadsheet refactoring suite}