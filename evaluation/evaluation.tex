% !TeX root = ../thesis.tex
% !TeX spellcheck = en_US

\chapter{Evaluation}
\label{chapter:evaluation}

\section{Refactorings}

The refactorings implemented in BumbleBee were not formally evaluated as part of this thesis.
Some of the refactorings implemented in this thesis are based on RefBook \cite{badame2012refactoring}, which did perform an evaluation, in which it was found that the more people preferred formulas after the \nameref{refac:extractformula} and \nameref{refac:introduceaggregate} refactorings were applied, while more people preferred the unrefactored formulas after the \nameref{refac:introducecellname} refactoring was applied.


\section{Parser}

In order to evaluate XLParser, it was used to parse all formulas extracted from the two publicly available spreadsheet research datasets, the EUSES corpus \cite{fisher2005euses} and Enron email corpus \cite{klimt2004enron, hermans2015enron}.

A ``scantool'' not developed by the thesis author was used to extract formulas from these two datasets, which succeeded for 19.601 of the 20.688 spreadsheets.
Not all spreadsheets could be read because they were either password-protected or could not be processed by the scantool.
A combined 22.310.406 formulas were found in these spreadsheets.
Duplicate formulas were filtered on a sheet-level by using the formula's R1C1 notation.
This resulted in 1.035.586 unique formulas.
26 formulas were discarded, because they were corrupted by the scan tool, bringing the total to 1.035.558 formulas.
The extracted list of formulas is available as part of XLParser in its repository for both the Enron \footnote{\tiny\url{https://github.com/spreadsheetlab/XLParser/blob/v1.2.1/src/XLParser.Tests/data/enron/formulas.txt}} and the EUSES \footnote{\tiny\url{https://github.com/spreadsheetlab/XLParser/blob/v1.2.1/src/XLParser.Tests/data/euses/formulas.txt}} datasets.

Of these formulas, exactly two could not be parsed: \f{=-NOX, Regi} and \f{-_SO2, regi}, a successful parse rate of 99.999\%.
The unparsable formulas are examples of unions without brackets which the parser cannot parse due to reasons outlined in Section \ref{subsec:desing:unions}.
 
\newpage 
 
It must be noted that both datasets used only contain spreadsheets created before 2005.
This means that they do not contain structured table references, as these were introduced in Excel 2007.
It thus remains unevaluated how well the parser handles formulas containing these constructs.

Another deficiency is that the scantool extracts formulas as they are stored inside the spreadsheet file.
When a formula references an external file, in the spreadsheet file the filename is stored in a separate lookup array and the file is replaced by a numeric index in the stored formula.
Thus a formula like \f{='C:\textbackslash path\textbackslash [file.xlsx]Sheet!A1'} would be stored as \f{=[1]Sheet!A1}.
However, Excel Add-Ins like BumbleBee receive the file name and path in the formula when requested from Excel.
It is thus possible that the parser cannot handle all formulas formulas containing external references in their non-numeric file reference form.
However, a member of the spreadsheet lab who used the parser to process the same two datasets reports using their non-numeric file reference form reports no additional parse errors, except in the case of a file path containing a space, which is a problem that most likely can be solved.

\subsection{Analysis}

For all the formulas extracted from the Enron and EUSES datasets a count was performed of which tokens (terminals, leaf nodes of the parse tree) and which production rules (non-terminal, internal nodes of the parse tree) were used when parsing the formula.

At least one reference was used in 99.2\% of the formulas, which shows that references are an integral part of spreadsheets.
Arithmetic operators are also very common (59.77\%), as is the use of built-in excel functions (35.82\%).

Perhaps more interesting are the less commonly used grammatical constructs: Empty arguments, Dynamic Data Exchange calls, Intersections and Unions, Multiple Sheets, Reserved Names and constant arrays.
All of these constructs are used in less than 0.05\% of all formulas.

The rarity of empty argument, DDE calls, reserved names and multiple sheets can be expected because of their very niche uses.
Intersections, unions and constant arrays are powerful constructs, but apparently users either do not know about these, prefer to use other constructs instead or their usage is too niche.

Interestingly, some of these constructs are quite hard to parse.
Especially unions and intersections complicate the grammar because of their curious syntax.
Because these constructs are so rare this means they are prime candidates to leave out of the grammar if simplification is desired.

\begin{table}
	\centering
	\resizebox{\columnwidth}{!}{%
	\begin{tabular}{llrrrrr}
		\hline
		Syntax & Example &  \multicolumn{2}{c}{Unique formulas} & \multicolumn{2}{c}{Total formulas} \\
		\hline
			\synt{Formula} & \texttt{=1+2} &  & \textbf{1,035,586} & & \textbf{22,310,406} & \\
			\synt{Reference} & \texttt{=E9/E10} &  & 962,783 & 92.97\% & 22,131,002 & \textbf{99.20\%}\\
			CELL & \texttt{=A5} &  & 951,521 & 91.88\% & 22,021,833 & 98.71\%\\
			\synt{FunctionCall} & \texttt{=SUM(A5:A22)} &  & 701,626 & 67.75\% & 18,944,204 & \textbf{84.91\%}\\
			\synt{BinOp} & \texttt{=H10-H8} &  & 397,580 & 38.39\% & 13,333,844 & \textbf{59.77\%}\\
			\synt{Constant} & \texttt{=A5+134} &  & 271,585 & 26.23\% & 8,731,489 & \textbf{39.14\%}\\
			EXCEL-FUNCTION & \texttt{=SUM(A5:A22)} &  & 264,673 & 25.56\% & 7,991,329 & 35.82\%\\
			NUMBER & \texttt{=(B8/48)*15} &  & 250,085 & 24.15\% & 7,849,495 & \textbf{35.18\%}\\
			\synt{Prefix} & \texttt{=Sheet1!B1} &  & 337,727 & 32.61\% & 5,599,011 & \textbf{25.10\%}\\
			\synt{RefFunctionName} & \texttt{=SUM(J9:INDEX(J9:J41,B43))} &  & 55,680 & 5.38\% & 5,349,237 & \textbf{23.98\%}\\
			SHEET & \texttt{=Sheet1!B1} &  & 303,981 & 29.35\% & 5,282,386 & 23.68\%\\
			REF-FUNCTION-COND & \texttt{=IF(A1<0,0,1)} &  & 50,171 & 4.84\% & 4,872,661 & \textbf{21.84\%}\\
			\synt{Reference} ':' \synt{Reference} & \texttt{=SUM(A5:A22)} &  & 184,451 & 17.81\% & 3,735,005 & \textbf{16.74\%}\\
			\synt{UnOpPrefix} & \texttt{=+B11+1} &  & 218,397 & 21.09\% & 3,289,326 & 14.74\%\\
			STRING & \texttt{=IF(AD3<0,"buy","sell")} &  & 56,635 & 5.47\% & 2,587,971 & \textbf{11.60\%}\\
			\synt{NamedRange} & \texttt{=SUM(freq)} &  & 20,686 & 2.00\% & 1,645,120 & \textbf{7.37\%}\\
			BOOL & \texttt{=IF(AND(R11=1,R14=TRUE),G19,0)} &  & 7,532 & 0.73\% & 1,183,798 & 5.31\%\\
			FILE & \texttt{=[11]Sheet1!C5} &  & 104,892 & 10.13\% & 1,135,185 & \textbf{5.09\%}\\
			REF-FUNCTION & \texttt{=SUM(J9:INDEX(J9:J41,B43))} &  & 9,907 & 0.96\% & 778,056 & \textbf{3.49\%}\\
			
			SHEET_QUOTED & \texttt{=('[2]Detail I\&E'!D62)/1000} &  & 33,781 & 3.26\% & 325,498 & 1.46\%\\
			UDF & \texttt{=SQRT(_eoq2(C5,C4,C6,C7))} &  & 21,352 & 2.06\% & \textbf{286,210} & \textbf{1.28\%}\\
			(' \synt{Reference} ')' & \texttt{=(2*(B29))/(1+B29)} &  & 6,394 & 0.62\% & 266,420 & 1.19\%\\
			_xll. & \texttt{=_xll.RiskTriang(F9,F7,F8)} &  & 11,922 & 1.15\% & 127,348 & \textbf{0.57\%}\\
			ERROR-REF & \texttt{=AVERAGE(\#REF!)} &  & 3,477 & 0.34\% & 123,447 & \textbf{0.55\%}\\
			VERTICAL-RANGE & \texttt{=COUNT(A:A)} &  & 851 & 0.08\% & 55,254 & \textbf{0.25\%}\\
			FILE '!' & \texttt{=[1]!today} &  & 2,040 & 0.20\% & \textbf{28,448} & \textbf{0.13\%}\\
			ERROR & \texttt{=IF(R14=TRUE,G19,\#N/A)} &  & 379 & 0.04\% & 27,237 & \textbf{0.12\%}\\
			'\%' & \texttt{=IF(E5>I8,3\%,0\%)} &  & 858 & 0.08\% & 16,606 & 0.07\%\\
				Empty argument & \texttt{=DCOUNT(Lettergrades,,I80:I81)} &  & 1,343 & 0.13\% & 10,512 & \textbf{0.05\%}\\
				DDECALL & \texttt{=TWINDDE|RSFRec!'NGH2 NET.CHNG'} &  & 3,276 & 0.32\% & \textbf{3,686} & 0.02\%\\
				Intersection & \texttt{=Ending_Inventory Jan} &  & 304 & 0.03\% & \textbf{2,343} & 0.01\%\\
				MULTIPLE-SHEETS & \texttt{=SUM(Sheet1:Sheet20!I29)} &  & 173 & 0.02\% & \textbf{1,986} & \textbf{0.01\%}\\
				UDF reference & \texttt{=[1]!wbname()} &  & 332 & 0.03\% & \textbf{855} & 0.00\%\\
				HORIZONTAL-RANGE & \texttt{=MATCH(F3,Prices!2:2,0)} &  & 11 & 0.00\% & 836 & \textbf{0.00\%}\\
				\synt{Union} & \texttt{=LARGE((F38,C38),1)} &  & 10 & 0.00\% & \textbf{385} & 0.00\%\\
				RESERVED_NAME & \texttt{=C23/_xlnm.Print_Area} &  & 70 & 0.01\% & \textbf{276} & 0.00\%\\
				FILE MULTIPLE-SHEETS & \texttt{=SUM([2]Section3A:formulas!B11)} &  & 4 & 0.00\% & 189 & 0.00\%\\
				\synt{ConstantArray} & \texttt{=FVSCHEDULE(1,{0.09;0.11;0.1})} &  & 15 & 0.00\% & \textbf{19} & 0.00\%\\
			\hline
	\end{tabular}%
	}
	\caption{Occurrence frequency of tokens and production rules in all parsed formulas.}
	\label{table:occurences}
\end{table}