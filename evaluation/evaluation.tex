% !TeX root = ../thesis.tex

\chapter{Evaluation}
\label{chapter:evaluation}

Two parts of this thesis can be evaluated, the stand-alone parser and the new refactorings implemented in BumbleBee.

\section{Refactorings}

The refactorings implemented in BumbleBee were not formally evaluated as part of this thesis.
Some of the refactorings implemented in this thesis are based on RefBook \cite{badame2012refactoring}, which did perform an evaluation, in which it was found that the more people preferred formulas after the \nameref{refac:extractformula} and \nameref{refac:introduceaggregate} refactorings were applied, while more people preferred the unrefactored formulas on which \nameref{refac:introducecellname} was applied.

\section{Parser}

In order to evaluate XLParser, it was used to parse all formulas extracted from the two publicly available spreadsheet research datasets, the EUSES corpus \cite{fisher2005euses} and Enron email corpus \cite{klimt2004enron, hermans2015enron}.

A ``scantool'' not developed by the thesis author was used to extract formulas from these two datasets, which succeeded for 19.601 of the 20.688 spreadsheets.
Not all spreadsheets could be read because they were either password-protected or could not be processed by the scantool.
A combined 22.310.406 formulas were found in these spreadsheets.
Duplicate formulas were filtered on a sheet-level by using the formula's R1C1 notation.
This resulted in 1.035.586 unique formulas.
26 formulas were discarded, because they were corrupted by the scan tool, bringing the total to 1.035.558 formulas.
The extracted list of formulas is available as part of XLParser in its repository for both the Enron \footnote{\url{https://github.com/spreadsheetlab/XLParser/blob/master/src/XLParser.Tests/data/enron/formulas.txt}} and the EUSES \footnote{\url{https://github.com/spreadsheetlab/XLParser/blob/master/src/XLParser.Tests/data/euses/formulas.txt}} datasets.

Of these formulas, exactly two could not be parsed: \f{=-NOX, Regi} and \f{-_SO2, regi}, which indicates a parse success rate of 
These are examples of unions without brackets which the parser cannot parse due to reasons outlined in Section \ref{subsec:desing:unions}.

\subsection{Analysis}

