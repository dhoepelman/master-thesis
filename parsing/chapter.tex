\chapter{Parsing spreadsheet formulas}

\todo{Notificatie dat deze sectie deels al gepubliceerd is (niet hopen)}

The formula parser component of Bumblebee is called XLParser which is the reference implementation of the grammar by by Aivaloglou, Hoepelman and Hermans \cite{xlparser}.
This component is fully independent, open-source and freely available \footnote{https://github.com/PerfectXL/XLParser}.

XLParser takes a formula and returns a parse tree.
While technically not strictly an AST as it retains several lexical elements, it is also not .
We will indicate the tree returned by XLParser as an AST from here on.

XLParser is built using the Irony Parser framework \footnote{https://irony.codeplex.com}, which uses a grammar defined in C\# to produce a parser based on the LALR(1) algorithm and .NET regular expressions for tokens.

The parser has evolved from previous research efforts, and is intended to be used for ongoing and future research into Microsoft Excel spreadsheet formulas.
With this intent in mind, the following design goals were formulated:

\begin{enumerate}
\label{sec:designgoals}
\item The parser must be compatible with the official language
\item Produced parse trees must be suited for further manipulation and analysis with minimal post-processing required
\item The grammar must be compact enough to feasibly implement with a parser generator
\end{enumerate}

While an official grammar for Excel formulas is published \cite{ExcelOfficialGrammar}, it does not meet the above requirements for two reasons.
Firstly, it is over 30 pages long and contains hundreds of production rules and thus fails requirement 3.
Secondly, because of the detail of the grammar and the large number of production rules the resulting parse trees are very complex and fail requirement 2.

\section{Lexical Analysis}
\label{sec:lexanalysis}

\begin{table}
\tiny
\caption{Lexical tokens used in the XLParser grammar, as refered to in section \ref{sec:lexanalysis}.}
\label{table:tokens}
\centerfloat
%\advance\leftskip-1cm
\stepcounter{footnote}

\begin{tabular}{@{}llll@{}}
	\toprule
	Token Name & Description & Contents & Priority \\
	\midrule
	%BINOP & Binary Operator & + $\mid$ - $\mid$ / $\mid$ * $\mid$ \textasciicircum $\mid$ \textless $\mid$ \textgreater $\mid$ = $\mid$ \textless= $\mid$ \textgreater= $\mid$ \textless \textgreater & 0 & + \\
	BOOL & Boolean literal & TRUE $\mid$ FALSE & 0 \\
	CELL & Cell reference & \$? [A-Z]+ \$? [0-9]+ & 2 \\
	DDECALL & Dynamic Data Exchange link & ' ([\textasciicircum{} '] $\mid$ '')+ ' & 0 \\
	ERROR & Error literal & \begin{tabular}[c]{@{}l@{}} \#NULL! $\mid$ \#DIV/0! $\mid$ \#VALUE! \\ $\mid$ \#NAME? $\mid$ \#NUM! $\mid$ \#N/A \end{tabular} & 0 \\
	ERROR-REF & Reference error literal & \#REF! & 0 \\
	EXCEL-FUNCTION & Excel built-in function & (Any entry from the function list\begin{footnotesize}\textsuperscript{\number\value{footnote}}\end{footnotesize}) \textbackslash( & 5        \\
	FILE & External file reference using number & \textbackslash[ [0-9]+ \textbackslash] & 5 \\
	FILENAME & External file reference using name & \textbackslash[ $\Box_4$+ \textbackslash] & -1 \\
	FILEPATH & Windows file path & [A-Z] : \textbackslash\textbackslash\ ($\Box_4$+ \textbackslash\textbackslash)* & 0 \\
	HORIZONTAL-RANGE & Range of rows & \$? [0-9]+ : \$? [0-9]+ & 0 \\
	MULTIPLE-SHEETS & Multiple sheet references & 
	%$\Box_2$+ : ($\Box_2$+ $\mid$ '($\Box_3$ $\mid$ '')+')!
	(($\Box_2$+ : $\Box_2$+)|( ' ($\Box_3$ $\mid$ '')+ : ($\Box_3$ $\mid$ '')+ ' )) !
	& 1 \\
	NAME & User Defined Name & [A-Z\_\textbackslash\textbackslash][A-Z0-9\textbackslash\textbackslash\_.$\Box_1$]* & -2 \\
	NAME-PREFIXED & \begin{tabular}[c]{@{}l@{}} User defined name which starts with \\ a string  that could be another token \end{tabular} & (TRUE $\mid$ FALSE $\mid$ [A-Z]+[0-9]+)    {[}A-Z0-9\_.$\Box_1${]}+                                                                             & 3 \\
	NUMBER & \begin{tabular}[c]{@{}l@{}}An integer, floating point\\     or scientific notation number literal\end{tabular} & [0-9]+ ,? [0-9]* (e [0-9]+)? & 0 \\
	REF-FUNCTION & Excel built-in reference function & (INDEX $\mid$ OFFSET $\mid$ INDIRECT)\textbackslash( & 5 \\
	REF-FUNCTION-COND & Excel built-in conditional reference function & (IF $\mid$ CHOOSE)\textbackslash( & 5 \\
	RESERVED-NAME & An Excel reserved name & \_xlnm\textbackslash.  [A-Z\_]+ & -1 \\
	SHEET & The name of a worksheet & $\Box_2$+ ! 
	& 5        \\
	SHEET-QUOTED & Quoted worksheet name & $\Box_3$+ ' ! & 5 \\
	STRING & String literal & " ([\textasciicircum{} "] $\mid$ "")* " & 0       \\
	SR-COLUMN & Structured reference column & \textbackslash[ [A-Z0-9\textbackslash\textbackslash\_.$\Box_1$]+ \textbackslash] & -3 \\
	%UNOP\_POSTFIX & Unary postfix operator & \% & 0 & \% \\
	%UNOP\_PREFIX & Unary prefix operator & + $\mid$ - & 0 & -                  \\
	UDF & User Defined Function & (\_xll\textbackslash.)? [A-Z\_\textbackslash][A-Z0-9\_\textbackslash\textbackslash.$\Box_1$]*  ( & 4 \\
	VERTICAL-RANGE & Range of columns & \$? [A-Z]+ : \$? [A-Z]+ & 0 \\ 
	\midrule
	Placeholder character & Placeholder for & Specification & \\
	$\Box_1$ & Extended characters & \begin{tabular}[c]{@{}l@{}} Non-control Unicode characters x80 and up \end{tabular} & \\
	$\Box_2$ & Sheet characters & \begin{tabular}[c]{@{}l@{}} Any character except\\ ' * [ ] \textbackslash\ : / ? ( ) ; \{ \} \# " = < > \& + - * / \textasciicircum{} \% , \texttt{\char32} \end{tabular}& \\
	$\Box_3$ & Enclosed sheet characters & \begin{tabular}[c]{@{}l@{}} Any character except ' * [ ] \textbackslash\ : / ? \end{tabular} & \\
	$\Box_4$ & Filename characters & Any character except " * [ ] \textbackslash\ : / ? < > $\mid$ \\
	\bottomrule
	\multicolumn{4}{l}{\begin{footnotesize}\textsuperscript{\number\value{footnote}} A function list is available as part of the reference implementation.\end{footnotesize}}\\
	\multicolumn{4}{l}{\begin{footnotesize} \hspace{0.5em} Lists provided by Microsoft are also available in \cite{ExcelFunctionReference} and \cite{ExcelOfficialGrammar}.\end{footnotesize}}\\
	
\end{tabular}

\stepcounter{footnote}
\end{table}

\todo{Move table to appendix}

Table \ref{table:tokens} contains the lexical tokens of the grammar, along with their identification patterns in a simple regular expression language. All tokens are case-insensitive.

This token list requires the scanner to support token priorities, which the Irony framework does.
While it is possible to remove this necessity by altering the tokens and production rules, this would make the resulting grammar and AST harder to use.

Some simple tokens are directly defined in the production rules in Figure \ref{figure:productions} in between quotes for readability and compactness.

\section{Syntactical Analysis}

\begin{figure}
\small
\caption{Production rules}
\label{figure:productions}
\begin{multicols*}{2}
\begin{grammar}
	<Start> ::= <Constant>
	\alt '=' <Formula>
	\alt `\{=' <Formula> `\}'
	
	<Formula> ::= <Constant>
	\alt <Reference>
	\alt <FunctionCall>
	\alt `(' <Formula> `)'
	\alt <ConstantArray>
	\alt "RESERVED-NAME"
	
	<Constant> ::= "NUMBER" | "STRING" | "BOOL" | "ERROR"
	%         \alt "STRING"
	%         \alt "BOOL"
	%         \alt "ERROR"
	
	<FunctionCall> ::= 
	<UnOpPrefix> <Formula>
	\alt <Formula> `\%'
	\alt <Formula> <BinOp> <Formula>
	\alt "EXCEL-FUNCTION" <Arguments> `)'
	
	<UnOpPrefix> ::= `+' | `-'
	
	<BinOp> ::= `+' | `-' | `*' | `/' | `\textasciicircum'
	\alt `<' | `>' | `=' | `<=' | `>=' | `<>'
	
	%<Function> ::=  | "UDF"
	
	<Arguments> ::= $\epsilon$ \alt <Argument> \{ `,' <Argument> \}
	
	<Argument> ::= <Formula> | $\epsilon$
	
	<Reference> ::= <ReferenceItem>
	\alt <RefFunctionCall>
	\alt `(' <Reference> `)' 
	\alt <Prefix> <ReferenceItem>
	%\alt <Prefix> "UDF" <Arguments> `)'
	%    \alt <DynamicDataExchange>
	\alt "FILE" `!' "DDECALL"
	
	%<NamedRange> ::= "NAMED-RANGE"
	%            \alt "NAMED-RANGE-PREFIXED"
	
	<RefFunctionCall> ::= `(' <Union> `)'
	\alt <RefFunctionName> <Arguments> `)'
	\alt <Reference> `:' <Reference>
	\alt <Reference> `\ ' <Reference>
	
	<ReferenceItem> ::= "CELL"
	\alt <NamedRange>
	\alt <StructuredReference>
	\alt "VERTICAL-RANGE"
	\alt "HORIZONTAL-RANGE"
	\alt "UDF" <Arguments> `)'
	\alt "ERROR-REF"
	
	\columnbreak
	
	<NamedRange> ::= <Name>
		
	<Name> ::= "NAME" | "NAME-PREFIXED"
	
	<File> ::= FILE
	\alt FILENAME
	\alt FILEPATH FILENAME
	
	<Prefix> ::= "SHEET"
	\alt `'' "SHEET-QUOTED"
	\alt <File> "SHEET"
	\alt `'' <File> "SHEET-QUOTED"
	\alt "FILE" `!'
	\alt "MULTIPLE-SHEETS"
	\alt <File> "MULTIPLE-SHEETS"
	
	<RefFunctionName> ::= "REF-FUNCTION"
	\alt "REF-FUNCTION-COND"
	
	<Union> ::= <Reference> \{ `,' <Reference> \}
	
	%<DynamicDataExchange> ::= "FILE" `!' "DDECALL"
	
	<ConstantArray> ::= `\{' <ArrColumns> `\}'
	
	<ArrColumns> ::= <ArrRows> \{ `;' <ArrRows> \}
	
	<ArrRows> ::= <ArrConst> \{ `,' <ArrConst> \}
	
	<ArrConst> ::= <Constant>
	\alt <UnOpPrefix> "NUMBER"
	\alt "ERROR-REF"
	
	<StructuredReference> ::= <SRCol>
	\alt `[' <SRExpr> `]'
	\alt <Name> <SRCol>
	\alt <Name> `[' <SRExpr> `]'
	
	<SRExpr> ::= <SRCol>
	\alt <SRCol> `:' <SRCol>
	\alt <SRCol> `,' <SRCol>
	\alt <SRCol> `,' <SRCol> `:' <SRCol>
	\alt <SRCol> `,' <SRCol> `,' <SRCol>
	\alt <SRCol> `,' <SRCol> `,' <SRCol> `:' <SRCol>
	
	<SRCol> ::= FILENAME
	\alt `[' <Name> `]'  
	\alt `[' SR-COLUMN `]'  
	
	
	
\end{grammar}
\end{multicols*}
\end{figure}

The complete production rules of the used grammar can be found in Figure \ref{figure:productions} in Extended BNF syntax.
The start symbol is $Start$.

\synt{Formula} and \synt{Reference} are the two most important production rules in this grammar.
They model respectively all expressions inside formulas, and expressions consisting purely of references.
This distinction is important because in some places in the language only reference expression can be used.
An example of this is the the range operator \f{:} which only accepts reference expressions as operands.
Thus a formula like \f{=1.5:2.5} will result in a parse error in both Microsoft Excel and XLParser, while a formula like \f{="a" + 2} will result in a runtime type system error.

The \synt{Formula} and \synt{Reference} production rules are also illustrated as syntax diagrams, with most production rules expanded, in Figures \ref{figure:Formula} and \ref{figure:Reference}.

\todo{Move these figures to appendix?}

\begin{figure}
	\caption{Syntax diagram of the \synt{Formula} production rule with most production rules expanded}
	\label{figure:Formula}
	%\centerfloat
	\input{parsing/formula-diagram.tex}
\end{figure}

\begin{figure}
	\caption{Syntax diagram of the \synt{Reference} production rule with most production rules expanded}
	\label{figure:Reference}
	\begin{grammar}
		<Reference> ::= \begin{syntdiag}[\scriptsize\sdlengths]
		\begin{stack} '$($' \begin{rep} <Reference> \\  '$,$' \end{rep} '$)$'\\ <Reference> \begin{stack} '$:$' \\ '\char32' \end{stack} <Reference> \\
		\begin{stack} \\ \begin{stack} \\ "FILE" \end{stack} \begin{stack} "SHEET" \\ "MULTIPLE_SHEETS" \end{stack} \\ "FILE" '$!$' \\ "QUOTED_FILE_SHEET" \end{stack}
		\begin{stack} \begin{stack} "CELL" \\ "VERTICAL-RANGE" \\ "HORIZONTAL-RANGE" \\ \begin{stack} "NAMED-RANGE" \\ "NAMED-RANGE-COMBINED" \end{stack} \\ "ERROR-REF" \end{stack} \\  \begin{stack} "REFERENCE-FUNCTION" \\ "UDF" \end{stack} \begin{rep} \begin{stack} \\ <Formula> \end{stack} \\  '$,$' \end{rep} '$)$'\end{stack}
		\\"FILE" '$!$' "DDECALL"
		\end{stack}
		\end{syntdiag}
	\end{grammar}
\end{figure}

\section{Precedence and ambiguity}

\begin{table}
\small
\caption{Operator precedence in formulas}
\label{table:operatorprec}
\begin{tabular}{lll}
Precedence & Operator(s) & Associativity \\
1 & = \textless \  \textgreater \  \textless= \  \textgreater= \  \textless\textgreater & Left         \\
2 & \& & Left \\
3 & + - (binary) & Left \\
4 & $\ast$ / & Left \\
5 & \textasciicircum & Left\footnotemark \\
6 & \% & \\
7 & + - (unary) & \\
8 & : , \texttt{\char32} & Left
\end{tabular}
\end{table}

\footnotetext{This is contrary to most other languages, where the exponentiation operator is right-associative. \\
In Excel \f{2\textasciicircum 1\textasciicircum 2} will be $(2^1)^2 = 4$, while in most other languages it will be $2^{1^2} = 2$}

The production rules are ambiguous, which means they cannot be directly used in a parser generator based on the LALR algorithm like Irony.

To resolve ambiguity with operators, e.g. whether to parse \f{=1+2*3} as \f{=(1+2)*3} or \f{=1+(2*3)}, operator precedence and associativity rules are defined.
These can be found in table \ref{table:operatorprec}.

However, even with precedence and associativity rules the grammar is still not fully un-ambigious.
This is due to trade-offs on parsing references, see section \ref{tradeoff:references}, and parsing unions (see section \ref{subsec:desing:unions}).
Ambiguity exists between the following production rules:
\begin{enumerate}
\item \begin{grammar}<Reference> ::= `(' <Reference> `)'\end{grammar}
\item \begin{grammar}<Reference> ::= `(' <Union> `)'\end{grammar}
\item \begin{grammar}<Formula> ::= `(' <Formula> `)'\end{grammar}
\end{enumerate}

A formula like \texttt{=(A1)} can be interpreted as either a bracketed reference, a union of one reference, or a reference within a bracketed formula.

In a LALR like the one Irony produces this ambiguity manifests in a state where, on a \texttt{')'} token, shifting on rule 1 and reducing on either rule 2 or 3 are possibilities, causing a shift-reduce conflict.
This was solved by instructing the parser generator to shift on rule 1 in case of this conflict, because this always is a correct interpretation and thus results in correct ASTs.

\section{Trade-offs and quirks}

The grammar presented in this chapter (\todo{section?}) contains some quirks on sub-optimalities, partly due to the Excel language itself, partly due to design decisions.
These are detailed here

\subsection{\textbf{References}}
\label{tradeoff:references}

In some places only reference expressions are accepted, instead of all expressions.
Furthermore references are of great importance in spreadsheet formulas, and thus of interest for analysis.
To support easier analysis (design goal 2) references have different production rules than other expressions.
This causes references to be easily identified and isolated, but has the downside of increasing ambiguity, as explained in Section \ref{sec:ambiguity}. 

Another approach would be to parse all expressions with the same production rules and implement a type system, however this would be very detrimental to both ease of analysis (design goal 2) and to ease of implementation (design goal 3).

\subsection{\textbf{Unions}}
\label{subsec:desing:unions}

The comma serves both as an union operator and a function argument separator.
This proves challenging to correctly implement in a LALR(1) grammar.

A straightforward implementation would use production rules similar to this:
\begin{grammar}
<Union> ::= <Reference> `,' <Reference>

<Arguments> ::= <Argument>
	\alt <Argument> `,' <Arguments>
\end{grammar}

However, this will cause a reduce-reduce conflict because the parser will have a state wherein it can reduce to both a \synt{Union} or \synt{Argument} on a \texttt{,} token.
Unfortunately there is no correct choice: in a formula like \texttt{=SUM(A1,1)} the parser must reduce on the \synt{Argument} nonterminal, while in a formula like \texttt{=A1,A1} the parser must reduce to the \synt{Union} nonterminal.
With the above production rules a LALR(1) parser could not correctly parse the language.

The presented grammar only parsers unions in between parentheses, e.g. \texttt{=SMALL((A1,A2),1)}.
This is a trade-off between a lower compatibility (design goal 1) and an easier implementation (design goal 3).
Because unions are very rare \cite{grammarpaper} this lower compatibility is deemed acceptable.

Additionally formulas that this grammar does not parse often result in runtime errors after evaluation.
For example \texttt{=A1,A1} does parse in a spreadsheet program, but produces the error \texttt{\#VALUE!} on evaluation.

Implementing the straightforward rules above, while desirable, is not possible without using a more powerful grammar class.