\chapter{Parsing spreadsheet formulas}
\label{chapter:parsing}

\emph{\small{Parts of this chapter have been published in "A grammar for Spreadsheet Formulas Evaluated on Two Large Datasets" by Aivaloglou, Hoepelman and Hermans\cite{xlparser}.}}
\vspace{1em}

This chapter assumes the reader is familiar with basic compiler construction theory.

\section{Motivation}

In order to implement refactorings of spreadsheets a refactoring tool must be able to manipulate spreadsheets.
Excel exposes an API to retrieve and set the contents of a spreadsheet file, but this API only works on a formula string level.
The usual way to implement refactorings is by manipulating the AST so that it represents the desired program, and then print that back to a string.
Because Excel does not expose the AST or parse tree of formulas, a refactoring tool needs to contain a parser for Excel formulas.

Bumblebee relied on a parser developed over the years for previous research, however this parser suffered from having new rules added on over time, often in an inconsistent manner and not fully supporting the whole language.
Furthermore the parser interpreted some language constructs wrong and missed several features, which made implementing refactorings hard.
For example operator precedence was not taken into account, causing the formula \f{=A1 + A2 * A3} was parsed as \f{=(A1 + A2) * A3}, which breakes the "Plus to SUM" refactoring.
The existing parser thus was insufficient to implement the Bumblebee refactorings.
A better parser was needed, thus the goal of this thesis shifted toward procuring a better parser for Excel formulas.

For Bumblebee and other research on spreadsheet formulas the following requirements about the parser and grammar were formulated:

\begin{enumerate}
\label{sec:designgoals}
\item The parser must be compatible with the official language
\item Produced parse trees must be suited for further manipulation and analysis with minimal post-processing required
\item The grammar must be compact enough to feasibly implement with a parser generator
\end{enumerate}

While an official grammar for Excel formulas is published \cite{ExcelOfficialGrammar}, it does not meet the above requirements for two reasons.
Firstly, it is over 30 pages long and contains hundreds of production rules and thus fails requirement 3.
Secondly, because of the detail of the grammar and the large number of production rules the resulting parse trees are very complex and fail requirement 2.

Because there is no suitable parser and grammar available that satisfy the above requirements, I decided to clean up and partially rewrite the parser in coorporation with other members of the Spreadsheet lab.
The end result of this effort is an independent, open-source parser for Excel formulas called XLParser\footnote{https://github.com/PerfectXL/XLParser}, about which a paper was published in IEEE conference SCAM 2015 \cite{xlparser}.

\section{Parser implementation}

The existing parsing was built upon the Irony\footnote{https://irony.codeplex.com/}, which is a C\# parser generator that procudes parsers based on the LALR(1) algorithm using a grammar defined in C\#.

Strictly speaking Irony produces a parse tree, however this tree is fairly abstract, leaving out elements such as punctuation and whitespace, and no separate AST is constructed in Bumblebee, this tree is instead directly manipulated.
To avoid confusion and use usual nomenclature I will from now on refer to this tree as the AST.

\subsection{Lexical Analysis}
\label{sec:lexanalysis}

\begin{table}
\tiny
\caption{Lexical tokens used in the XLParser grammar, as refered to in section \ref{sec:lexanalysis}.}
\label{table:tokens}
\centerfloat
%\advance\leftskip-1cm
\stepcounter{footnote}

\begin{tabular}{@{}llll@{}}
	\toprule
	Token Name & Description & Contents & Priority \\
	\midrule
	%BINOP & Binary Operator & + $\mid$ - $\mid$ / $\mid$ * $\mid$ \textasciicircum $\mid$ \textless $\mid$ \textgreater $\mid$ = $\mid$ \textless= $\mid$ \textgreater= $\mid$ \textless \textgreater & 0 & + \\
	BOOL & Boolean literal & TRUE $\mid$ FALSE & 0 \\
	CELL & Cell reference & \$? [A-Z]+ \$? [0-9]+ & 2 \\
	DDECALL & Dynamic Data Exchange link & ' ([\textasciicircum{} '] $\mid$ '')+ ' & 0 \\
	ERROR & Error literal & \begin{tabular}[c]{@{}l@{}} \#NULL! $\mid$ \#DIV/0! $\mid$ \#VALUE! \\ $\mid$ \#NAME? $\mid$ \#NUM! $\mid$ \#N/A \end{tabular} & 0 \\
	ERROR-REF & Reference error literal & \#REF! & 0 \\
	EXCEL-FUNCTION & Excel built-in function & (Any entry from the function list\begin{footnotesize}\textsuperscript{\number\value{footnote}}\end{footnotesize}) \textbackslash( & 5        \\
	FILE & External file reference using number & \textbackslash[ [0-9]+ \textbackslash] & 5 \\
	FILENAME & External file reference using name & \textbackslash[ $\Box_4$+ \textbackslash] & -1 \\
	FILEPATH & Windows file path & [A-Z] : \textbackslash\textbackslash\ ($\Box_4$+ \textbackslash\textbackslash)* & 0 \\
	HORIZONTAL-RANGE & Range of rows & \$? [0-9]+ : \$? [0-9]+ & 0 \\
	MULTIPLE-SHEETS & Multiple sheet references & 
	%$\Box_2$+ : ($\Box_2$+ $\mid$ '($\Box_3$ $\mid$ '')+')!
	(($\Box_2$+ : $\Box_2$+)|( ' ($\Box_3$ $\mid$ '')+ : ($\Box_3$ $\mid$ '')+ ' )) !
	& 1 \\
	NAME & User Defined Name & [A-Z\_\textbackslash\textbackslash][A-Z0-9\textbackslash\textbackslash\_.$\Box_1$]* & -2 \\
	NAME-PREFIXED & \begin{tabular}[c]{@{}l@{}} User defined name which starts with \\ a string  that could be another token \end{tabular} & (TRUE $\mid$ FALSE $\mid$ [A-Z]+[0-9]+)    {[}A-Z0-9\_.$\Box_1${]}+                                                                             & 3 \\
	NUMBER & \begin{tabular}[c]{@{}l@{}}An integer, floating point\\     or scientific notation number literal\end{tabular} & [0-9]+ ,? [0-9]* (e [0-9]+)? & 0 \\
	REF-FUNCTION & Excel built-in reference function & (INDEX $\mid$ OFFSET $\mid$ INDIRECT)\textbackslash( & 5 \\
	REF-FUNCTION-COND & Excel built-in conditional reference function & (IF $\mid$ CHOOSE)\textbackslash( & 5 \\
	RESERVED-NAME & An Excel reserved name & \_xlnm\textbackslash.  [A-Z\_]+ & -1 \\
	SHEET & The name of a worksheet & $\Box_2$+ ! 
	& 5        \\
	SHEET-QUOTED & Quoted worksheet name & $\Box_3$+ ' ! & 5 \\
	STRING & String literal & " ([\textasciicircum{} "] $\mid$ "")* " & 0       \\
	SR-COLUMN & Structured reference column & \textbackslash[ [A-Z0-9\textbackslash\textbackslash\_.$\Box_1$]+ \textbackslash] & -3 \\
	%UNOP\_POSTFIX & Unary postfix operator & \% & 0 & \% \\
	%UNOP\_PREFIX & Unary prefix operator & + $\mid$ - & 0 & -                  \\
	UDF & User Defined Function & (\_xll\textbackslash.)? [A-Z\_\textbackslash][A-Z0-9\_\textbackslash\textbackslash.$\Box_1$]*  ( & 4 \\
	VERTICAL-RANGE & Range of columns & \$? [A-Z]+ : \$? [A-Z]+ & 0 \\ 
	\midrule
	Placeholder character & Placeholder for & Specification & \\
	$\Box_1$ & Extended characters & \begin{tabular}[c]{@{}l@{}} Non-control Unicode characters x80 and up \end{tabular} & \\
	$\Box_2$ & Sheet characters & \begin{tabular}[c]{@{}l@{}} Any character except\\ ' * [ ] \textbackslash\ : / ? ( ) ; \{ \} \# " = < > \& + - * / \textasciicircum{} \% , \texttt{\char32} \end{tabular}& \\
	$\Box_3$ & Enclosed sheet characters & \begin{tabular}[c]{@{}l@{}} Any character except ' * [ ] \textbackslash\ : / ? \end{tabular} & \\
	$\Box_4$ & Filename characters & Any character except " * [ ] \textbackslash\ : / ? < > $\mid$ \\
	\bottomrule
	\multicolumn{4}{l}{\begin{footnotesize}\textsuperscript{\number\value{footnote}} A function list is available as part of the reference implementation.\end{footnotesize}}\\
	\multicolumn{4}{l}{\begin{footnotesize} \hspace{0.5em} Lists provided by Microsoft are also available in \cite{ExcelFunctionReference} and \cite{ExcelOfficialGrammar}.\end{footnotesize}}\\
	
\end{tabular}

\stepcounter{footnote}
\end{table}

Table \ref{table:tokens} contains the lexical tokens of the grammar, along with their identification patterns in a simple regular expression language. All tokens are case-insensitive.
Characters are defined as unicode characters x9,xA,xD and x20 and upwards.

This grammar requires the parser to support token priorities, which Irony does.
Removing the necessity for token priorities is possible by altering the tokens and production rules, but makes the grammar more complicated and the resulting tree harder to use, thus being detrimental to design goals 2 and 3.

Some simple tokens (e.g. '\%', '!') are directly defined in the production rules in Figure \ref{figure:productions} in between quotes for readability and compactness.

\subsubsection{\textbf{Dates}}

The appearance of date and time values in spreadsheets depends on the presentation settings of cells. Internally, date and time values are stored as positive floating point numbers with the integer portion representing the number of days since a Jan 0 1900 epoch and the fractional portion representing the portion of the day passed.

When extracting formulas from spreadsheets, only the floating point value is available.
The parser will thus never see the human notation of the date.
For this reason, the grammar only parses numeric dates and times and these are not distinguishable from other numbers.


\todo{Move table to appendix?}

\subsection{Syntactical Analysis}

\begin{figure}
\small
\caption{Production rules}
\label{figure:productions}
\begin{multicols*}{2}
\begin{grammar}
	<Start> ::= <Constant>
	\alt '=' <Formula>
	\alt `\{=' <Formula> `\}'
	
	<Formula> ::= <Constant>
	\alt <Reference>
	\alt <FunctionCall>
	\alt `(' <Formula> `)'
	\alt <ConstantArray>
	\alt "RESERVED-NAME"
	
	<Constant> ::= "NUMBER" | "STRING" | "BOOL" | "ERROR"
	%         \alt "STRING"
	%         \alt "BOOL"
	%         \alt "ERROR"
	
	<FunctionCall> ::= "EXCEL-FUNCTION" <Arguments> `)'
	\alt <UnOpPrefix> <Formula>
	\alt <Formula> `\%'
	\alt <Formula> <BinOp> <Formula>
	
	<UnOpPrefix> ::= `+' | `-'
	
	<BinOp> ::= `+' | `-' | `*' | `/' | `\textasciicircum'
	\alt `<' | `>' | `=' | `<=' | `>=' | `<>'
	
	%<Function> ::=  | "UDF"
	
	<Arguments> ::= <Argument> \{ `,' <Argument> \} | $\epsilon$
	
	<Argument> ::= <Formula> | $\epsilon$
	
	<Reference> ::= <ReferenceItem>
	\alt <RefFunctionCall>
	\alt `(' <Reference> `)' 
	\alt <Prefix> <ReferenceItem>
	%\alt <Prefix> "UDF" <Arguments> `)'
	%    \alt <DynamicDataExchange>
	\alt "FILE" `!' "DDECALL"
	
	%<NamedRange> ::= "NAMED-RANGE"
	%            \alt "NAMED-RANGE-PREFIXED"
	<NamedRange> ::= "NR" | "NR-PREFIXED"
	
	\columnbreak
	
	<RefFunctionCall> ::= <Union>
	\alt <RefFunctionName> <Arguments> `)'
	\alt <Reference> `:' <Reference>
	\alt <Reference> `\ ' <Reference>
	
	<ReferenceItem> ::= "CELL"
	\alt <NamedRange>
	\alt "VERTICAL-RANGE"
	\alt "HORIZONTAL-RANGE"
	\alt "UDF" <Arguments> `)'
	\alt "ERROR-REF"
	
	<Prefix> ::= "SHEET"
	\alt "FILE" "SHEET"
	\alt "FILE" `!'
	\alt "QUOTED-FILE-SHEET"
	\alt "MULTIPLE-SHEETS"
	\alt "FILE" "MULTIPLE-SHEETS"
	
	<RefFunctionName> ::= "REF-FUNCTION"
	\alt "REF-FUNCTION-COND"
	
	
	
	<Union> ::= `(' <Reference> \{ `,' <Reference> \} `)'
	
	%<DynamicDataExchange> ::= "FILE" `!' "DDECALL"
	
	<ConstantArray> ::= `\{' <ArrayColumns> `\}'
	
	<ArrayColumns> ::= <ArrayRows> \{ `;' <ArrayRows> \}
	
	<ArrayRows> ::= <ArrayConst> \{ `,' <ArrayConst> \}
	
	<ArrayConst> ::= <Constant>
	\alt <UnOpPrefix> "NUMBER"
	\alt "ERROR-REF"
	
\end{grammar}
\end{multicols*}
\end{figure}

The complete production rules of the grammar are listed in Extended BNF syntax in Figure \ref{figure:productions}.
Patterns inside \{ and \} can be repeated zero or more times.
The start symbol is $Start$. An example parse tree produced using this grammar is drawn in Figure \ref{figure:parsetrees}(b).
\synt{Formula} and \synt{Reference} are the two most important production rules.

The \synt{Formula} rule covers all types of spreadsheet formula expressions: they can be constants (\texttt{=5}), references (\texttt{=A3}), function calls (\texttt{=SUM(A1:A3)}), array constants (\texttt{=\{1,2;3,4\}}, explained in Section \ref{sec:arrayformulas}), or reserved names (\texttt{=_xlnm.Print_Area}). Function calls invoke actual named (system or user defined) functions or operators applied to one or more formulas.

The \synt{Reference} rule, as shown in Figure \ref{figure:Reference}, supports internal (in the same or in different sheets), or external single cell references, cell ranges, horizontal and vertical ranges, named ranges and reference-returning, built-in or user-defined, functions.

\todo{Move these figures to appendix?}

%\begin{figure}
%	\caption{Syntax diagram of the \synt{Formula} production rule with most production rules expanded}
%	\label{figure:Formula}
%	%\centerfloat
%	\begin{grammar}
	<Formula> ::= 
	\begin{syntdiag}[\footnotesize\sdlengths]
		\begin{stack} 
			\begin{stack}
				%"EXCEL-FUNCTION" \begin{rep} \begin{stack} \\ <Formula> \end{stack} \\  '$,$' \end{rep} '$)$' \\
				"EXCEL-FUNCTION" <Arguments> '$)$' \\
				\begin{stack} \\ '$+$' \\ '$-$' \end{stack} <Formula> \begin{stack} \\ '$\%$' \end{stack}\\ 
				%<Formula>  \begin{stack} '$+$' \\ '$-$' \\ '$*$' \\ '$/$' \\ '\textasciicircum'  \\ '\textless' \\ '\textgreater' \\ '$=$' \\ '$\textless=$' \\ '$\textgreater=$' \\ '$\textless\textgreater$'   \end{stack} <Formula>
				<Formula> <BinOp> <Formula>
			\end{stack} \\
			\begin{stack} "NUMBER" \\ "STRING" \\ "BOOL" \\ "ERROR"  \end{stack} \\ "RESERVED-NAME" \\ <Reference> \\ '$($' <Formula> '$)$' \\  '$\{$' \begin{rep} \begin{rep}	\begin{stack} \begin{stack} \begin{stack} \\ '$+$' \\ '$-$' \end{stack} "NUMBER" \\ "STRING" \\ "BOOL" \\ "ERROR"  \end{stack} \\ "ERROR-REF" \end{stack} \\  '$,$' \end{rep} \\ '$;$' \end{rep} '$\}$'
		\end{stack}
	\end{syntdiag}
\end{grammar}
%\end{figure}
%
%\begin{figure}
%	\caption{Syntax diagram of the \synt{Reference} production rule with most production rules expanded}
%	\label{figure:Reference}
%	\begin{grammar}
		<Reference> ::= \begin{syntdiag}[\scriptsize\sdlengths]
		\begin{stack} '$($' \begin{rep} <Reference> \\  '$,$' \end{rep} '$)$'\\ <Reference> \begin{stack} '$:$' \\ '\char32' \end{stack} <Reference> \\
		\begin{stack} \\ \begin{stack} \\ "FILE" \end{stack} \begin{stack} "SHEET" \\ "MULTIPLE_SHEETS" \end{stack} \\ "FILE" '$!$' \\ "QUOTED_FILE_SHEET" \end{stack}
		\begin{stack} \begin{stack} "CELL" \\ "VERTICAL-RANGE" \\ "HORIZONTAL-RANGE" \\ \begin{stack} "NAMED-RANGE" \\ "NAMED-RANGE-COMBINED" \end{stack} \\ "ERROR-REF" \end{stack} \\  \begin{stack} "REFERENCE-FUNCTION" \\ "UDF" \end{stack} \begin{rep} \begin{stack} \\ <Formula> \end{stack} \\  '$,$' \end{rep} '$)$'\end{stack}
		\\"FILE" '$!$' "DDECALL"
		\end{stack}
		\end{syntdiag}
	\end{grammar}
%\end{figure}

\subsection{Precedence and ambiguity}

\begin{table}
\small
\caption{Operator precedence in formulas}
\label{table:operatorprec}
\begin{tabular}{lll}
Precedence & Operator(s) & Associativity \\
1 & = \textless \  \textgreater \  \textless= \  \textgreater= \  \textless\textgreater & Left         \\
2 & \& & Left \\
3 & + - (binary) & Left \\
4 & $\ast$ / & Left \\
5 & \textasciicircum & Left\footnotemark \\
6 & \% & \\
7 & + - (unary) & \\
8 & : , \texttt{\char32} & Left
\end{tabular}
\end{table}

\footnotetext{This is contrary to most other languages, where the exponentiation operator is right-associative. \\
In Excel \f{2\textasciicircum 1\textasciicircum 2} will be $(2^1)^2 = 4$, while in most other languages it will be $2^{1^2} = 2$}

The production rules are ambiguous, which means they cannot be directly used in a parser generator based on the LALR algorithm like Irony.

To resolve ambiguity with operators, e.g. whether to parse \f{=1+2*3} as \f{=(1+2)*3} or \f{=1+(2*3)}, operator precedence and associativity rules are defined.
These can be found in table \ref{table:operatorprec}.

However, even with precedence and associativity rules the grammar is still not fully un-ambigious.
This is due to trade-offs on parsing references, see section \ref{tradeoff:references}, and parsing unions (see section \ref{subsec:desing:unions}).
Ambiguity exists between the following production rules:
\begin{enumerate}
	\item \begin{grammar}<Reference> ::= `(' <Reference> `)'\end{grammar}
	\item \begin{grammar}<Union> ::= `(' <Reference> \{ `,' <Reference> \} `)'\end{grammar}
	\item \begin{grammar}<Formula> ::= `(' <Formula> `)'\end{grammar}
\end{enumerate}

A formula like \texttt{=(A1)} can be interpreted as either a bracketed reference, a union of one reference, or a reference within a bracketed formula.

In a LALR parser, which Irony produces, this ambiguity manifests in a state where, on a \texttt{')'} token, shifting on rule 1 and reducing on either rule 2 or 3 are possibilities, causing a shift-reduce conflict.
This was solved by instructing the parser generator to shift on rule 1 in case of this conflict, because this always is a correct interpretation and thus results in correct ASTs.

\section{Trade-offs and quirks}

The grammar presented in this chapter (\todo{section?}) contains some quirks on sub-optimalities, partly due to the Excel language itself, partly due to design decisions.
These are detailed in this section.

\subsection{References}
\label{tradeoff:references}

In some places only reference expressions are accepted, instead of all expressions.
Furthermore references are of great importance in spreadsheet formulas, and thus of interest for analysis.
To support easier analysis (design goal 2) references have different production rules than other expressions.
This causes references to be easily identified and isolated, but has the downside of increasing ambiguity, as explained in Section \ref{sec:ambiguity}. 

Another approach would be to parse all expressions with the same production rules and implement a type system, however this would be detrimental to both ease of analysis (design goal 2) and to ease of implementation (design goal 3).

\subsection{Unions}
\label{subsec:desing:unions}

The comma serves both as an union operator and a function argument separator.
This proves challenging to correctly implement in a LALR(1) grammar.

A straightforward implementation would use production rules similar to this:
\begin{grammar}
<Union> ::= <Reference> `,' <Reference>

<Arguments> ::= <Argument>
	\alt <Argument> `,' <Arguments>
\end{grammar}

However, this will cause a reduce-reduce conflict because the parser will have a state wherein it can reduce to both a \synt{Union} or \synt{Argument} on a \texttt{,} token.
Unfortunately there is no correct choice: in a formula like \texttt{=SUM(A1,1)} the parser must reduce on the \synt{Argument} nonterminal, while in a formula like \texttt{=A1,A1} the parser must reduce to the \synt{Union} nonterminal.
With the above production rules a LALR(1) parser could not correctly parse the language.

The presented grammar only parsers unions in between parentheses, e.g. \texttt{=SMALL((A1,A2),1)}.
This is a trade-off between a lower compatibility (design goal 1) and an easier implementation (design goal 3).
Because unions are very rare \cite{xlparser} this lower compatibility is deemed acceptable.

Additionally formulas that this grammar does not parse often result in runtime errors after evaluation.
For example \texttt{=A1,A1} does parse in a spreadsheet program, but produces the error \texttt{\#VALUE!} on evaluation.

Implementing the straightforward rules above, while desirable, is not possible without using a more powerful grammar class.

\subsection{Evaluation}

\todo{Kort iets schrijven over hoe de parser geevalueerd is, samenvatting van evaluatie sectie in paper}