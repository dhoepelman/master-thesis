% !TeX root = ../thesis.tex
% !TeX spellcheck = en_US

\chapter{Conclusion}

While extending an existing spreadsheet refactoring tool it was found that the existing spreadsheet formula parser was insufficient to support refactoring without the risk of introducing errors.
To solve this problem, the existing parser was improved, named XLParser and publicly released as open-source\footnote{\url{https://github.com/spreadsheetlab/XLParser}}.
The parser is geared towards spreadsheet research and therefore is highly compatible with the official language and produces parse trees which are suitable for further analysis and manipulation.
A corresponding EBNF grammar is presented in this thesis.
This parser was used to implement six refactorings, three of which were not implemented before and two of which improve upon previous implementations.

\subsubsection{Future Work}

There is still a lot of space for progress in the spreadsheet refactoring field.
Of particular interest are the similarities between relational databases and spreadsheets, both contain data in a tabular format which can be linked through joins and lookup function respectively.
This implies certain database design principles such as normalization might be applicable to spreadsheets.

Another area of interest is the user interface for spreadsheet refactoring software.
The refactorings in this thesis are already disabled and enabled based on what the user selects to refactor, but it would be even better if refactorings were pro-actively suggested. 
Unfortunately the most prominent spreadsheet program currently used, Microsoft Excel, provides a large barrier for this as it severely restricts what Add-Ins can do.
Microsoft could greatly reduce the barrier for spreadsheet tools by altering the Excel API, for example by allowing Add-Ins access to the undo and redo stacks or formula AST.