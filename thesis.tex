
\documentclass[12pt,a4paper,onecolumn,oneside]{memoir}
%twoside for print version
\RequirePackage[xetex]{graphicx}
\graphicspath{ {}{img/} }

\usepackage[xetex,table,xcdraw]{xcolor}

\usepackage[
breaklinks=true,colorlinks=true,
linkcolor=blue,urlcolor=blue,citecolor=blue,% PDF VIEW
%linkcolor=black,urlcolor=black,citecolor=black,% PRINT
bookmarks=true,bookmarksopenlevel=2]{hyperref}

\usepackage{geometry}
% PDF VIEW
% \geometry{total={210mm,297mm},
% left=25mm,right=25mm,%
% bindingoffset=0mm, top=25mm,bottom=25mm}
% PRINT
\geometry{total={210mm,297mm},
left=20mm,right=20mm,
bindingoffset=10mm, top=25mm,bottom=25mm}

\OnehalfSpacing
%\linespread{1.3}

\usepackage{cite}

% TABLES
\usepackage{tablefootnote}
\usepackage{booktabs}
\usepackage{pbox}
\renewcommand{\arraystretch}{1.2} 
%\renewcommand*\cmidrule{} % No table middle lines
%\renewcommand{\arraystretch}{1.5} % Additional spacing with no middle lines
%\renewcommand*\cmidrule{\hdashline[1pt/2pt]}% Dashed middle lines
\renewcommand*\cmidrule{\midrule[0.001em]} % Thin table middle lines
%\renewcommand*\cmidrule{\midrule} % Thick table middle lines

% GRAMMAR
\usepackage[rounded]{syntax}
\setlength{\grammarparsep}{4pt plus 1pt minus 1pt}

\usepackage{amsmath}
\usepackage{amssymb}

\usepackage{multicol}
\setlength{\columnsep}{1cm}

% Graphs
\usepackage{tkz-berge}
\usetikzlibrary {positioning}
\definecolor {processblue}{cmyk}{0.96,0,0,0}
%%% STYLE OF PAGES NUMBERING
%\pagestyle{companion} %\nouppercaseheads 
%\pagestyle{headings}
%\pagestyle{ruled}
%\pagestyle{plain}
%\makepagestyle{plain}
% based on companion style
\makepagestyle{hoepelman}
%\newlength{\headwidth}
\setlength{\headwidth}{\textwidth}
  %\addtolength{\headwidth}{\marginparsep}
  %\addtolength{\headwidth}{\marginparwidth}
\makerunningwidth{hoepelman}{\headwidth}
\makeheadrule{hoepelman}{\headwidth}{\normalrulethickness}
\makeheadposition{hoepelman}{flushright}{flushleft}{}{}
\makepsmarks{hoepelman}{%
  \def\chaptermark##1{\markboth{##1}{##1}}    % left mark & right marks
  \def\sectionmark##1{\markright{%
    \ifnum \c@secnumdepth>\z@
      \thesection. \ %
    \fi
    ##1}}
  \def\tocmark{\markboth{\contentsname}{\contentsname}}
  \def\lofmark{\markboth{\listfigurename}{\listfigurename}}
  \def\lotmark{\markboth{\listtablename}{\listtablename}}
  \def\bibmark{\markboth{\bibname}{\bibname}}
  \def\indexmark{\markboth{\indexname}{\indexname}}}
\makepsmarks{hoepelman}{%
  \nouppercaseheads
  \createmark{chapter}{both}{nonumber}{}{}
  \createmark{section}{right}{shownumber}{}{. \space}
  \createplainmark{toc}{both}{\contentsname}
  \createplainmark{lof}{both}{\listfigurename}
  \createplainmark{lot}{both}{\listtablename}
  \createplainmark{bib}{both}{\bibname}
  \createplainmark{index}{both}{\indexname}
  \createplainmark{glossary}{both}{\glossaryname}}
%\makeevenfoot{hoepelman}{}{}{}
%\makeoddfoot{hoepelman}{}{}{}
\makeevenhead{hoepelman}{\normalfont\bfseries\thepage}{}{\normalfont\bfseries\leftmark}
\makeoddhead{hoepelman}{\normalfont\bfseries\thepage}{}{\normalfont\bfseries\leftmark}
%\makeoddhead{hoepelman}{\normalfont\bfseries\rightmark}{}{\normalfont\bfseries\thepage}

\pagestyle{hoepelman}

%%% CHAPTER'S STYLE
\makeatletter
\newcommand\thickhrulefill{\leavevmode \leaders \hrule height 1ex \hfill \kern \z@}
\setlength\midchapskip{10pt}
\makechapterstyle{VZ14}{
\renewcommand\chapternamenum{}
\renewcommand\printchaptername{}
\renewcommand\chapnamefont{\Large\scshape}
\renewcommand\printchapternum{%
\chapnamefont\null\thickhrulefill\quad
\@chapapp\space\thechapter\quad\thickhrulefill}
\renewcommand\printchapternonum{%
\par\thickhrulefill\par\vskip\midchapskip
\hrule\vskip\midchapskip
}
\renewcommand\chaptitlefont{\Huge\scshape\centering}
\renewcommand\afterchapternum{%
\par\nobreak\vskip\midchapskip\hrule\vskip\midchapskip}
\renewcommand\afterchaptertitle{%
\par\vskip\midchapskip\hrule\nobreak\vskip\afterchapskip}
}
\makeatother
\chapterstyle{VZ14}


%%% STYLE OF SECTIONS, SUBSECTIONS, AND SUBSUBSECTIONS
\setsecheadstyle{\Large\bfseries\sffamily\raggedright}
\setsubsecheadstyle{\large\bfseries\sffamily\raggedright}
\setsubsubsecheadstyle{\bfseries\sffamily\raggedright}


% Custom commands
\newcommand\textlcsc[1]{\textsc{\MakeLowercase{#1}}}

\newcommand{\todo}[1]{\textbf{TODO: #1}}
\newcommand{\ignore}[1]{}
%\usepackage[utf8]{inputenc}
\usepackage[T1]{fontenc}
\usepackage{microtype}
%\usepackage{fontspec}

%Default: computer modern

% 
%\usepackage{kpfonts}

% Times-like
%\usepackage{newtxtext}
%\usepackage{newtxmath}

% Times-like
%\usepackage{mathptmx}
%\usepackage{courier}

%Calibri
%\setmainfont{Calibri}

% Palatino
%\usepackage{newpxtext}
%\usepackage{newpxmath}

%\usepackage{libertine}
%\usepackage[libertine]{newtxmath}

\usepackage[scaled]{helvet}
\renewcommand\familydefault{\sfdefault}
\usepackage{sansmath}

\maxsecnumdepth{subsection} % chapters, sections, and subsections are numbered
\maxtocdepth{subsection} % chapters, sections, and subsections are in the Table of Contents

\begin{document}

\thispagestyle{empty}
\onecolumn
{%%%
\sffamily
\centering

~\vspace{\fill}

{\huge \bfseries
Inter-cell Spreadsheet Refactoring and Parsing Spreadsheet Formulas
}

\vspace{2.0cm}

by

\vspace{2.0cm}

{\LARGE
David Jonathan Hoepelman
}

\vspace{3.0cm}

\textlcsc{Master Thesis} \\
\today

\vspace{2.5cm}

For obtaining the degree of \\
Master of Science \\
in Computer Science - Software Technology \\

\vspace{0.5cm}

Faculty Electrical Engineering, Mathematics and Computer Science (EEMCS)\\
Delft University of Technology

\vspace{1.5cm}

\includegraphics{tudelft}
\hspace{0.5cm}
%\includegraphics[height=8mm]{serg}
%\hspace{0.5cm}
\includegraphics[height=8mm]{spreadsheet-lab}
\hspace{0.5cm}

\vspace{\fill}

}

\cleardoublepage

%%%---%%%---%%%---%%%---%%%---%%%---%%%---%%%---%%%---%%%---%%%---%%%---%%%
\tableofcontents*

\clearpage
%\twocolumn

%%%---%%%---%%%---%%%---%%%---%%%---%%%---%%%---%%%---%%%---%%%---%%%---%%%
%%%---%%%---%%%---%%%---%%%---%%%---%%%---%%%---%%%---%%%---%%%---%%%---%%%

\chapter{Introduction}
Blaatverhaal over waarom spreadsheets veel gebruikt worden en daarom belangrijk zijn, maintainability nightmare en hoe refactoring kan helpen.

\chapter{Spreadsheet anatomy}

Spreadsheets are cell-oriented dataflow programs which are Turing complete \cite{ExcelTuringComplete}.

A single spreadsheet \emph{file} corresponds to a single (\emph{work})\emph{book}.
A workbook can contain any number of (\emph{work})\emph{sheets}.
A sheet consists of a two-dimensional grid of \emph{cells}.
The grid consists or verticals \emph{rows} and horizontal \emph{columns}.
Rows are numbered sequentially top-to-bottom starting at 1, while columns are numbered left-to-right alphabetically, i.e. base-26 using A to Z as digits, starting at `A', making column 27 `AA'.

A cell can be empty or contain a \emph{constant value}, a \emph{formula} or an \emph{array formula}.

\section{Formulas}

Formulas consist of expressions which can contain constant values, arithmetic operators and \emph{function calls} such as \texttt{SUM(\ldots)} and, most importantly, \emph{references} to other cells.
Function arguments are separated by commas.

\subsection{References}
References are the core component of spreadsheets.
The value of any cell can be used in a formula by concatenating its column and row number, producing a reference like \texttt{B5}.
If the value of a cell changes this new value will be propagated to all formulas that use it.

When copying a cell to another cell by default references will be adjusted by the offset, for example copying \texttt{=A1} from cell B1 to C2 will cause the copied formula to become \texttt{=B2}.
This can be prevented by making the reference absolute by prepending a \texttt{\$} to the column index, row index or both.
The formula \texttt{=\$A\$1} will remain the same on copy while \texttt{=\$A1} will still have its row number adjusted.

An alternate style called R1C1 as opposed to the above A1 style exists, but it is very rarely seen or used by users.
In R1C1 references one specifies either the offset to a cell between square brackets or its concrete location.
In R1C1 style \texttt{R[4]C[-2]} means the cell two columns to the left and four rows down, while \texttt{R2C2} refers to cell B2.
The biggest advantage of R1C1 is that it causes identical formulas to be the same even when they operate on different cells or data because of their position.
These properties make R1C1 useful as an internal representation, but the grammar presented in this paper is intended for the common A1 reference style.

References can also be \emph{ranges}, which are collections of cells.
Ranges can be constructed by three operators: the range operator \texttt{:}, the union operator \texttt{,} (a comma) and the intersection operator \texttt{\char32} (a single whitespace).
The range operator creates a rectangular range with the two cells as top-left and bottom-right corners, so \texttt{=SUM(A1:B10)} will sum all cells in columns A and B with row number 1 through 10.
The range operator is also used to construct ranges of whole rows or columns, for example \texttt{3:5} is the range of the complete rows three through five, and \texttt{A:D} is the range of columns A through D.
The union operator, which is different from the mathematical union as duplicates are allowed, combines two references, so \texttt{A1,C5} will be a range of two cells, \texttt{A1} and \texttt{C5}.
Lastly the intersection operator takes only the cells which are in both arguments, \texttt{=A:A 5:5} will thus be equivalent to \texttt{=A5}.

A user can also give a name to any collection of cells, thus creating a \emph{named range} which can be referenced in formulas by name.

\subsection{Inter-sheet and external references}
\label{subsection:ExternalRefsDDE}

By default references are to cells or ranges in the same sheet as the formula, but this can be modified with a prefix. A prefix consists of some identifier, followed by an exclamation mark followed by the actual reference.

The most common use case is to reference another sheet in the same workbook, where the prefix is simply the sheetname: \texttt{=Sheetname!A1}. References to external spreadsheet files are also possible, which is done by prepending the file name in between square brackets: \texttt{=[Filename]Sheetname!A1} or \texttt{=[Filename]!NamedRange}.
A peculiar type of prefix are those that indicate multiple sheets: \texttt{=Sheet1:Sheet10!A1} means A1 in Sheet1 through Sheet10.
Sheet names can also be between single quotes: \texttt{='Sheetname with space'!A1}. 

In Windows versions of Microsoft Excel formulas can also call external programs through Dynamic Data Exchange (DDE). DDE links are a special case of references, used for receiving data from other applications. They take the form of \texttt{=Program|Topic!Arguments}, e.g. \texttt{=Database|TableA!Column1}.

\subsection{Array Formulas and Arrays}
\label{sec:arrayformulas}
In spreadsheet programs it is possible to work with one- or two-dimensional matrices.

When constructed from constant values they are called \emph{array constants}, e.g. \texttt{\{1,2;3,4\}}.
They are surrounded by curly brackets, columns are separated by commas, and rows by semicolons.
Several matrix operations are available, for example \texttt{=SUM(\{1,2,3\}*10)} will evaluate to 60.

\emph{Array Formulas} use the same syntax as normal formulas, except that the user must enter \emph{Ctrl} + \emph{Shift} + \emph{Enter} to signal that it is an Array formula.
Excel and LibreOffice surround the formula with curly braces.
Google docs works differently and requires the user to surround an array formula with \texttt{ARRAYFORMULA($\ldots$)}.

Marking a formula as an array formula will enable one- or two-dimensional ranges to be treated as array.
For example if \texttt{A1},\texttt{A3},\texttt{A3} contain the values \texttt{1},\texttt{2},\texttt{3} the array formula \texttt{\{=SUM(A1:A3*10)\}} will evaluate to \texttt{60}. Furthermore, an array formula allows the user to return multiple results, which will be presented in multiple cells.
The array formula \texttt{\{=\{1,2,3\}*\{4,5,6\}\}} will show \texttt{4}, \texttt{10} and \texttt{18} in three different cells.

\subsection{Other features}

\chapter{Previous and related work}

\section{Refactoring}

\section{The BumbleBee spreadsheet refactoring suite}

\section{Other spreadsheet refactoring efforts}


\chapter{Candidate spreadsheet refactorings}

\section{Translating refactorings to the spreadsheet domain}

Tabel met mogelijke refactorings.
Subsecties voor alle geïmplementeerde refactorings.

\chapter{Implementation}

\section{Parsing formulas}

\subsection{Excel Formula Grammar}

\section{Transforming formula ASTs}

\section{Implementing refactorings}

\subsection{Extract formula}

\subsection{Inline formula}

\subsection{Addition to SUMIF}

\subsection{\textasciicircum\textasciicircum or Aggregate to conditional Aggregate}

\subsection{Group References}

\chapter{Evaluation}


\chapter{Conclusion}

\section{Future Work}

\bibliographystyle{unsrt}
\bibliography{thesis}

\end{document}

