
\documentclass[12pt,a4paper,onecolumn,oneside]{memoir}
%twoside for print version
\RequirePackage[xetex]{graphicx}
\graphicspath{ {}{img/} }

\usepackage[xetex,table,xcdraw]{xcolor}

\usepackage[
breaklinks=true,colorlinks=true,
linkcolor=blue,urlcolor=blue,citecolor=blue,% PDF VIEW
%linkcolor=black,urlcolor=black,citecolor=black,% PRINT
bookmarks=true,bookmarksopenlevel=2]{hyperref}

\usepackage{geometry}
% PDF VIEW
% \geometry{total={210mm,297mm},
% left=25mm,right=25mm,%
% bindingoffset=0mm, top=25mm,bottom=25mm}
% PRINT
\geometry{total={210mm,297mm},
left=20mm,right=20mm,
bindingoffset=10mm, top=25mm,bottom=25mm}

\OnehalfSpacing
%\linespread{1.3}

\usepackage{cite}

% TABLES
\usepackage{tablefootnote}
\usepackage{booktabs}
\usepackage{pbox}
\renewcommand{\arraystretch}{1.2} 
%\renewcommand*\cmidrule{} % No table middle lines
%\renewcommand{\arraystretch}{1.5} % Additional spacing with no middle lines
%\renewcommand*\cmidrule{\hdashline[1pt/2pt]}% Dashed middle lines
\renewcommand*\cmidrule{\midrule[0.001em]} % Thin table middle lines
%\renewcommand*\cmidrule{\midrule} % Thick table middle lines

% GRAMMAR
\usepackage[rounded]{syntax}
\setlength{\grammarparsep}{4pt plus 1pt minus 1pt}

\usepackage{amsmath}
\usepackage{amssymb}

\usepackage{multicol}
\setlength{\columnsep}{1cm}

% Graphs
\usepackage{tkz-berge}
\usetikzlibrary {positioning}
\definecolor {processblue}{cmyk}{0.96,0,0,0}
%%% STYLE OF PAGES NUMBERING
%\pagestyle{companion} %\nouppercaseheads 
%\pagestyle{headings}
%\pagestyle{ruled}
%\pagestyle{plain}
%\makepagestyle{plain}
% based on companion style
\makepagestyle{hoepelman}
%\newlength{\headwidth}
\setlength{\headwidth}{\textwidth}
  %\addtolength{\headwidth}{\marginparsep}
  %\addtolength{\headwidth}{\marginparwidth}
\makerunningwidth{hoepelman}{\headwidth}
\makeheadrule{hoepelman}{\headwidth}{\normalrulethickness}
\makeheadposition{hoepelman}{flushright}{flushleft}{}{}
\makepsmarks{hoepelman}{%
  \def\chaptermark##1{\markboth{##1}{##1}}    % left mark & right marks
  \def\sectionmark##1{\markright{%
    \ifnum \c@secnumdepth>\z@
      \thesection. \ %
    \fi
    ##1}}
  \def\tocmark{\markboth{\contentsname}{\contentsname}}
  \def\lofmark{\markboth{\listfigurename}{\listfigurename}}
  \def\lotmark{\markboth{\listtablename}{\listtablename}}
  \def\bibmark{\markboth{\bibname}{\bibname}}
  \def\indexmark{\markboth{\indexname}{\indexname}}}
\makepsmarks{hoepelman}{%
  \nouppercaseheads
  \createmark{chapter}{both}{nonumber}{}{}
  \createmark{section}{right}{shownumber}{}{. \space}
  \createplainmark{toc}{both}{\contentsname}
  \createplainmark{lof}{both}{\listfigurename}
  \createplainmark{lot}{both}{\listtablename}
  \createplainmark{bib}{both}{\bibname}
  \createplainmark{index}{both}{\indexname}
  \createplainmark{glossary}{both}{\glossaryname}}
%\makeevenfoot{hoepelman}{}{}{}
%\makeoddfoot{hoepelman}{}{}{}
\makeevenhead{hoepelman}{\normalfont\bfseries\thepage}{}{\normalfont\bfseries\leftmark}
\makeoddhead{hoepelman}{\normalfont\bfseries\thepage}{}{\normalfont\bfseries\leftmark}
%\makeoddhead{hoepelman}{\normalfont\bfseries\rightmark}{}{\normalfont\bfseries\thepage}

\pagestyle{hoepelman}

%%% CHAPTER'S STYLE
\makeatletter
\newcommand\thickhrulefill{\leavevmode \leaders \hrule height 1ex \hfill \kern \z@}
\setlength\midchapskip{10pt}
\makechapterstyle{VZ14}{
\renewcommand\chapternamenum{}
\renewcommand\printchaptername{}
\renewcommand\chapnamefont{\Large\scshape}
\renewcommand\printchapternum{%
\chapnamefont\null\thickhrulefill\quad
\@chapapp\space\thechapter\quad\thickhrulefill}
\renewcommand\printchapternonum{%
\par\thickhrulefill\par\vskip\midchapskip
\hrule\vskip\midchapskip
}
\renewcommand\chaptitlefont{\Huge\scshape\centering}
\renewcommand\afterchapternum{%
\par\nobreak\vskip\midchapskip\hrule\vskip\midchapskip}
\renewcommand\afterchaptertitle{%
\par\vskip\midchapskip\hrule\nobreak\vskip\afterchapskip}
}
\makeatother
\chapterstyle{VZ14}


%%% STYLE OF SECTIONS, SUBSECTIONS, AND SUBSUBSECTIONS
\setsecheadstyle{\Large\bfseries\sffamily\raggedright}
\setsubsecheadstyle{\large\bfseries\sffamily\raggedright}
\setsubsubsecheadstyle{\bfseries\sffamily\raggedright}


% Custom commands
\newcommand\textlcsc[1]{\textsc{\MakeLowercase{#1}}}

\newcommand{\todo}[1]{\textbf{TODO: #1}}
\newcommand{\ignore}[1]{}
%\usepackage[utf8]{inputenc}
\usepackage[T1]{fontenc}
\usepackage{microtype}
%\usepackage{fontspec}

%Default: computer modern

% 
%\usepackage{kpfonts}

% Times-like
%\usepackage{newtxtext}
%\usepackage{newtxmath}

% Times-like
%\usepackage{mathptmx}
%\usepackage{courier}

%Calibri
%\setmainfont{Calibri}

% Palatino
%\usepackage{newpxtext}
%\usepackage{newpxmath}

%\usepackage{libertine}
%\usepackage[libertine]{newtxmath}

\usepackage[scaled]{helvet}
\renewcommand\familydefault{\sfdefault}
\usepackage{sansmath}

\maxsecnumdepth{subsection} % chapters, sections, and subsections are numbered
\maxtocdepth{subsection} % chapters, sections, and subsections are in the Table of Contents

\newcommand{\todo}[1]{\textbf{TODO: #1}}
\newcommand{\ignore}[1]{}
\newcommand{\f}[1]{\texttt{#1}}
\newcommand{\key}[1]{\textbf{#1}}

\begin{document}

\thispagestyle{empty}
\onecolumn
{%%%
\sffamily
\centering

~\vspace{\fill}

{\huge \bfseries
Inter-cell Spreadsheet Refactoring and Parsing Spreadsheet Formulas
}

\vspace{2.0cm}

by

\vspace{2.0cm}

{\LARGE
David Jonathan Hoepelman
}

\vspace{3.0cm}

\textlcsc{Master Thesis} \\
\today

\vspace{2.5cm}

For obtaining the degree of \\
Master of Science \\
in Computer Science - Software Technology \\

\vspace{0.5cm}

Faculty Electrical Engineering, Mathematics and Computer Science (EEMCS)\\
Delft University of Technology

\vspace{1.5cm}

\includegraphics{tudelft}
\hspace{0.5cm}
%\includegraphics[height=8mm]{serg}
%\hspace{0.5cm}
\includegraphics[height=8mm]{spreadsheet-lab}
\hspace{0.5cm}

\vspace{\fill}

}

\cleardoublepage

%%%---%%%---%%%---%%%---%%%---%%%---%%%---%%%---%%%---%%%---%%%---%%%---%%%
\tableofcontents*

\clearpage
%\twocolumn

%%%---%%%---%%%---%%%---%%%---%%%---%%%---%%%---%%%---%%%---%%%---%%%---%%%
%%%---%%%---%%%---%%%---%%%---%%%---%%%---%%%---%%%---%%%---%%%---%%%---%%%

\chapter{Introduction}

Like all people, I sometimes get asked what I do for a living.
When I tell someone I am writing my master thesis in Computer Science, their eyes start to glaze over as they anticipate some explanation peppered with terms they will not understand about 
I then tell them my thesis is about spreadsheets and ask if they have ever worked with Excel, and nearly everyone who has ever worked in business or research has.
Nearly everyone has a horror story about that one unmaintainable spreadsheet that they had to work on, or that day their reporting system broke down because 2009 turned into 2010 and the spreadsheet only looked at the last digit.

This anecdotal evidence is mirrored in research.
Panko \cite{panko2006facing} estimates that 80\% to 95\% of businesses use spreadsheets in one of their processes.
Furthermore, almost all spreadsheets contains at least one error and 1 to 5\% of spreadsheet cells contains an error according to Panko \cite{panko1998we}.
Spreadsheets perform roles very similar to software in that they perform business-critical roles, are inherited throughout the organization and maintained by different users and accrue technical dept during and after the initial development period \cite{panko1998we}.
In short, spreadsheets can be classified as programs, and spreadsheet creators are end-user programmers.

%\chapter{Spreadsheet anatomy}
\chapter{Anatomy of spreadsheets and spreadsheet formulas}
\label{chapter:anatomy}

\todo{Notificatie dat dit hoofdstuk deels al gepubliceerd is}

Altough exotic forms of spreadsheets are available and have been researched, all currently widely used implementation use the following model:
\begin{itemize}
\item A single spreadsheet \key{file} corresponds to a single (\key{work})\key{book}.
\item A workbook can contain any number of (\key{work})\key{sheets}.
\item A sheet consists of a \key{two-dimensional grid} (table) of \key{cells}.
\item A vertical unit in the grid is called a \key{column} and a horizontal unit a \key{row}.
Rows are numbered sequentially top-to-bottom starting at 1, while columns are numbered left-to-right alphabetically, i.e. base-26 using A to Z as digits.
A column or row can also mean all cells contained in that column or row.
\item A \key{cell} can contain a \key{constant value} of any type, a calculation called a \key{formula} or a matrix calculation called an \key{array formula}.
\item An (array) formula can \key{reference} other cells to use their values. When the value of a referenced cell changes, this new value is propagated and the dependent formula values are recalculated.
\end{itemize}

\begin{figure}
\caption{An example dataflow program and its spreadsheet implementation}
\label{fig:dataflow-example}
\centerfloat
\scalebox{0.83}{
\begin{tikzpicture}[-latex ,auto ,node distance =1.3 cm and 1.8cm ,on grid , semithick,
,
state/.style ={ circle ,top color =white ,
draw, minimum width =0.9 cm}]
\node[state] (Input3) {3};
\node[state] (Input5) [below=of Input3] {5};
\node[state] (Plus) [right=of Input5] {$+$};
\node[state] (Input2) [below=of Input5] {2};
\node[state] (Times) [below right=of Plus] {$\times$};
\node[state] (Output) [right=of Times] {\small{\texttt{out}}};
\path (Input3) edge [right] node[right] {} (Plus);
\path (Input5) edge [right] node[right] {} (Plus);
\path (Plus) edge [right] node[right] {} (Times);
\path (Input2) edge [right] node[right] {} (Times);
\path (Times) edge [right] node[right] {} (Output);
%\path (A) edge [loop left] node[left] {$1/4$} (A);
%\path (C) edge [bend left =25] node[below =0.15 cm] {$1/2$} (A);
%\path (A) edge [bend right = -15] node[below =0.15 cm] {$1/2$} (C);
%\path (A) edge [bend left =25] node[above] {$1/4$} (B);
%\path (B) edge [bend left =15] node[below =0.15 cm] {$1/2$} (A);
%\path (C) edge [bend left =15] node[below =0.15 cm] {$1/2$} (B);
%\path (B) edge [bend right = -25] node[below =0.15 cm] {$1/2$} (C);
\end{tikzpicture}
}\hspace{4em}
\includegraphics[width=0.5\textwidth]{anatomy/dataflow-program-excel-implementation}
\end{figure}

This model is a variation of the dataflow programming model.
A dataflow program is a directed graph, where data flows between operation in nodes along the graphs edges.
In spreadsheets, cells represent the nodes of a dataflow program and edges are represented by references.
An example dataflow program and its spreadsheet implementation can be seen in Figure \ref{fig:dataflow-example}.

The spreadsheet model is Turing complete, as proven by an Excel 2010 implementation of a Turing machine \cite{ExcelTuringComplete}.

\section{Formulas}

Formulas consist of expressions which can contain constant values, \key{function calls} and \key{operators} and, most importantly, \key{references} to other cells.
A cell is identified as a formula cell because all formulas must start with the equals sign \f{=}.

\subsection{Function calls and operators}

Function calls are done by starting with the function name, followed by the arguments in parentheses, separated by a comma.
All spreadsheet implementations provide a range of built-in functions, and in most spreadsheet implementations it is possible to define new functions yourself.
However this is not done directly inside the spreadsheet, and uses an alternate programming language.
In Microsoft Excel and Openoffice.org/Libreoffice this is done with a variant of the BASIC programming language, while in Google Docs this is done with Javascript.

The operators \f{+ - * / = <> <= < >} and \f{>=} can be used according to their usual semantics.
\f{+} and \f{-} are available both unary (\f{=-1}) and binary (\f{=1-1}).
Additionally the \f{\%} unary operator is defined to transform a number into a percentage, \f{\^} is the exponentiation operator and \f{\&} is the text concatenation operator.

Spreadsheets contain three fairly unique binary operators with the range operator \f{:}, the union operator \f{,} and the intersection operator \f{\char32}.
Te semantics of these operators are detailed in section \ref{sec:references}.

\subsection{References}
\label{sec:references}
References are the core component of spreadsheets.
The value of any cell can be used in a formula by concatenating its column and row number, producing a reference like \f{B5}.
This is called A1-style referencing and is by far the most comment in modern spreadsheet implementations.
If the value of a cell changes this new value will be propagated to all formulas that use it.

When copying a cell to another cell by default references will be adjusted by the offset, for example copying \f{=A1} from cell B1 to C2 will cause the copied formula to become \f{=B2}.
This can be prevented by making the reference absolute by prepending a \f{\$} to the column index, row index or both.
The formula \f{=\$A\$1} will remain the same on copy while \f{=\$A1} will still have its row number adjusted when copied.

\subsection{Ranges}
References can also be \emph{ranges}, which are collections of cells.
Ranges can be constructed by three operators: the range operator \f{:}, the union operator \f{,} (a comma) and the intersection operator \f{\char32} (a space).
The range operator \f{:} creates a rectangular range with the two cells as top-left and bottom-right corners, so \f{=SUM(A1:B10)} will sum all cells in columns A and B with row number 1 through 10.
The range operator is also used to construct ranges of whole rows or columns, for example \f{3:5} is the range of the complete rows three through five, and \f{A:D} is the range of columns A through D.
The union operator, which is different from the mathematical union as duplicates are allowed, combines two references, so \f{A1,C5} will be a range of two cells, \f{A1} and \f{C5}.
Lastly the intersection operator takes only the cells which are in both arguments, \f{=A:A 5:5} will thus be equivalent to \f{=A5}.

A user can also give a name to any collection of cells, thus creating a \emph{named range} which can be referenced in formulas by name.

\subsection{Structured References}

A recent addition to the formula language introduced in Excel 2007 are structured (table) references.
To use this feature, a table can be given a name and must have headers.
One can then reference a column in the table by entering \f{TableName[ColumnName]}.
Inside the square brackets reference operators can be used to construct more complex references, \f{TableName[Column1,Column4]} references two columns.
Several keywords 

There is no way to reference a specific row, except the current row, for example if a formula is places in \f{A3} it can only reference row number 3.
The \f{\#This Row} keyword and the \f{@} operator are used for this: \f{TableName[\#This Row]} and \f{TableName[@]} both reference the current row number in the provided table, and \f{TableName[@ColumnName]} references the cell in the provided column of the current row number.

This feature is meant to make formulas easier to read, by references in a formula like \f{=SUM(F1:F10)-SUM(I1:I10)} with human-readable names like \f{=SUM(Budget[Revenue]) - SUM(Budget[Expenses])}.

\subsection{R1C1 reference style}

An alternate style called R1C1 as opposed to the above A1 style exists, but it is very rarely seen or used by users.
In R1C1 reference style one specifies either the offset to a cell between square brackets or its concrete location.
In R1C1 style \f{R[4]C[-2]} means the cell two columns to the left and four rows down, while \f{R2C2} refers to cell B2.
The biggest advantage of R1C1 is that it causes identical formulas to be the same even when they operate on different cells or data because of their position.
These properties make R1C1 useful as an internal representation in spreadsheet implementations.

\subsection{Inter-sheet and external references}
\label{subsection:ExternalRefsDDE}

By default references are to cells or ranges in the same sheet as the formula, but this can be modified with a prefix. A prefix consists of some identifier, followed by an exclamation mark followed by the actual reference.

The most common use case is to reference another sheet in the same workbook, where the prefix is simply the sheet name: \f{=Sheetname!A1}.
References to external spreadsheet files are also possible, which is done by prepending the file name in between square brackets: \f{=[Filename]Sheetname!A1} or \f{=[Filename]!NamedRange}.
A peculiar type of prefix are those that indicate multiple sheets: \f{=Sheet1:Sheet10!A1} means A1 in Sheet1 through Sheet10.
Sheet names can also be between single quotes: \f{='Sheetname with space'!A1}. 

In Windows versions of Microsoft Excel formulas can also call external programs through Dynamic Data Exchange (DDE).
DDE links are a special case of references, used for receiving data from other applications.
They take the form of \f{=Program|Topic!Arguments}, e.g. \f{=Database|TableA!Column1}.

\subsection{Case sensitivity}

Formulas are case-insensitive outside string literals.
Identifiers have a canonical capitalization, and while user can type the identifier with any casing only the canonical form will be displayed.
While the canonical capitalization of built-in identifiers,functions and reserved names, is usually uppercase, the canonical capitalization of an user-defined identifier, an user defined function or a named range, is as the user defined it originally.

\subsection{Whitespace sensitivity}

The excel formula language is whitespace sensitive in several places:
\begin{itemize}
\item Inside string literals
\item To separate syntactic units: \f{=AB} has a different meaning than \f{=A B}
\item Whitespace is not allowed between function names and the argument list: \f{=SUM  (1)} is invalid.
\item Whitespace is not allowed inside internal or external references: \f{=Sheet1 !A1} is invalid.
\item The intersection operator is a single space: \f{=A:A 3:3} is the intersection of column \f{A} and row \f{3}, equivalent to \f{=A3}
\end{itemize}

\section{Array Formulas and Arrays}
\label{sec:arrayformulas}
In spreadsheet programs it is possible to work with one- or two-dimensional matrices.

When constructed from constant values they are called \emph{array constants}, e.g. \f{\{1,2;3,4\}}.
They are surrounded by curly brackets, columns are separated by commas, and rows by semicolons.
Several matrix operations are available, for example \f{=SUM(\{1,2,3\}*10)} will evaluate to 60.

\emph{Array Formulas} use the same syntax as normal formulas, except that the user must enter \emph{Ctrl} + \emph{Shift} + \emph{Enter} to signal that it is an Array formula.
Excel and LibreOffice surround the formula with curly braces.
Google docs works differently and requires the user to surround an array formula with \f{ARRAYFORMULA($\ldots$)}.

Marking a formula as an array formula will enable one- or two-dimensional reference ranges to be treated as matrices, and several matrix operators and functions will be available and allow a formula to return multiple values, to be displayed in multiple cells.
For example if \f{A1},\f{A3},\f{A3} contain the values \f{1},\f{2},\f{3} the array formula \f{\{=SUM(A1:A3*10)\}} will evaluate to \f{60}.
Furthermore, an array formula allows the user to return multiple results, which will be presented in multiple cells.
The array formula \f{\{=\{1,2,3\}*\{4,5,6\}\}} will show \f{4}, \f{10} and \f{18} in three different cells.

\section{Type system}

The type system in Microsoft Excel is weak with most typed being able to be coerced into others.
Explicit type conversion is done by certain built-in functions.
For values inside formulas, the following types exist:

\begin{itemize}
\item[Boolean] values are either \f{TRUE} or \f{FALSE}. Booleans can be coerced to numbers where \f{TRUE} will become 1 and \f{FALSE} will become 0 and strings.
\item[Numeric] values are in the range of 8-byte IEEE doubles. Numbers can be provided as integers, decimals or in scientific notation.
When coerced to booleans 0 will become \f{FALSE}, all other values will be \f{TRUE}. Number can also be coerced into strings.
\item[String] values are any Unicode character enclosed in quotation marks \f{"}.
Two quotation marks serve as the escape character, thus \f{""""} represent the string ".
If the contents of a cell start with a \texttt{'} the rest of that cell content is interpreted as a string.

When coerced to booleans all strings except the empty string are \f{TRUE}, the empty string is \f{FALSE}.
When coerced to a numeric value the spreadsheet program will accept any string representing valid numeric user input and otherwise give the error \f{\#VALUE!}.
\item[Error] values are \f{\#DIV/0!}, \f{\#NAME?}, \f{\#NULL!}, \f{\#NUM!}, \f{\#N/A!}, \f{\#VALUE!} and \f{\#REF!}.
Errors behave similar to exceptions or the maybe monad in that they will propagate throughout a calculation.

Errors cannot be coerced, but can be explicitly converted by functions such as \f{ISERROR}.

\item[Ranges and arrays] are one- or two-dimensional matrices of any non-array values. Arrays are rarely used outside of array formulas, but ranges are very common in formulas.
However, these types usually only serve as inputs for functions and are thus fairly transparent to the user outside of array formulas.
Both types usually cannot generally be, doing so will result in the \f{\#VALUE!} error.
\end{itemize}

Some other "display types" exists, these can change the way the data is presented to or validated from the user and can have implications when inter-operating with other programs.
Usually the user can mark a cell as containing one of these types, or Excel can automatically mark a cell to be of this type based on heuristics.
In formulas and internally these types are all represented by one of the above types.
A few of these are commonly used:
\begin{itemize}
\item[Dates and times] are internally stored as a floating point with the integer portion being the number of days since an the epoch January 1st 1900, incorrectly considering 1900 a leap year, and the remainder being the portion of the day that passed.
Excel displays dates and times as is customary in the locale of the user.
When interoperating a date or time value will be exported as a datetime type value of that system.
\item[Currency] is stored as other numbers and displayed in the format customary for the specific currency. When interoperating with some other systems currency values will be exported using arbitrary-precision arithmetic formats.
\item[Percentages] are stored as other numbers, but displayed as if multiplied by 100\%.
\end{itemize}

\chapter{Previous and related work}

\section{Refactoring}

\section{The BumbleBee spreadsheet refactoring suite}

\section{Other spreadsheet refactoring efforts}


\chapter{Candidate spreadsheet refactorings}

\section{Comparing spreadsheets to other programming paradigms}

\subsection{Reactive programming}
As explained in chapter \ref{chapter:anatomy}, the spreadsheet programming model is a variation of the dataflow programming model.
Thus the most natural first comparison for spreadsheet is the \emph{reactive programming} paradigm, which is another variation of the dataflow programming model.
Viewed from the RP model, all spreadsheet cells are observable values and every formula is an observer.
One can attach a formula observer to a cell observable by making a reference, e.g. a formula \f{=A1+1} in cell B1 listens to the value of A1 and produces the value for B1.
However currently no literature on refactoring reactive programs is known to the author, so no additional refactoring were found by comparing spreadsheets to reactive programs.

\subsection{Functional programming}
Spreadsheets program do not contain state (mutable variables) or side-effects
\footnote{As long as one remains within the spreadsheet model. User-defined functions and inter-operation with external programs can cause side-effects.}
and this immutability aspect is often associated with functional programming languages.
However spreadsheets lack several other important concepts of functional languages such as first-class and higher-order functions, recursion\footnote{Most spreadsheet implementations allow for a form of recursion in the form of cycles in the dependency graph called circular references. However this is not the normal modus-operandi, triggers a warning and is generally only used to perform iterative calculations which are not otherwise possible in spreadsheets.}, lazy evaluation and type interference.

\subsection{Translation of concepts}

Some literature exist on refactoring functional programs, most prominent is the tool \textbf{Ha}skell \textbf{Re}factoring tool (HaRe) \cite{thompson2005refactoring}.
Most of the described refactorings are either specific to Haskell, or only applicable because of concepts not available in spreadsheets.

This makes all of the current literature on refactoring functional programs not applicable.
\todo{Cite functional refactoring literature}

\subsection{Object-oriented programming}

By far 



\section{Translating refactorings to the spreadsheet domain}

Tabel met mogelijke refactorings.
Subsecties voor alle geïmplementeerde refactorings.

%\chapter{Parsing spreadsheet formulas}
\chapter{Parsing spreadsheet formulas}

\emph{\small{Parts of this chapter have been published in "A grammar for Spreadsheet Formulas Evaluated on Two Large Datasets" by by Aivaloglou, Hoepelman and Hermans\cite{xlparser}.}}
\vspace{1em}

This chapter assumes the reader is familiar with basic compiler construction theory.

\section{Motivation}

In order to implement refactorings of spreadsheets Bumblebee must be able to manipulate to manipulate spreadsheets.
Excel exposes an API to retrieve and set the contents of a spreadsheet file, but this API only works on a formula string level.
The usual way to implement refactorings is by manipulating the AST so that it represents the desired program, and then print that back to a string.
Because Excel does not expose the AST or parse tree of formulas, Bumblebee needs to contain a parser for Excel formulas.

Bumblebee relied on a parser developed over the years for previous research, however this parser suffered from having new rules tacked on over time, often in an inconsistent manner and not fully supporting the whole language.
Furthermore the parser interperted some lanuage constructs from and missed several features, which made implementing refactorings hard.
For example operator precedence was not taken into account, causing the formula \f{=A1 + A2 * A3} was parsed as \f{=(A1 + A2) * A3}, which breakes the "Plus to SUM" refactoring.
The existing parser thus was insufficient to implement the Bumblebee refactorings.

For Bumblebee and other research on spreadsheet formulas the following requirements about the parser and grammar were formulated:

\begin{enumerate}
\label{sec:designgoals}
\item The parser must be compatible with the official language
\item Produced parse trees must be suited for further manipulation and analysis with minimal post-processing required
\item The grammar must be compact enough to feasibly implement with a parser generator
\end{enumerate}

While an official grammar for Excel formulas is published \cite{ExcelOfficialGrammar}, it does not meet the above requirements for two reasons.
Firstly, it is over 30 pages long and contains hundreds of production rules and thus fails requirement 3.
Secondly, because of the detail of the grammar and the large number of production rules the resulting parse trees are very complex and fail requirement 2.

Because there is no suitable parser and grammar available that satisfy the above requirements, I decided to clean up and partially rewrite the parser in coorporation with other members of the Spreadsheet lab.
The end result of this effort is an independent, open-source parser for Excel formulas called XLParser \footnote{https://github.com/PerfectXL/XLParser}, about which a paper was published in IEEE conference SCAM 2015 \cite{xlparser}.

\section{Parser implementation}

The existing parsing was built upon the Irony\footnote{https://irony.codeplex.com/}, which is a C\# parser generator that procudes parsers based on the LALR(1) algorithm using a grammar defined in C\#.

Strictly speaking Irony produces a parse tree, however this tree is fairly abstract, leaving out elements such as punctuation and whitespace, and no separate AST is constructed in Bumblebee, this tree is instead directly manipulated.
To avoid confusion and use usual nomenclature I will from now on refer to this tree as the AST.

\subsection{Lexical Analysis}
\label{sec:lexanalysis}

\begin{table}
\tiny
\caption{Lexical tokens used in the XLParser grammar, as refered to in section \ref{sec:lexanalysis}.}
\label{table:tokens}
\centerfloat
%\advance\leftskip-1cm
\stepcounter{footnote}

\begin{tabular}{@{}llll@{}}
	\toprule
	Token Name & Description & Contents & Priority \\
	\midrule
	%BINOP & Binary Operator & + $\mid$ - $\mid$ / $\mid$ * $\mid$ \textasciicircum $\mid$ \textless $\mid$ \textgreater $\mid$ = $\mid$ \textless= $\mid$ \textgreater= $\mid$ \textless \textgreater & 0 & + \\
	BOOL & Boolean literal & TRUE $\mid$ FALSE & 0 \\
	CELL & Cell reference & \$? [A-Z]+ \$? [0-9]+ & 2 \\
	DDECALL & Dynamic Data Exchange link & ' ([\textasciicircum{} '] $\mid$ '')+ ' & 0 \\
	ERROR & Error literal & \begin{tabular}[c]{@{}l@{}} \#NULL! $\mid$ \#DIV/0! $\mid$ \#VALUE! \\ $\mid$ \#NAME? $\mid$ \#NUM! $\mid$ \#N/A \end{tabular} & 0 \\
	ERROR-REF & Reference error literal & \#REF! & 0 \\
	EXCEL-FUNCTION & Excel built-in function & (Any entry from the function list\begin{footnotesize}\textsuperscript{\number\value{footnote}}\end{footnotesize}) \textbackslash( & 5        \\
	FILE & External file reference using number & \textbackslash[ [0-9]+ \textbackslash] & 5 \\
	FILENAME & External file reference using name & \textbackslash[ $\Box_4$+ \textbackslash] & -1 \\
	FILEPATH & Windows file path & [A-Z] : \textbackslash\textbackslash\ ($\Box_4$+ \textbackslash\textbackslash)* & 0 \\
	HORIZONTAL-RANGE & Range of rows & \$? [0-9]+ : \$? [0-9]+ & 0 \\
	MULTIPLE-SHEETS & Multiple sheet references & 
	%$\Box_2$+ : ($\Box_2$+ $\mid$ '($\Box_3$ $\mid$ '')+')!
	(($\Box_2$+ : $\Box_2$+)|( ' ($\Box_3$ $\mid$ '')+ : ($\Box_3$ $\mid$ '')+ ' )) !
	& 1 \\
	NAME & User Defined Name & [A-Z\_\textbackslash\textbackslash][A-Z0-9\textbackslash\textbackslash\_.$\Box_1$]* & -2 \\
	NAME-PREFIXED & \begin{tabular}[c]{@{}l@{}} User defined name which starts with \\ a string  that could be another token \end{tabular} & (TRUE $\mid$ FALSE $\mid$ [A-Z]+[0-9]+)    {[}A-Z0-9\_.$\Box_1${]}+                                                                             & 3 \\
	NUMBER & \begin{tabular}[c]{@{}l@{}}An integer, floating point\\     or scientific notation number literal\end{tabular} & [0-9]+ ,? [0-9]* (e [0-9]+)? & 0 \\
	REF-FUNCTION & Excel built-in reference function & (INDEX $\mid$ OFFSET $\mid$ INDIRECT)\textbackslash( & 5 \\
	REF-FUNCTION-COND & Excel built-in conditional reference function & (IF $\mid$ CHOOSE)\textbackslash( & 5 \\
	RESERVED-NAME & An Excel reserved name & \_xlnm\textbackslash.  [A-Z\_]+ & -1 \\
	SHEET & The name of a worksheet & $\Box_2$+ ! 
	& 5        \\
	SHEET-QUOTED & Quoted worksheet name & $\Box_3$+ ' ! & 5 \\
	STRING & String literal & " ([\textasciicircum{} "] $\mid$ "")* " & 0       \\
	SR-COLUMN & Structured reference column & \textbackslash[ [A-Z0-9\textbackslash\textbackslash\_.$\Box_1$]+ \textbackslash] & -3 \\
	%UNOP\_POSTFIX & Unary postfix operator & \% & 0 & \% \\
	%UNOP\_PREFIX & Unary prefix operator & + $\mid$ - & 0 & -                  \\
	UDF & User Defined Function & (\_xll\textbackslash.)? [A-Z\_\textbackslash][A-Z0-9\_\textbackslash\textbackslash.$\Box_1$]*  ( & 4 \\
	VERTICAL-RANGE & Range of columns & \$? [A-Z]+ : \$? [A-Z]+ & 0 \\ 
	\midrule
	Placeholder character & Placeholder for & Specification & \\
	$\Box_1$ & Extended characters & \begin{tabular}[c]{@{}l@{}} Non-control Unicode characters x80 and up \end{tabular} & \\
	$\Box_2$ & Sheet characters & \begin{tabular}[c]{@{}l@{}} Any character except\\ ' * [ ] \textbackslash\ : / ? ( ) ; \{ \} \# " = < > \& + - * / \textasciicircum{} \% , \texttt{\char32} \end{tabular}& \\
	$\Box_3$ & Enclosed sheet characters & \begin{tabular}[c]{@{}l@{}} Any character except ' * [ ] \textbackslash\ : / ? \end{tabular} & \\
	$\Box_4$ & Filename characters & Any character except " * [ ] \textbackslash\ : / ? < > $\mid$ \\
	\bottomrule
	\multicolumn{4}{l}{\begin{footnotesize}\textsuperscript{\number\value{footnote}} A function list is available as part of the reference implementation.\end{footnotesize}}\\
	\multicolumn{4}{l}{\begin{footnotesize} \hspace{0.5em} Lists provided by Microsoft are also available in \cite{ExcelFunctionReference} and \cite{ExcelOfficialGrammar}.\end{footnotesize}}\\
	
\end{tabular}

\stepcounter{footnote}
\end{table}

Table \ref{table:tokens} contains the lexical tokens of the grammar, along with their identification patterns in a simple regular expression language. All tokens are case-insensitive.
Characters are defined as unicode characters x9,xA,xD and x20 and upwards.

This grammar requires the parser to support token priorities, which Irony does.
Removing the necessity for token priorities is possible by altering the tokens and production rules, but makes the grammar more complicated and the resulting tree harder to use, thus being detrimental to design goals 2 and 3.

Some simple tokens (e.g. '\%', '!') are directly defined in the production rules in Figure \ref{figure:productions} in between quotes for readability and compactness.

\subsubsection{\textbf{Dates}}

The appearance of date and time values in spreadsheets depends on the presentation settings of cells. Internally, date and time values are stored as positive floating point numbers with the integer portion representing the number of days since a Jan 0 1900 epoch and the fractional portion representing the portion of the day passed.

For this reason, the grammar only parses numeric dates and times and these are not distinguishable from other numbers.


\todo{Move table to appendix?}

\subsection{Syntactical Analysis}

\begin{figure}
\small
\caption{Production rules}
\label{figure:productions}
\begin{multicols*}{2}
\begin{grammar}
	<Start> ::= <Constant>
	\alt '=' <Formula>
	\alt `\{=' <Formula> `\}'
	
	<Formula> ::= <Constant>
	\alt <Reference>
	\alt <FunctionCall>
	\alt `(' <Formula> `)'
	\alt <ConstantArray>
	\alt "RESERVED-NAME"
	
	<Constant> ::= "NUMBER" | "STRING" | "BOOL" | "ERROR"
	%         \alt "STRING"
	%         \alt "BOOL"
	%         \alt "ERROR"
	
	<FunctionCall> ::= 
	<UnOpPrefix> <Formula>
	\alt <Formula> `\%'
	\alt <Formula> <BinOp> <Formula>
	\alt "EXCEL-FUNCTION" <Arguments> `)'
	
	<UnOpPrefix> ::= `+' | `-'
	
	<BinOp> ::= `+' | `-' | `*' | `/' | `\textasciicircum'
	\alt `<' | `>' | `=' | `<=' | `>=' | `<>'
	
	%<Function> ::=  | "UDF"
	
	<Arguments> ::= $\epsilon$ \alt <Argument> \{ `,' <Argument> \}
	
	<Argument> ::= <Formula> | $\epsilon$
	
	<Reference> ::= <ReferenceItem>
	\alt <RefFunctionCall>
	\alt `(' <Reference> `)' 
	\alt <Prefix> <ReferenceItem>
	%\alt <Prefix> "UDF" <Arguments> `)'
	%    \alt <DynamicDataExchange>
	\alt "FILE" `!' "DDECALL"
	
	%<NamedRange> ::= "NAMED-RANGE"
	%            \alt "NAMED-RANGE-PREFIXED"
	
	<RefFunctionCall> ::= `(' <Union> `)'
	\alt <RefFunctionName> <Arguments> `)'
	\alt <Reference> `:' <Reference>
	\alt <Reference> `\ ' <Reference>
	
	<ReferenceItem> ::= "CELL"
	\alt <NamedRange>
	\alt <StructuredReference>
	\alt "VERTICAL-RANGE"
	\alt "HORIZONTAL-RANGE"
	\alt "UDF" <Arguments> `)'
	\alt "ERROR-REF"
	
	\columnbreak
	
	<NamedRange> ::= <Name>
		
	<Name> ::= "NAME" | "NAME-PREFIXED"
	
	<File> ::= FILE
	\alt FILENAME
	\alt FILEPATH FILENAME
	
	<Prefix> ::= "SHEET"
	\alt `'' "SHEET-QUOTED"
	\alt <File> "SHEET"
	\alt `'' <File> "SHEET-QUOTED"
	\alt "FILE" `!'
	\alt "MULTIPLE-SHEETS"
	\alt <File> "MULTIPLE-SHEETS"
	
	<RefFunctionName> ::= "REF-FUNCTION"
	\alt "REF-FUNCTION-COND"
	
	<Union> ::= <Reference> \{ `,' <Reference> \}
	
	%<DynamicDataExchange> ::= "FILE" `!' "DDECALL"
	
	<ConstantArray> ::= `\{' <ArrColumns> `\}'
	
	<ArrColumns> ::= <ArrRows> \{ `;' <ArrRows> \}
	
	<ArrRows> ::= <ArrConst> \{ `,' <ArrConst> \}
	
	<ArrConst> ::= <Constant>
	\alt <UnOpPrefix> "NUMBER"
	\alt "ERROR-REF"
	
	<StructuredReference> ::= <SRCol>
	\alt `[' <SRExpr> `]'
	\alt <Name> <SRCol>
	\alt <Name> `[' <SRExpr> `]'
	
	<SRExpr> ::= <SRCol>
	\alt <SRCol> `:' <SRCol>
	\alt <SRCol> `,' <SRCol>
	\alt <SRCol> `,' <SRCol> `:' <SRCol>
	\alt <SRCol> `,' <SRCol> `,' <SRCol>
	\alt <SRCol> `,' <SRCol> `,' <SRCol> `:' <SRCol>
	
	<SRCol> ::= FILENAME
	\alt `[' <Name> `]'  
	\alt `[' SR-COLUMN `]'  
	
	
	
\end{grammar}
\end{multicols*}
\end{figure}

The complete production rules of the grammar are listed in Extended BNF syntax in Figure \ref{figure:productions}.
Patterns inside \{ and \} can be repeated zero or more times.
The start symbol is $Start$. An example parse tree produced using this grammar is drawn in Figure \ref{figure:parsetrees}(b).
\synt{Formula} and \synt{Reference} are the two most important production rules.

The \synt{Formula} rule covers all types of spreadsheet formula expressions: they can be constants (\texttt{=5}), references (\texttt{=A3}), function calls (\texttt{=SUM(A1:A3)}), array constants (\texttt{=\{1,2;3,4\}}, explained in Section \ref{sec:arrayformulas}), or reserved names (\texttt{=_xlnm.Print_Area}). Function calls invoke actual named (system or user defined) functions or operators applied to one or more formulas.

The \synt{Reference} rule, as shown in Figure \ref{figure:Reference}, supports internal (in the same or in different sheets), or external single cell references, cell ranges, horizontal and vertical ranges, named ranges and reference-returning, built-in or user-defined, functions.

\todo{Move these figures to appendix?}

%\begin{figure}
%	\caption{Syntax diagram of the \synt{Formula} production rule with most production rules expanded}
%	\label{figure:Formula}
%	%\centerfloat
%	\input{parsing/formula-diagram.tex}
%\end{figure}
%
%\begin{figure}
%	\caption{Syntax diagram of the \synt{Reference} production rule with most production rules expanded}
%	\label{figure:Reference}
%	\begin{grammar}
		<Reference> ::= \begin{syntdiag}[\scriptsize\sdlengths]
		\begin{stack} '$($' \begin{rep} <Reference> \\  '$,$' \end{rep} '$)$'\\ <Reference> \begin{stack} '$:$' \\ '\char32' \end{stack} <Reference> \\
		\begin{stack} \\ \begin{stack} \\ "FILE" \end{stack} \begin{stack} "SHEET" \\ "MULTIPLE_SHEETS" \end{stack} \\ "FILE" '$!$' \\ "QUOTED_FILE_SHEET" \end{stack}
		\begin{stack} \begin{stack} "CELL" \\ "VERTICAL-RANGE" \\ "HORIZONTAL-RANGE" \\ \begin{stack} "NAMED-RANGE" \\ "NAMED-RANGE-COMBINED" \end{stack} \\ "ERROR-REF" \end{stack} \\  \begin{stack} "REFERENCE-FUNCTION" \\ "UDF" \end{stack} \begin{rep} \begin{stack} \\ <Formula> \end{stack} \\  '$,$' \end{rep} '$)$'\end{stack}
		\\"FILE" '$!$' "DDECALL"
		\end{stack}
		\end{syntdiag}
	\end{grammar}
%\end{figure}

\subsection{Precedence and ambiguity}

\begin{table}
\small
\caption{Operator precedence in formulas}
\label{table:operatorprec}
\begin{tabular}{lll}
Precedence & Operator(s) & Associativity \\
1 & = \textless \  \textgreater \  \textless= \  \textgreater= \  \textless\textgreater & Left         \\
2 & \& & Left \\
3 & + - (binary) & Left \\
4 & $\ast$ / & Left \\
5 & \textasciicircum & Left\footnotemark \\
6 & \% & \\
7 & + - (unary) & \\
8 & : , \texttt{\char32} & Left
\end{tabular}
\end{table}

\footnotetext{This is contrary to most other languages, where the exponentiation operator is right-associative. \\
In Excel \f{2\textasciicircum 1\textasciicircum 2} will be $(2^1)^2 = 4$, while in most other languages it will be $2^{1^2} = 2$}

The production rules are ambiguous, which means they cannot be directly used in a parser generator based on the LALR algorithm like Irony.

To resolve ambiguity with operators, e.g. whether to parse \f{=1+2*3} as \f{=(1+2)*3} or \f{=1+(2*3)}, operator precedence and associativity rules are defined.
These can be found in table \ref{table:operatorprec}.

However, even with precedence and associativity rules the grammar is still not fully un-ambigious.
This is due to trade-offs on parsing references, see section \ref{tradeoff:references}, and parsing unions (see section \ref{subsec:desing:unions}).
Ambiguity exists between the following production rules:
\begin{enumerate}
	\item \begin{grammar}<Reference> ::= `(' <Reference> `)'\end{grammar}
	\item \begin{grammar}<Union> ::= `(' <Reference> \{ `,' <Reference> \} `)'\end{grammar}
	\item \begin{grammar}<Formula> ::= `(' <Formula> `)'\end{grammar}
\end{enumerate}

A formula like \texttt{=(A1)} can be interpreted as either a bracketed reference, a union of one reference, or a reference within a bracketed formula.

In a LALR like the one Irony produces this ambiguity manifests in a state where, on a \texttt{')'} token, shifting on rule 1 and reducing on either rule 2 or 3 are possibilities, causing a shift-reduce conflict.
This was solved by instructing the parser generator to shift on rule 1 in case of this conflict, because this always is a correct interpretation and thus results in correct ASTs.

\section{Trade-offs and quirks}

The grammar presented in this chapter (\todo{section?}) contains some quirks on sub-optimalities, partly due to the Excel language itself, partly due to design decisions.
These are detailed here

\subsection{\textbf{References}}
\label{tradeoff:references}

In some places only reference expressions are accepted, instead of all expressions.
Furthermore references are of great importance in spreadsheet formulas, and thus of interest for analysis.
To support easier analysis (design goal 2) references have different production rules than other expressions.
This causes references to be easily identified and isolated, but has the downside of increasing ambiguity, as explained in Section \ref{sec:ambiguity}. 

Another approach would be to parse all expressions with the same production rules and implement a type system, however this would be detrimental to both ease of analysis (design goal 2) and to ease of implementation (design goal 3).

\subsection{\textbf{Unions}}
\label{subsec:desing:unions}

The comma serves both as an union operator and a function argument separator.
This proves challenging to correctly implement in a LALR(1) grammar.

A straightforward implementation would use production rules similar to this:
\begin{grammar}
<Union> ::= <Reference> `,' <Reference>

<Arguments> ::= <Argument>
	\alt <Argument> `,' <Arguments>
\end{grammar}

However, this will cause a reduce-reduce conflict because the parser will have a state wherein it can reduce to both a \synt{Union} or \synt{Argument} on a \texttt{,} token.
Unfortunately there is no correct choice: in a formula like \texttt{=SUM(A1,1)} the parser must reduce on the \synt{Argument} nonterminal, while in a formula like \texttt{=A1,A1} the parser must reduce to the \synt{Union} nonterminal.
With the above production rules a LALR(1) parser could not correctly parse the language.

The presented grammar only parsers unions in between parentheses, e.g. \texttt{=SMALL((A1,A2),1)}.
This is a trade-off between a lower compatibility (design goal 1) and an easier implementation (design goal 3).
Because unions are very rare \cite{xlparser} this lower compatibility is deemed acceptable.

Additionally formulas that this grammar does not parse often result in runtime errors after evaluation.
For example \texttt{=A1,A1} does parse in a spreadsheet program, but produces the error \texttt{\#VALUE!} on evaluation.

Implementing the straightforward rules above, while desirable, is not possible without using a more powerful grammar class.

\subsection{Evaluation}

\todo{Kort iets schrijven over hoe de parser geevalueerd is, samenvatting van evaluatie sectie in paper}

\chapter{Implementing refactorings}

Parse -> Transform -> Print

\section{Transforming formula ASTs}

\section{Refactoring-specific implementation}

\subsection{Extract formula}

\subsection{Inline formula}

\subsection{Addition to SUMIF}

\subsection{\textasciicircum\textasciicircum or Aggregate to conditional Aggregate}

\subsection{Group References}

\chapter{Evaluation}


\chapter{Conclusion}

\section{Future Work}

\bibliographystyle{unsrt}
\bibliography{thesis}

\end{document}

