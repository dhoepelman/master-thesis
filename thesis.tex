% !TeX encoding = UTF-8
% !TeX program = xelatex
% !TeX spellcheck = en_US

\documentclass[12pt,a4paper,onecolumn,oneside,parskip]{memoir}
%twoside for print version
\RequirePackage[xetex]{graphicx}
\graphicspath{ {}{img/} }

\usepackage[xetex,table,xcdraw]{xcolor}

\usepackage[
breaklinks=true,colorlinks=true,
linkcolor=blue,urlcolor=blue,citecolor=blue,% PDF VIEW
%linkcolor=black,urlcolor=black,citecolor=black,% PRINT
bookmarks=true,bookmarksopenlevel=2]{hyperref}

\usepackage{geometry}
% PDF VIEW
% \geometry{total={210mm,297mm},
% left=25mm,right=25mm,%
% bindingoffset=0mm, top=25mm,bottom=25mm}
% PRINT
\geometry{total={210mm,297mm},
left=20mm,right=20mm,
bindingoffset=10mm, top=25mm,bottom=25mm}

\OnehalfSpacing
%\linespread{1.3}

\usepackage{cite}

% TABLES
\usepackage{tablefootnote}
\usepackage{booktabs}
\usepackage{pbox}
\renewcommand{\arraystretch}{1.2} 
%\renewcommand*\cmidrule{} % No table middle lines
%\renewcommand{\arraystretch}{1.5} % Additional spacing with no middle lines
%\renewcommand*\cmidrule{\hdashline[1pt/2pt]}% Dashed middle lines
\renewcommand*\cmidrule{\midrule[0.001em]} % Thin table middle lines
%\renewcommand*\cmidrule{\midrule} % Thick table middle lines

% GRAMMAR
\usepackage[rounded]{syntax}
\setlength{\grammarparsep}{4pt plus 1pt minus 1pt}

\usepackage{amsmath}
\usepackage{amssymb}

\usepackage{multicol}
\setlength{\columnsep}{1cm}

% Graphs
\usepackage{tkz-berge}
\usetikzlibrary {positioning}
\definecolor {processblue}{cmyk}{0.96,0,0,0}
%%% STYLE OF PAGES NUMBERING
%\pagestyle{companion} %\nouppercaseheads 
%\pagestyle{headings}
%\pagestyle{ruled}
%\pagestyle{plain}
%\makepagestyle{plain}
% based on companion style
\makepagestyle{hoepelman}
%\newlength{\headwidth}
\setlength{\headwidth}{\textwidth}
  %\addtolength{\headwidth}{\marginparsep}
  %\addtolength{\headwidth}{\marginparwidth}
\makerunningwidth{hoepelman}{\headwidth}
\makeheadrule{hoepelman}{\headwidth}{\normalrulethickness}
\makeheadposition{hoepelman}{flushright}{flushleft}{}{}
\makepsmarks{hoepelman}{%
  \def\chaptermark##1{\markboth{##1}{##1}}    % left mark & right marks
  \def\sectionmark##1{\markright{%
    \ifnum \c@secnumdepth>\z@
      \thesection. \ %
    \fi
    ##1}}
  \def\tocmark{\markboth{\contentsname}{\contentsname}}
  \def\lofmark{\markboth{\listfigurename}{\listfigurename}}
  \def\lotmark{\markboth{\listtablename}{\listtablename}}
  \def\bibmark{\markboth{\bibname}{\bibname}}
  \def\indexmark{\markboth{\indexname}{\indexname}}}
\makepsmarks{hoepelman}{%
  \nouppercaseheads
  \createmark{chapter}{both}{nonumber}{}{}
  \createmark{section}{right}{shownumber}{}{. \space}
  \createplainmark{toc}{both}{\contentsname}
  \createplainmark{lof}{both}{\listfigurename}
  \createplainmark{lot}{both}{\listtablename}
  \createplainmark{bib}{both}{\bibname}
  \createplainmark{index}{both}{\indexname}
  \createplainmark{glossary}{both}{\glossaryname}}
%\makeevenfoot{hoepelman}{}{}{}
%\makeoddfoot{hoepelman}{}{}{}
\makeevenhead{hoepelman}{\normalfont\bfseries\thepage}{}{\normalfont\bfseries\leftmark}
\makeoddhead{hoepelman}{\normalfont\bfseries\thepage}{}{\normalfont\bfseries\leftmark}
%\makeoddhead{hoepelman}{\normalfont\bfseries\rightmark}{}{\normalfont\bfseries\thepage}

\pagestyle{hoepelman}

%%% CHAPTER'S STYLE
\makeatletter
\newcommand\thickhrulefill{\leavevmode \leaders \hrule height 1ex \hfill \kern \z@}
\setlength\midchapskip{10pt}
\makechapterstyle{VZ14}{
\renewcommand\chapternamenum{}
\renewcommand\printchaptername{}
\renewcommand\chapnamefont{\Large\scshape}
\renewcommand\printchapternum{%
\chapnamefont\null\thickhrulefill\quad
\@chapapp\space\thechapter\quad\thickhrulefill}
\renewcommand\printchapternonum{%
\par\thickhrulefill\par\vskip\midchapskip
\hrule\vskip\midchapskip
}
\renewcommand\chaptitlefont{\Huge\scshape\centering}
\renewcommand\afterchapternum{%
\par\nobreak\vskip\midchapskip\hrule\vskip\midchapskip}
\renewcommand\afterchaptertitle{%
\par\vskip\midchapskip\hrule\nobreak\vskip\afterchapskip}
}
\makeatother
\chapterstyle{VZ14}


%%% STYLE OF SECTIONS, SUBSECTIONS, AND SUBSUBSECTIONS
\setsecheadstyle{\Large\bfseries\sffamily\raggedright}
\setsubsecheadstyle{\large\bfseries\sffamily\raggedright}
\setsubsubsecheadstyle{\bfseries\sffamily\raggedright}


% Custom commands
\newcommand\textlcsc[1]{\textsc{\MakeLowercase{#1}}}

\newcommand{\todo}[1]{\textbf{TODO: #1}}
\newcommand{\ignore}[1]{}
%\usepackage[utf8]{inputenc}
\usepackage[T1]{fontenc}
\usepackage{microtype}
%\usepackage{fontspec}

%Default: computer modern

% 
%\usepackage{kpfonts}

% Times-like
%\usepackage{newtxtext}
%\usepackage{newtxmath}

% Times-like
%\usepackage{mathptmx}
%\usepackage{courier}

%Calibri
%\setmainfont{Calibri}

% Palatino
%\usepackage{newpxtext}
%\usepackage{newpxmath}

%\usepackage{libertine}
%\usepackage[libertine]{newtxmath}

\usepackage[scaled]{helvet}
\renewcommand\familydefault{\sfdefault}
\usepackage{sansmath}

\maxsecnumdepth{subsection} % Chapters, sections, and subsections are numbered
\maxtocdepth{subsection} % Chapters, sections, and subsections are in the Table of Contents

\newcommand{\todo}[1]{\textbf{TODO: #1}}
\newcommand{\ignore}[1]{}
\newcommand{\f}[1]{\texttt{#1}}
\newcommand{\key}[1]{\textbf{#1}}
\newcommand{\rf}[1]{\textsc{\lowercase{#1}}}
\newcommand{\fw}{\textbf{\{Note in Future work\}}}
\newcommand{\noparskip}[1]{{\parskip=0pt
#1
}}
\newcommand{\rom}[1]{\uppercase\expandafter{\romannumeral #1\relax}}

\nonzeroparskip

\begin{document}

\thispagestyle{empty}
\onecolumn
{%%%
\sffamily
\centering

~\vspace{\fill}

{\huge \bfseries
Inter-cell Spreadsheet Refactoring and Parsing Spreadsheet Formulas
}

\vspace{2.0cm}

by

\vspace{2.0cm}

{\LARGE
David Jonathan Hoepelman
}

\vspace{3.0cm}

\textlcsc{Master Thesis} \\
\today

\vspace{2.5cm}

For obtaining the degree of \\
Master of Science \\
in Computer Science - Software Technology \\

\vspace{0.5cm}

Faculty Electrical Engineering, Mathematics and Computer Science (EEMCS)\\
Delft University of Technology

\vspace{1.5cm}

\includegraphics{tudelft}
\hspace{0.5cm}
%\includegraphics[height=8mm]{serg}
%\hspace{0.5cm}
\includegraphics[height=8mm]{spreadsheet-lab}
\hspace{0.5cm}

\vspace{\fill}

}

\cleardoublepage

%%%---%%%---%%%---%%%---%%%---%%%---%%%---%%%---%%%---%%%---%%%---%%%---%%%
\tableofcontents*

\clearpage
%\twocolumn

%%%---%%%---%%%---%%%---%%%---%%%---%%%---%%%---%%%---%%%---%%%---%%%---%%%
%%%---%%%---%%%---%%%---%%%---%%%---%%%---%%%---%%%---%%%---%%%---%%%---%%%


\chapter{Introduction}
\label{chapter:introduction}

Like all people, I sometimes get asked what I do for a living.
When I tell someone I am writing my master thesis in Computer Science, their eyes start to glaze over as they anticipate some explanation peppered with terms they will not understand about.
I then tell them my thesis is about spreadsheets and ask if they have ever worked with Excel, and nearly everyone who has ever worked in business or research has.
Nearly everyone has a horror story about that one unmaintainable spreadsheet that they had to work on, or that day their reporting system broke down because 2009 turned into 2010 and the spreadsheet only looked at the last digit.

This anecdotal evidence is mirrored in research.
Panko \cite{panko2006facing} estimates that 80\% to 95\% of businesses use spreadsheets in one of their processes.
Furthermore, almost all spreadsheets contain at least one error, and 1 to 5\% of spreadsheet cells contains an error according to Panko \cite{panko1998we}.
Spreadsheets perform roles very similar to software in that they perform business-critical roles, are inherited throughout the organization and maintained by different users and accrue technical dept during and after the initial development period \cite{panko1998we}.
In short, spreadsheets can be classified as programs, and spreadsheet creators as end-user programmers.

This view, ``spreadsheets are code'', could be the one-sentence summary of the ideology of the Spreadsheet Lab, which is a part of the TU Delft Software Engineering Research Group (SERG).
Using this view as a baseline, the group works on translating tried and proven software engineering methods to the spreadsheet domain so that they can be used to improve spreadsheets, spreadsheet development practices and help spreadsheet programmers.
As part of this effort, a spreadsheet formula refactoring tool called BumbleBee was developed by Hermans and Dig \cite{hermans2014bumblebee}.
This tool allows a formula to be transformed into another by defining a transformation rule, which works very similar to a pattern or regular expression replacement in a text editor.

However this approach has the downside that it can only considers one formula, and not the spreadsheet as a whole.
This leads to a lack of power to implement all spreadsheet refactorings, such as those implemented by Badame and Dig \cite{badame2012refactoring} in earlier work.
I joined the group to extend the capabilities of BumbleBee so that it could take context into account when performing refactorings, and implement more refactorings in BumbleBee.

\begin{figure}
\centerfloat
\begin{subfigure}[c]{0.1\textwidth}
\f{=1+2+3}
\end{subfigure}
\begin{subfigure}[c]{0.08\textwidth}
$\xrightarrow[Ch.~\ref{chapter:parsing}]{Parsing}$
\end{subfigure}
\begin{subfigure}[c]{0.17\textwidth}
\begin{tikzpicture}[-latex ,auto ,node distance =1.3 cm and 0.5cm ,on grid , semithick,
,
state/.style ={ circle ,top color =white ,
draw, minimum width =0.75 cm}]
\node[state] (RootPlus) {$+$};
\node[state] (SecondPlus) [below left=of RootPlus] {$+$};
\node[state] (Input1) [below left=of SecondPlus] {1};
\node[state] (Input2) [below right=of SecondPlus] {2};
\node[state] (Input3) [below right=of RootPlus] {3};

\path (RootPlus) edge node {} (Input3);
\path (RootPlus) edge node {} (SecondPlus);
\path (SecondPlus) edge node {} (Input1);
\path (SecondPlus) edge node {} (Input2);
\end{tikzpicture}
\end{subfigure}
\begin{subfigure}[c]{0.14\textwidth}
$\xrightarrow[Ch.~\ref{sec:implementingrefactorings}]{Refactoring}$
\end{subfigure}
\begin{subfigure}[c]{0.17\textwidth}
\begin{tikzpicture}[-latex ,auto ,node distance =1.3 cm and 0.5cm ,on grid , semithick,
,
state/.style ={ circle ,top color =white ,
draw, minimum width =0.75 cm}]
\node[state] (Root) {$F$};
\node[state] (11) [below left=of Root] {\tiny{\f{SUM}}};
\node[state] (12) [below right=of Root] {[]};
\node[state, node distance = 1.3cm and 0.85cm] (21) [below left =of 12] {1};
\node[state, node distance = 1.32cm and 0.85cm] (22) [below =of 12] {2};
\node[state, node distance = 1.3cm and 0.85cm] (23) [below right=of 12] {3};

\path (Root) edge node {} (11);
\path (Root) edge node {} (12);
\path (12) edge node {} (21);
\path (12) edge node {} (22);
\path (12) edge node {} (23);
\end{tikzpicture}
\end{subfigure}
\begin{subfigure}[c]{0.1\textwidth}
$\xrightarrow[Ch.~\ref{sec:printing}]{Printing}$
\end{subfigure}
\begin{subfigure}[c]{0.145\textwidth}
\f{=SUM(1,2,3)}
\end{subfigure}
\caption{Overview of refactoring process}
\label{fig:refactoring-process}
\end{figure}

After the initial literature research I started implementing refactorings, but encountered a fundamental problem in doing so.
The standard way of implementing refactorings, illustrated in Figure \ref{fig:refactoring-process}, is by parsing the source code to an Abstract Syntax Tree (AST), which represents the structure of the program.
This AST can then be manipulated into the desired form, after which it can be converted back to source code (this is called printing or pretty-printing).
While BumbleBee contained a home-grown parser, I found a range of formulas that were either not parsable, or parsed into an incorrect AST.
This made me refocus the purpose of my thesis into making a better parser for Excel formulas, as this would not only be very useful for implementing refactorings but would be beneficial to all future spreadsheet research projects.
Using this new parser, I implemented several refactorings, which are described in Chapter \ref{chapter:implementingrefactorings}.

\section{About this thesis}

\subsection{Contribution}

The contributions of this thesis are twofold.
Firstly I improved and open-sourced a stand-alone formula parser called XLParser, which is available online\footnote{\url{https://github.com/spreadsheetlab/XLParser}}.
The parser was tested on over a million formulas and failed to parse merely two formulas.
Details of this parser are published by Aivaloglou, Hoepelman and Hermans \cite{xlparser}.
This paper is partially re-used in this thesis report.

The second contribution is an improved version of BumbleBee\footnote{\todo{URL Bumblebee when released.}}, which implements the refactorings described in Chapter \ref{chapter:implementingrefactorings}: \nameref{refac:extractformula}, \nameref{refac:inlineformula}, \nameref{refac:introducecellname}, \nameref{refac:groupreferences}, \nameref{refac:introduceaggregate} and \nameref{refac:introduceconditionalaggregate}.


Of these refactorings, \rf{Inline Formula}, \rf{Group References} and \rf{Introduce Conditional Aggregate} where not implemented in any previous work known to the author, and \rf{Extract Formula} and \rf{Introduce Aggregate} offer significant improvements over previous implementations.

\subsection{Attribution}

This thesis was performed at the TU Delft Spreadsheet Lab, and is partially based on a collaborative effort in that group.
During this thesis the Excel Formula parser XLParser was developed, and a paper describing it was accepted into the 15th IEEE International Working Conference on Source Code Analysis and Manipulation (SCAM 2015): \emph{A Grammar for Spreadsheet Formulas Evaluated on Two Large Datasets} by Efthimia Aivaloglou, David Hoepelman and Felienne Hermans \cite{xlparser}.
This paper is attached verbatim as published in appendix \ref{appendix:xlparser}.
Chapter \ref{chapter:anatomy} is based on ection \rom{2}, and was primarily written by the thesis author.
Chapter \ref{chapter:parsing} is an updated and extended version of Section \rom{3}, and was primarily written by the thesis author with contributions from Efthimia Aivaloglou, except for Chapter \ref{chapter:evaluation}, which incorporates parts of Section \rom{4} of the paper which was primarily written by Efthimia Aivaloglou, who also performed the evaluation. 

The grammar implementation (XLParser) was primarily done by the thesis author, but the implementation is based on the previous (un-named) parser which was primarily developed by Efthimia Aivalgoglou and Felienne Hermans.
The spreadsheet scantool used to evaluate XLParser, which extracts formulas from spreadsheet files, was developed as part of Infotron B.V.\footnote{\url{http://www.infotron.nl/}}, with many authors. The thesis author did not contribute to this tool.

The refactorings described in Chapter \ref{chapter:implementingrefactorings} were added to the existing BumbleBee Excel add-in developed by Felienne Hermans, but these refactorings were solely implemented by the author with very little of the existing infrastructure used.

\subsection{Outline}

Chapter \ref{chapter:previouswork} details previous and related work on spreadsheet refactoring.
A passing knowledge of Excel and more in-depth knowledge of Excel formulas is needed to read the rest of this thesis, which is bundled in Chapter \ref{chapter:anatomy} "\nameref{chapter:anatomy}".

Chapter \ref{chapter:parsing} covers how XLParser parses spreadsheet formulas and why it was designed as it was.
Chapter \ref{chapter:implementingrefactorings} describes which spreadsheet refactorings were implemented and how this was done.
Chapter \ref{chapter:evaluation} provides an overview of the evaluation done and Chapter \ref{chapter:conclusion} contains concluding remarks.

\clearpage
\section{Timeline and decisions taken}

\begin{tabularx}{\textwidth}{lX}
\toprule
December 2014 & Literature study on refactoring, refactoring spreadsheets, converting spreadsheets to programs. \\
& Thesis topic selection. \\
January 2015 & Studied practicality of generic spreadsheet refactoring language based on Bumblebee transformation language, deemed inviable. \\
& Gathered existing refactorings from spreadsheet literature and translated Fowler refactorings. \\
& Decided which refactorings to initially implement \\
& Familiarization with existing BumbleBee code \\
February 2015 & Implementing \rf{Inline Formula} \\
& Extended Bumblebee and parser to account for sheet and file names \\
March 2015 & Implementing \rf{Extract formula} \\
& Writing of paper "End user programming" \footnote{First version was rejected from IEEE special issue on Refactoring: Accelerating Software Change. A significantly extended and improved version was submitted to ICSE later, but I did not contribute to these changes.} \\
April 2015 & Writing of paper "End user programming" \\
& Various improvements to parser \\
May 2015 & Decision to rewrite parser to solve several fundamental problems \\
& Start work on XLParser \\
& Implementing \rf{Introduce (Conditional) Aggregate} \\
& Implementing \rf{Group references} \\
June 2015 & Continued work on XLParser: Initial release \\
& Writing of XLParser paper \\
July 2015 & Changing of refactoring UI to context-aware right-click menu, similar to IDE's \\
August 2015 & Continued work on XLParser: Several fixes to XLParser parse trees \\
& Camera-ready adjustment of XLParser paper \\
& Constructed demo application\footnote{\url{http://xlparser.perfectxl.nl/demo/}} for XLParser which shows the parse trees. \\
September 2015 & Continued work on XLParser: Adding structured references, file paths \\
& Porting BumbleBee and refactorings to XLParser-based implementations \\
& Writing of this thesis \\
October 2015 & Implementing \rf{Introduce cell name} \\
& Porting BumbleBee and refactorings to XLParser-based implementations. \\
& Writing of this thesis \\
\bottomrule
\end{tabularx}

% !TeX root = ../thesis.tex

\chapter{Previous and related work}
\label{chapter:previouswork}

\section{Refactoring}

Refactoring is ``the process of changing a software system in such a way that it does not alter the external behaviour of the code, yet improves its internal structure'' \ref{opdyke1992refactoring}.
The process is probably as old as programming itself, and was known since at least 1986 as ``Restructuring'' \ref{arnold1986introduction}.
The term ``Refactoring'' was coined by Opdyke in 1992 \ref{opdyke1992refactoring} and originally specifically meant the restructuring of Object Oriented Programs. 
Over time the specific the popularity of both the practice and the term ``Refactoring'' increased, greatly helped by Fowler's 1999 ``Refactoring: improving the design of existing code'' \cite{fowler1999refactoring} and the rise of more flexible ``Agile'' software development.
Currently, all major programming Integrated Development Environments like Visual Studio, Eclipse, Netbeans and IntelliJ include support for automated software refactorings.

\section{Refactorings provided by excel}

Several useful refactorings one could think of are already provided by Excel, although Excel does not call them refactorings.

Firstly the cut, copy and paste functionality of Excel is very interesting.
If a user cuts a selection of cells and pastes it elsewhere, all references to those cells made in unselected will be moved as well.
For example if a user cut-pastes cell \f{A1} to location \f{C3}, the formula \f{=A1} will be changed to \f{=C3} in other cells, even though they were not selected by the user.
This is very similar to the \rf{Move Method} \cite{fowler1999refactoring} refactoring, because not only are the cells (method contents) themselves moved, references to them (call sites) are a adjusted for the new location as well.
Another option Excel provides when copy-pasting is ``Paste Values'', with which a formula is replaced by its evaluated value and remains constant from that point on.

Excel having built-in support for these refactorings shows that there is a need among spreadsheet users to refactor their spreadsheets, and that advanced spreadsheet users are likely already comfortable with the concept of refactoring, albeit it not by name.

\section{Spreadsheet refactoring}

\section{The BumbleBee spreadsheet refactoring suite}

%\chapter{Spreadsheet anatomy}
% !TeX root = ../thesis.tex
% !TeX spellcheck = en_US

\chapter{Anatomy of spreadsheets and spreadsheet formulas}
\label{chapter:anatomy}

\noindent
\begin{figure}[h]
\centerfloat
\input{anatomy/refactoring-process-anatomy}
\caption{This chapter details the boxed elements of the refactoring process: the properties and syntax of spreadsheets and formulas strings.}
\end{figure}

In order to be able to create a parser or refactor-tool for spreadsheets, a solid understanding of their inner workings is necessary, especially of the formula language.

By a large margin, the most widely used spreadsheet system is Microsoft Office Excel, which has a self-reported install base of 1.2 billion users \cite{microsoftByTheNumbers}.
Two less-popular but still common implementations are Apache OpenOffice Calc and LibreOffice Calc, which evolved from the same product and thus have largely identical semantics.
LibreOffice has a self-reported install base of 75 million users.
Google Sheets is another common implementation, and is special because it was the first large web-based collaborative spreadsheet program.
Google does not report usage numbers, but in 2010 has said it is used by "tens of millions" of users.
There are many other spreadsheet implementations, but none of them come close to Microsoft Excel's market share, and most are smaller than LibreOffice and Google Spreadsheets.
When referencing "all" spreadsheet programs, this indicates the three implementations previously mentioned: Microsoft Excel, LibreOffice and Google Sheets.
These three implementations are the systems studied because of their market share.

\newpage

Although exotic forms of spreadsheets are available and have been researched, all mainstream implementations use the following model:
\begin{itemize}
\item A single spreadsheet \key{file} corresponds to a single (\key{work})\key{book}.
\item A workbook can contain any number of (\key{work})\key{sheets}.
\item A sheet consists of a \key{two-dimensional grid} (table) of \key{cells}.
\item A vertical unit in the grid is called a \key{column} and a horizontal unit a \key{row}.
Rows are numbered sequentially top-to-bottom starting at 1, while columns are numbered left-to-right alphabetically, i.e. base-26 using A to Z as digits.
A column or row can also mean all cells contained in that column or row.
\item A \key{cell} can contain a \key{constant value} of any type, a calculation called a \key{formula} or a matrix calculation called an \key{array formula}.
\item An (array) formula can \key{reference} other cells to use their values. When the value of a referenced cell changes, this new value is propagated and the dependent formula values are recalculated.
\end{itemize}

\begin{figure}
\centerfloat
\begin{subfigure}[b]{0.45\textwidth}
\centering
\scalebox{0.71}{
\begin{tikzpicture}[-latex ,auto ,node distance =1.3 cm and 1.8cm ,on grid , semithick,
,
state/.style ={ circle ,top color =white ,
draw, minimum width =0.9 cm}]
\node[state] (Input3) {3};
\node[state] (Input5) [below=of Input3] {5};
\node[state] (Plus) [right=of Input5] {$+$};
\node[state] (Input2) [below=of Input5] {2};
\node[state] (Times) [below right=of Plus] {$\times$};
\node[state] (Output) [right=of Times] {\small{\texttt{out}}};
\path (Input3) edge [right] node[right] {} (Plus);
\path (Input5) edge [right] node[right] {} (Plus);
\path (Plus) edge [right] node[right] {} (Times);
\path (Input2) edge [right] node[right] {} (Times);
\path (Times) edge [right] node[right] {} (Output);
%\path (A) edge [loop left] node[left] {$1/4$} (A);
%\path (C) edge [bend left =25] node[below =0.15 cm] {$1/2$} (A);
%\path (A) edge [bend right = -15] node[below =0.15 cm] {$1/2$} (C);
%\path (A) edge [bend left =25] node[above] {$1/4$} (B);
%\path (B) edge [bend left =15] node[below =0.15 cm] {$1/2$} (A);
%\path (C) edge [bend left =15] node[below =0.15 cm] {$1/2$} (B);
%\path (B) edge [bend right = -25] node[below =0.15 cm] {$1/2$} (C);
\end{tikzpicture}
}
\caption{Dataflow program}
\end{subfigure}
\hspace{0.05\textwidth}
\begin{subfigure}[b]{0.45\textwidth}
\includegraphics[width=\textwidth]{anatomy/dataflow-program-excel-implementation}
\caption{Spreadsheet implementation}
\end{subfigure}
\caption{An example dataflow program and its spreadsheet implementation}
\label{fig:dataflow-example}
\end{figure}

This model is a variation of the dataflow programming model.
A dataflow program is a directed graph, where data flows between operation in nodes along the graphs edges.
In spreadsheets, cells represent the nodes of a dataflow program and edges are represented by references.
An example dataflow program and its spreadsheet implementation can be seen in Figure \ref{fig:dataflow-example}.

The spreadsheet model is Turing complete, as proven by an Excel 2010 implementation of a Turing machine \cite{ExcelTuringComplete}.

\section{Formulas}

All spreadsheet programs currently use a 
Currently two major formula languages exist in spreadsheets. The first is the Office Open XML spreadsheet language which is a standarization of the Excel language, which is currently used by Microsoft Excel.
The second is OASIS OpenDocument OpenFormula, which is part of the OpenDocument format which is based on the OpenOffice.org format.
Google Sheets claims to implement OpenFormula, but has some differences which are noted in their specific sections.
Both languages are very similar and this section covers both unless otherwise noted.

Formulas consist of an expression which can contain constant values, function calls and operators and, most importantly, references to other cells.
A cell is identified as a formula cell because all formulas must start with the equals sign \f{=}.

\subsection{Function calls and operators}

Function calls are performed, similar to other programming languages, by starting with the function name, followed by the arguments in parentheses, separated by a comma.
All spreadsheet implementations provide a range of built-in functions, and in most spreadsheet implementations it is possible to define new functions yourself.
However in current implementations this is not done directly inside the spreadsheet, instead using an alternate programming language.
In Microsoft Excel and LibreOffice this is done with a variant of the BASIC programming language, while in Google Sheets this is done with Javascript.
A way for the user to define functions in the spreadsheet itself has been proposed by Peyton Jones et. al \cite{jones2003user}, but as of now has not been implemented in mainstream spreadsheet programs yet.

The binary operators \f{+ - * / = >= <= < >} and \f{<>} (inequality) can be used according to their usual semantics.
\f{+} and \f{-} are available both unary (\f{=-1}) and binary (\f{=1-1}).
Additionally the \f{\%} postfix unary operator is defined to transform a number into a percentage (divining it by 100), \f{\textasciicircum} is the exponentiation operator and \f{\&} is the text concatenation operator.

Spreadsheet program contain three fairly unique binary operators, the semantics of which are detailed in Section \ref{sec:references}.
Firstly there is the range operator \f{:} then the union operator (\f{,} in Excel and \f{\textasciitilde} in OpenFormula) and lastly the intersection operator (\f{\char32} in Excel and \f{!} in OpenFormula).
Note that OpenFormula diverges from the Excel syntax, possibly because the comma already is used in other places in the language and a single space as an operator is highly unusual.
The Excel characters for the operators will be used in the remainder of this thesis.

\subsection{References}
\label{sec:references}
References are the core component of spreadsheets.
The value of any cell can be used in a formula by concatenating its column and row number, producing a reference like \f{B5}.
This is called A1-style referencing and is by far the most comment in modern spreadsheet implementations.
If the value of a cell changes, this new value will be propagated to all formulas that use it.

When copying a cell to another cell, by default references will be adjusted by the offset, for example copying \f{=A1} from cell B1 to C2 will cause the copied formula to become \f{=B2}.
This can be prevented by making the reference absolute by prepending a \f{\$} to the column index, row index or both.
The formula \f{=\$A\$1} will remain the same on copy while \f{=\$A1} will still have its row number adjusted when copied, as ilustrated in Figure \ref{fig:copy-modifiers}.

\begin{figure}
\centerfloat
\includegraphics[width=0.5\textwidth]{anatomy/copying}
\caption{Different copy-paste behavior depending on \f{\$} modifier, copy-direction is given by the arrow.}
\label{fig:copy-modifiers}
\end{figure}


\subsection{Ranges}
References can also be \emph{ranges}, which are collections of cells.
Ranges can be constructed by three operators: the range operator \f{:}, the union operator \f{,} (a comma) and the intersection operator \f{\char32} (a space).
The range operator \f{:} creates a rectangular range with the two cells as top-left and bottom-right corners, so \f{=SUM(A1:B10)} will sum all cells in columns A and B with row number 1 through 10.
The range operator is also used to construct ranges of whole rows or columns, for example \f{3:5} is the range of the complete rows three through five, and \f{A:D} is the range of columns A through D.
The union operator, which is different from the mathematical union as duplicates are allowed, combines two references, so \f{A1,C5} will be a range of two cells, \f{A1} and \f{C5}.
Lastly the intersection operator takes only the cells which are in both arguments, \f{=A:A 5:5} will thus be equivalent to \f{=A5}.

A user can also give a name to any collection of cells, thus creating a \emph{named range} which can be referenced in formulas by name.
For example one can give the name \f{TAX\_RATE} to cell \f{A2} and then use this in a formula: \f{=C3+C3*TAX\_RATE} instead of \f{=C3+C3*\$A\$2}.

\subsection{Structured References}

A recent, Excel 2007, addition to the Excel formula language are structured (table) references.
To use this feature, a table must be given a name and column headers.
One can then reference a column in the table by entering \f{TableName[ColumnName]}.
Inside the square brackets reference operators can be used to construct more complex references, \f{TableName[Column1,Column4]} references two columns.

There is no way to reference a specific row, except the current row, for example if a formula is placed in \f{A3} it can only reference row number 3.
The \f{\#This Row} keyword and the \f{@} operator are used for this: \f{TableName[\#This Row]} and \f{TableName[@]} both reference the current row number in the provided table, and \f{TableName[@ColumnName]} references the cell in the provided column of the current row number.

This feature is meant to make formulas easier to read, by references in a formula like \f{=SUM(F1:F10)-SUM(I1:I10)} with human-readable names like \f{=SUM(Budget[Revenue]) - SUM(Budget[Expenses])}.

\subsection{R1C1 reference style}

An alternate style called R1C1 as opposed to the above A1 style exists, but it is only rarely used by users in modern spreadsheet implementations.
In R1C1 reference style one specifies either the offset to a cell between square brackets or its concrete location.
In R1C1 style \f{R[4]C[-2]} means the cell two columns to the left and four rows down, while \f{R2C2} refers to cell B2.
The biggest advantage of R1C1 is that it causes identical formulas to be the same even when they operate on different cells or data because of their position, illustrated in Figure \ref{fig:r1c1comp}.
These properties make R1C1 useful as an internal representation in spreadsheet implementations and in a spreadsheet refactoring tool.

\begin{figure}
\centerfloat
\begin{subfigure}[t]{0.35\textwidth}
\includegraphics[width=1\textwidth]{anatomy/r1c1comp-a1}
\caption{A1-style formulas}
\end{subfigure}
\hspace{0.1\textwidth}
\begin{subfigure}[t]{0.35\textwidth}
\includegraphics[width=1\textwidth]{anatomy/r1c1comp-r1c1}
\caption{R1C1-style formulas}
\end{subfigure}
\caption{A1 vs R1C1 behavior on identical formulas with respect to cell position}
\label{fig:r1c1comp}
\end{figure}

\subsection{Non-local references}
\label{subsection:ExternalRefsDDE}

References refer to cells or ranges in the same sheet as the formula by default, but this can be modified with a prefix. A non-local reference consists a prefix indicating the location, followed by an exclamation mark, followed by the actual reference.

The most common use case is to reference another sheet in the same workbook, where the prefix is simply the sheet name: \f{=Sheetname!A1}.
Sheet names can also be between single quotes if they contain special characters: \f{='Sheetname with space'!A1}. 
References to external spreadsheet files are also possible, which is done by providing the file name in between square brackets and optionally the file path: \f{=[Filename]Sheetname!A1} or \f{='C:{\textbackslash}Path{\textbackslash}[Filename]Sheet'!A1}.
A peculiar type of prefix are those that indicate multiple sheets: \f{=Sheet1:Sheet10!A1} means A1 in Sheet1 through Sheet10.

In Windows versions of Microsoft Excel formulas can also call external programs through Dynamic Data Exchange (DDE).
DDE links are a special case of references, used for receiving data from other applications.
They take the form of \f{=Program|Topic!Arguments}, e.g. \f{=Database|TableA!Column1}.

\subsection{Case sensitivity}

Formulas are case-insensitive outside of the trivial case of string literals.
Identifiers have a canonical capitalization, and while user can type the identifier with any casing only the canonical form will be displayed.
While the canonical capitalization of built-in identifiers,functions and reserved names is usually uppercase, the canonical capitalization of an user-defined identifier, an user defined function or a named range, is as the user defined it originally.

\subsection{Whitespace sensitivity}

The Excel formula language is whitespace sensitive in several places:
\begin{itemize}
\item Whitespace is not allowed between function names and the argument list: \f{=SUM  (1)} is invalid.
\item Whitespace is not allowed inside internal or external references: \f{=Sheet1 !A1} is invalid.
\item The intersection operator is a single space: \f{=A:A 3:3} is the intersection of column \f{A} and row \f{3}, equivalent to \f{=A3}. (Excel formula language only)
\end{itemize}

\newpage

\section{Array Formulas and Arrays}
\label{sec:arrayformulas}
In spreadsheet formulas it is possible to transform one- or two-dimensional matrices.

When constructed from constant values they are called \emph{array constants}, e.g. \f{\{1,2;3,4\}} constructs a two-by-two matrix.
They are surrounded by curly brackets, columns are separated by commas, and rows by semicolons.
Several matrix operations are available, for example \f{=SUM(\{1,2,3\}*10)} will evaluate to 60.

\emph{Array formulas} use the same syntax as normal formulas, except that the user must enter \emph{Ctrl} + \emph{Shift} + \emph{Enter} to signal that it is an Array formula.
Excel and LibreOffice surround the formula with curly braces.
Google Sheets works differently and requires the user to surround an array formula with \f{ARRAYFORMULA($\ldots$)}.

Marking a formula as an array formula will enable one- or two-dimensional reference ranges to be treated as matrices, and several matrix operators and functions will be available. 
For example if \f{A1},\f{A3},\f{A3} contain the values \f{1},\f{2},\f{3} the array formula \f{\{=SUM(A1:A3*10)\}} will evaluate to \f{60}.
Furthermore, an array formula allows the user to return multiple results, which will be presented in multiple cells.
The array formula \f{\{=\{1,2,3\}*\{4,5,6\}\}} will show \f{4}, \f{10} and \f{18} in three different cells.

\section{Type system}

The type system in Microsoft Excel is weak with most types being able to be coerced into others.
For values inside formulas, the following types exist:

\begin{itemize}
\item[Boolean] values are either \f{TRUE} or \f{FALSE}. Booleans can be coerced to numbers where \f{TRUE} will become 1 and \f{FALSE} will become 0 and strings.
\item[Numeric] values are in the range of 8-byte IEEE doubles. Numbers can be provided as integers, decimals or in scientific notation.
When coerced to booleans 0 will become \f{FALSE}, all other values will be \f{TRUE}. Numbers can also be coerced into strings, or type-casted with the \f{TEXT} function.

\item[String] values are any Unicode character enclosed in quotation marks \f{"}.
Two quotation marks serve as the escape character, thus \f{""""} represent the string ".
If the contents of a cell start with a \texttt{'} the rest of that cell content is interpreted as a string.

When coerced to booleans all strings except the empty string are \f{TRUE}, the empty string is \f{FALSE}.
When coerced to a numeric value the spreadsheet program will accept any string representing valid numeric user input and otherwise give the error \f{\#VALUE!}. Explicit conversion to a numeric value is done with the \f{VALUE} function.

\item[Error] values are \f{\#DIV/0!}, \f{\#NAME?}, \f{\#NULL!}, \f{\#NUM!}, \f{\#N/A!}, \f{\#VALUE!} and \f{\#REF!}.
Errors behave similar to exceptions in that they will propagate throughout a calculation. Errors cannot be coerced.

\item[Ranges and arrays] are one- or two-dimensional matrices of any non-array values. Arrays are rarely used outside of array formulas, but ranges are very common in formulas.
However, these types usually only serve as inputs for functions and are thus fairly transparent to the user outside of array formulas.
Both types usually cannot generally be coerced, doing so will result in the \f{\#VALUE!} error.
\end{itemize}

Some other "display types" exists, these can change the way the data is presented to or validated from the user and can have implications when inter-operating with other programs.
Usually the user can mark a cell as containing one of these types, or Excel can automatically mark a cell to be of this type based on heuristics.
In formulas and internally these types are all represented by one of the above types.
A few of these are commonly used:
\begin{itemize}
\item[Dates and times] are internally stored as a floating point with the integer portion being the number of days since the epoch January 1st 1900, incorrectly considering 1900 a leap year, and the remainder being the portion of the day that passed.
Excel displays dates and times as is customary in the locale of the user.
When interoperating a date or time value will be exported as a datetime type value of that system.
\item[Currency] is stored as other numbers and displayed in the format customary for the specific currency. When interoperating with some other systems currency values will be exported using arbitrary-precision arithmetic formats.
\item[Percentages] are stored as other numbers, but displayed as if multiplied by 100\%.
\end{itemize}


%\chapter{Candidate spreadsheet refactorings}
%\label{chapter:candidaterefactorings}
%
%\section{Comparing spreadsheets to other programming paradigms}
%
%\subsection{Reactive programming}
%As explained in Chapter \ref{chapter:anatomy}, the spreadsheet programming model is a variation of the dataflow programming model.
%Thus the most natural first comparison for spreadsheet is the \emph{reactive programming} paradigm, which is another variation of the dataflow programming model.
%Viewed from the RP model, all spreadsheet cells are observable values and every formula is an observer.
%One can attach a formula observer to a cell observable by making a reference, e.g. a formula \f{=A1+1} in cell B1 listens to the value of A1 and produces the value for B1.
%However currently no literature on refactoring reactive programs is known to the author, so no additional refactoring were found by comparing spreadsheets to reactive programs.
%
%\subsection{Functional programming}
%Spreadsheets program do not contain state (mutable variables) or side-effects
%\footnote{As long as one remains within the spreadsheet model. User-defined functions and inter-operation with external programs can cause side-effects.}
%and this immutability aspect is often associated with functional programming languages.
%However spreadsheets lack several other important concepts of functional languages such as first-class and higher-order functions, recursion\footnote{Most spreadsheet implementations allow for a form of recursion in the form of cycles in the dependency graph called circular references. However this is not the normal modus-operandi, triggers a warning and is generally only used to perform iterative calculations which are not otherwise possible in spreadsheets.}, lazy evaluation and type interference.
%
%\subsection{Translation of concepts}
%
%Some literature exist on refactoring functional programs, most prominent is the tool \textbf{Ha}skell \textbf{Re}factoring tool (HaRe) \cite{thompson2005refactoring}.
%Most of the described refactorings are either specific to Haskell, or only applicable because of concepts not available in spreadsheets.
%
%This makes all of the current literature on refactoring functional programs not applicable.
%\todo{Cite functional refactoring literature}
%
%\subsection{Object-oriented programming}
%
%By far 
%
%
%
%\section{Translating refactorings to the spreadsheet domain}
%
%Tabel met mogelijke refactorings.
%Subsecties voor alle geïmplementeerde refactorings.

%\chapter{Parsing spreadsheet formulas}
% !TeX root = ../thesis.tex
% !TeX spellcheck = en_US

\chapter{Parsing spreadsheet formulas}
\label{chapter:parsing}

\noindent
\begin{figure}[h]
\centerfloat
\input{parsing/refactoring-process-parsing}
\caption{This chapter details parsing, converting a string formula to an AST.}
\end{figure}


\noindent
This chapter assumes the reader is familiar with basic parser theory, a good overview of which can be found in \cite{dragonbook}.

\section{Motivation}
\label{sec:motivation}

In order to implement refactorings of spreadsheets a refactoring tool must be able to manipulate spreadsheets.
Excel exposes an API to retrieve and set the contents of a spreadsheet file, but this API only works on a formula string level.
The usual way to implement refactorings is by manipulating the original program's AST until it represents the desired program, and then print that back to a string.
Because Excel does not expose the AST or parse tree of formulas, a refactoring tool needs to contain a parser for Excel formulas.

Bumblebee relied on a parser developed over the years for previous research, however this parser suffered from having new rules added on over time, often in an inconsistent manner and not fully supporting the whole language.
Furthermore the parser interpreted some language constructs wrong and missed several features, which made implementing refactorings hard.
For example operator precedence was not taken into account, causing the formula \f{=A1 + A2 * A3} to be parsed as \f{=(A1 + A2) * A3}, which subtly and unexpectedly breaks numerous refactorings.
The existing parser thus was insufficient to implement the Bumblebee refactorings.
A better parser was needed, thus the goal of this thesis shifted toward procuring a better parser for Excel formulas.

For Bumblebee and other research on spreadsheet formulas the following design goals were formulated for the parser and grammar within the spreadsheet lab group:

\begin{enumerate}
\label{sec:designgoals}
\item The parser must be compatible with the official language
\item Produced parse trees must be suited for further manipulation and analysis with minimal post-processing required
\item The grammar must be compact enough to feasibly implement with a parser generator
\end{enumerate}

While an official grammar for Excel formulas is published \cite{ExcelOfficialGrammar}, it does not meet the above requirements for two reasons.
Firstly, it is over 30 pages long and contains hundreds of production rules and thus fails Requirement 3.
Secondly, because of the detail of the grammar and the large number of production rules the resulting parse trees are very complex and fail requirement 2.

Because there is no suitable parser and grammar available that satisfy the above requirements, I decided to clean up and partially rewrite the parser in coorporation with other members of the Spreadsheet lab.
The end result of this effort is an independent, open-source parser for Excel formulas called XLParser\footnote{https://github.com/PerfectXL/XLParser}, about which a paper was published in IEEE conference SCAM 2015 \cite{xlparser}.

\section{Improvements over existing parser}

The improvements XLParser made over the existing parser generally fall into one of the following categories:

\subsection{More frequent rejecting of invalid formulas}

XLParser is often less forgiving than the previous parser, and rejects more types of invalid formulas.
This is most prominently noticable in reference expressions, becaues the old parser does not differentiate between reference and non-reference expressions.
Therefore formulas like \f{=1 1} and \f{=LARGE((1,2,3),4)} are considered valid, while they are not and would be rejected by Excel.

\subsection{Broader parsing of valid formulas}

As can be found in Chapter \ref{chapter:evaluation}, XLParser has a very high parse success rate.

Several language features were absent in the previous parser.
Examples are ranges with multiple limits (\f{=SUM(A1:B2:C3:D4)}), structured table references (\f{=TableName[ColumnName]}), array constants \f{=SUM(\{1,2,3\})} and functions in reference expression (\f{=SUM(IF(TRUE,A1,B2):C5)}).
While sometimes these features were very rare, they still were encountered in the available datasets.

Furthermore the previous parser relied on a tool which extracted the formulas as stored in spreadsheets, while BumbleBee is used as an Excel add-in and therefore receives its formulas from Excel.
These formulas sometimes slightly differ in at least one aspect: when external files are referenced a numeric reference is stored while Excel provides either the filename or the file path and name.
Thus a formula could be received as \f{=[1]Sheet!A1} from the tool, and \f{=[File]Sheet!A1} or \f{='C:\textbackslash Path\textbackslash [File.xlxs]Sheet'!A1} by an Excel Add-In.
XLParser supports al three formats, while the previous parser only supported the first.

\subsection{AST improvements}

\subsubsection{Correctness}

While the AST correctness is unverified for both XLParser and the previous parser, several improvements have been made.
Operator precedence has been mentioned before, this was not taken into consideration in the previous parser version, providing very problematic in BumbleBee's use-case.
Several smaller corrections have also been made.
For example in the previous version \f{=F(1,,1)} and \f{=F(1,1)} produced an identical AST, while they have a different meaning, especially in the case of user defined functions.

\subsubsection{Homogenization}
The previous parser was constructed with a relatively clean base grammar, but over time rules were added in a fairly ad-hoc manner.
This caused the rules and therefore the AST to become quite messy in some cases.
XLParser solved this by reducing the grammar to a relatively clean grammar again.
However, this advantage is subjective and hard to quantify.
An example of this are User Defined Functions which produces a very different AST depending on whether they were internal \f{=UDF()} or external \f{=[1]UDF()}.
Another example are prefixes, all of the following used different tokens and productions rules: \f{=Sheet!A1}, \f{='Sheet'!A1}, \f{=[1]Sheet!A1} and \f{='[1]Sheet'!A1} while in XLParser the tokens are unformalized, and the production rules are cleaner in the authors opinion.


\section{Parser implementation}

The existing parsing was built using the Irony parser framework\footnote{https://irony.codeplex.com/}, which is a C\# parser generator that produces parsers based on the LALR(1) algorithm using a grammar defined in C\#.

Strictly speaking Irony produces a parse tree, however this tree is fairly high-level for a parse tree, leaving out elements such as punctuation and whitespace, and no separate AST is (currently) constructed in XLParser or BumbleBee, instead this tree is directly manipulated.
To avoid confusion and use usual nomenclature I will from now on refer to this tree as the AST.

\subsection{Lexical Analysis}
\label{sec:lexanalysis}

\begin{table}
\tiny
\centerfloat
%\advance\leftskip-1cm
\stepcounter{footnote}

\begin{tabular}{@{}llll@{}}
	\toprule
	Token Name & Description & Contents & Priority \\
	\midrule
	%BINOP & Binary Operator & + $\mid$ - $\mid$ / $\mid$ * $\mid$ \textasciicircum $\mid$ \textless $\mid$ \textgreater $\mid$ = $\mid$ \textless= $\mid$ \textgreater= $\mid$ \textless \textgreater & 0 & + \\
	BOOL & Boolean literal & TRUE $\mid$ FALSE & 0 \\
	CELL & Cell reference & \$? [A-Z]+ \$? [0-9]+ & 2 \\
	DDECALL & Dynamic Data Exchange link & ' ([\textasciicircum{} '] $\mid$ '')+ ' & 0 \\
	ERROR & Error literal & \begin{tabular}[c]{@{}l@{}} \#NULL! $\mid$ \#DIV/0! $\mid$ \#VALUE! \\ $\mid$ \#NAME? $\mid$ \#NUM! $\mid$ \#N/A \end{tabular} & 0 \\
	ERROR-REF & Reference error literal & \#REF! & 0 \\
	EXCEL-FUNCTION & Excel built-in function & (Any entry from the function list\begin{footnotesize}\textsuperscript{\number\value{footnote}}\end{footnotesize}) \textbackslash( & 5        \\
	FILE & External file reference using number & \textbackslash[ [0-9]+ \textbackslash] & 5 \\
	FILENAME & External file reference using name & \textbackslash[ $\Box_4$+ \textbackslash] & -1 \\
	FILEPATH & Windows file path & [A-Z] : \textbackslash\textbackslash\ ($\Box_4$+ \textbackslash\textbackslash)* & 0 \\
	HORIZONTAL-RANGE & Range of rows & \$? [0-9]+ : \$? [0-9]+ & 0 \\
	MULTIPLE-SHEETS & Multiple sheet references & 
	%$\Box_2$+ : ($\Box_2$+ $\mid$ '($\Box_3$ $\mid$ '')+')!
	(($\Box_2$+ : $\Box_2$+)|( ' ($\Box_3$ $\mid$ '')+ : ($\Box_3$ $\mid$ '')+ ' )) !
	& 1 \\
	NAME & User Defined Name & [A-Z\_\textbackslash\textbackslash][A-Z0-9\textbackslash\textbackslash\_.$\Box_1$]* & -2 \\
	NAME-PREFIXED & \begin{tabular}[c]{@{}l@{}} User defined name which starts with \\ a string  that could be another token \end{tabular} & (TRUE $\mid$ FALSE $\mid$ [A-Z]+[0-9]+)    {[}A-Z0-9\_.$\Box_1${]}+                                                                             & 3 \\
	NUMBER & \begin{tabular}[c]{@{}l@{}}An integer, floating point\\     or scientific notation number literal\end{tabular} & [0-9]+ ,? [0-9]* (e [0-9]+)? & 0 \\
	REF-FUNCTION & Excel built-in reference function & (INDEX $\mid$ OFFSET $\mid$ INDIRECT)\textbackslash( & 5 \\
	REF-FUNCTION-COND & Excel built-in conditional reference function & (IF $\mid$ CHOOSE)\textbackslash( & 5 \\
	RESERVED-NAME & An Excel reserved name & \_xlnm\textbackslash.  [A-Z\_]+ & -1 \\
	SHEET & The name of a worksheet & $\Box_2$+ ! 
	& 5        \\
	SHEET-QUOTED & Quoted worksheet name & $\Box_3$+ ' ! & 5 \\
	STRING & String literal & " ([\textasciicircum{} "] $\mid$ "")* " & 0       \\
	SR-COLUMN & Structured reference column & \textbackslash[ [A-Z0-9\textbackslash\textbackslash\_.$\Box_1$]+ \textbackslash] & -3 \\
	%UNOP\_POSTFIX & Unary postfix operator & \% & 0 & \% \\
	%UNOP\_PREFIX & Unary prefix operator & + $\mid$ - & 0 & -                  \\
	UDF & User Defined Function & (\_xll\textbackslash.)? [A-Z\_\textbackslash][A-Z0-9\_\textbackslash\textbackslash.$\Box_1$]*  ( & 4 \\
	VERTICAL-RANGE & Range of columns & \$? [A-Z]+ : \$? [A-Z]+ & 0 \\ 
	\midrule
	Placeholder character & Placeholder for & Specification & \\
	$\Box_1$ & Extended characters & \begin{tabular}[c]{@{}l@{}} Non-control Unicode characters x80 and up \end{tabular} & \\
	$\Box_2$ & Sheet characters & \begin{tabular}[c]{@{}l@{}} Any character except\\ ' * [ ] \textbackslash\ : / ? ( ) ; \{ \} \# " = < > \& + - * / \textasciicircum{} \% , \texttt{\char32} \end{tabular}& \\
	$\Box_3$ & Enclosed sheet characters & \begin{tabular}[c]{@{}l@{}} Any character except ' * [ ] \textbackslash\ : / ? \end{tabular} & \\
	$\Box_4$ & Filename characters & Any character except " * [ ] \textbackslash\ : / ? < > $\mid$ \\
	\bottomrule
	\multicolumn{4}{l}{\begin{footnotesize}\textsuperscript{\number\value{footnote}} A function list is available as part of the reference implementation.\end{footnotesize}}\\
	\multicolumn{4}{l}{\begin{footnotesize} \hspace{0.5em} Lists provided by Microsoft are also available in \cite{ExcelFunctionReference} and \cite{ExcelOfficialGrammar}.\end{footnotesize}}\\
	
\end{tabular}

\stepcounter{footnote}
\caption{Lexical tokens used in the XLParser grammar, as refered to in section \ref{sec:lexanalysis}.}
\label{table:tokens}
\end{table}

Table \ref{table:tokens} contains the lexical tokens of the grammar, along with their identification patterns in a simple regular expression language. All tokens are case-insensitive.
Characters are defined as Unicode code points x9 (tab),xA (newline),xD (carriage return) and x20 (space) and upwards.

This grammar requires the parser to support token priorities, which Irony does.
Removing the necessity for token priorities is possible by altering the tokens and production rules, but makes the grammar more complicated and the resulting tree harder to use, thus being detrimental to design goals 2 and 3.

Some simple tokens (e.g. '\%', '!') are directly defined in the production rules in Figure \ref{figure:productions} in between quotes for readability and compactness.

\subsubsection{\textbf{Dates}}

The appearance of date and time values in spreadsheets depends on the presentation settings of cells. Internally, date and time values are stored as positive floating point numbers with the integer portion representing the number of days since a Jan 0 1900 epoch\footnote{Note that 1900 is incorrectly considered a leap year, due to a bug in Lotus 1-2-3 (first released in 1983) which was deliberately copied into the first Excel release and has since then been preserved for backwards compatibility reasons.} and the fractional portion representing the portion of the day passed.

When extracting formulas from spreadsheets, only the floating point value is available.
The parser will thus never see the formatted notation of the date.
For this reason, the grammar only parses numeric dates and times and these are not distinguishable from other numbers.

\newpage

\subsection{Syntactical Analysis}

\begin{figure}
\small
\begin{multicols*}{2}
\begin{grammar}
	<Start> ::= <Constant>
	\alt '=' <Formula>
	\alt `\{=' <Formula> `\}'
	
	<Formula> ::= <Constant>
	\alt <Reference>
	\alt <FunctionCall>
	\alt `(' <Formula> `)'
	\alt <ConstantArray>
	\alt "RESERVED-NAME"
	
	<Constant> ::= "NUMBER" | "STRING" | "BOOL" | "ERROR"
	%         \alt "STRING"
	%         \alt "BOOL"
	%         \alt "ERROR"
	
	<FunctionCall> ::= 
	<UnOpPrefix> <Formula>
	\alt <Formula> `\%'
	\alt <Formula> <BinOp> <Formula>
	\alt "EXCEL-FUNCTION" <Arguments> `)'
	
	<UnOpPrefix> ::= `+' | `-'
	
	<BinOp> ::= `+' | `-' | `*' | `/' | `\textasciicircum'
	\alt `<' | `>' | `=' | `<=' | `>=' | `<>'
	
	%<Function> ::=  | "UDF"
	
	<Arguments> ::= $\epsilon$ \alt <Argument> \{ `,' <Argument> \}
	
	<Argument> ::= <Formula> | $\epsilon$
	
	<Reference> ::= <ReferenceItem>
	\alt <RefFunctionCall>
	\alt `(' <Reference> `)' 
	\alt <Prefix> <ReferenceItem>
	%\alt <Prefix> "UDF" <Arguments> `)'
	%    \alt <DynamicDataExchange>
	\alt "FILE" `!' "DDECALL"
	
	%<NamedRange> ::= "NAMED-RANGE"
	%            \alt "NAMED-RANGE-PREFIXED"
	
	<RefFunctionCall> ::= `(' <Union> `)'
	\alt <RefFunctionName> <Arguments> `)'
	\alt <Reference> `:' <Reference>
	\alt <Reference> `\ ' <Reference>
	
	<ReferenceItem> ::= "CELL"
	\alt <NamedRange>
	\alt <StructuredReference>
	\alt "VERTICAL-RANGE"
	\alt "HORIZONTAL-RANGE"
	\alt "UDF" <Arguments> `)'
	\alt "ERROR-REF"
	
	\columnbreak
	
	<NamedRange> ::= <Name>
		
	<Name> ::= "NAME" | "NAME-PREFIXED"
	
	<File> ::= FILE
	\alt FILENAME
	\alt FILEPATH FILENAME
	
	<Prefix> ::= "SHEET"
	\alt `'' "SHEET-QUOTED"
	\alt <File> "SHEET"
	\alt `'' <File> "SHEET-QUOTED"
	\alt "FILE" `!'
	\alt "MULTIPLE-SHEETS"
	\alt <File> "MULTIPLE-SHEETS"
	
	<RefFunctionName> ::= "REF-FUNCTION"
	\alt "REF-FUNCTION-COND"
	
	<Union> ::= <Reference> \{ `,' <Reference> \}
	
	%<DynamicDataExchange> ::= "FILE" `!' "DDECALL"
	
	<ConstantArray> ::= `\{' <ArrColumns> `\}'
	
	<ArrColumns> ::= <ArrRows> \{ `;' <ArrRows> \}
	
	<ArrRows> ::= <ArrConst> \{ `,' <ArrConst> \}
	
	<ArrConst> ::= <Constant>
	\alt <UnOpPrefix> "NUMBER"
	\alt "ERROR-REF"
	
	<StructuredReference> ::= <SRCol>
	\alt `[' <SRExpr> `]'
	\alt <Name> <SRCol>
	\alt <Name> `[' <SRExpr> `]'
	
	<SRExpr> ::= <SRCol>
	\alt <SRCol> `:' <SRCol>
	\alt <SRCol> `,' <SRCol>
	\alt <SRCol> `,' <SRCol> `:' <SRCol>
	\alt <SRCol> `,' <SRCol> `,' <SRCol>
	\alt <SRCol> `,' <SRCol> `,' <SRCol> `:' <SRCol>
	
	<SRCol> ::= FILENAME
	\alt `[' <Name> `]'  
	\alt `[' SR-COLUMN `]'  
	
	
	
\end{grammar}
\end{multicols*}
\caption{Production rules}
\label{figure:productions}
\end{figure}

The complete production rules of the grammar are listed in Extended BNF syntax in Figure \ref{figure:productions}.
Patterns inside \{ and \} can be repeated zero or more times.
The start symbol is $Start$. An example parse tree produced using this grammar is drawn in Figure \ref{figure:parsetrees}(b).

The \synt{Formula} rule covers all types of spreadsheet formula expressions: they can be constants (\texttt{=5}), references (\texttt{=A3}), function calls and operators (\texttt{=SUM(A1,A2)}), array constants (\texttt{=\{1,2;3,4\}} or reserved names (\texttt{=_xlnm.Print_Area}).
The \synt{Reference} rule covers a subset of expressions known as references expressions: internal or external cell and range references, functions and operators which can return references, named ranges, structured ranges and dynamic data exchanges.

%\begin{figure}
%	\caption{Syntax diagram of the \synt{Formula} production rule with most production rules expanded}
%	\label{figure:Formula}
%	%\centerfloat
%	\input{parsing/formula-diagram.tex}
%\end{figure}
%
%\begin{figure}
%	\caption{Syntax diagram of the \synt{Reference} production rule with most production rules expanded}
%	\label{figure:Reference}
%	\begin{grammar}
		<Reference> ::= \begin{syntdiag}[\scriptsize\sdlengths]
		\begin{stack} '$($' \begin{rep} <Reference> \\  '$,$' \end{rep} '$)$'\\ <Reference> \begin{stack} '$:$' \\ '\char32' \end{stack} <Reference> \\
		\begin{stack} \\ \begin{stack} \\ "FILE" \end{stack} \begin{stack} "SHEET" \\ "MULTIPLE_SHEETS" \end{stack} \\ "FILE" '$!$' \\ "QUOTED_FILE_SHEET" \end{stack}
		\begin{stack} \begin{stack} "CELL" \\ "VERTICAL-RANGE" \\ "HORIZONTAL-RANGE" \\ \begin{stack} "NAMED-RANGE" \\ "NAMED-RANGE-COMBINED" \end{stack} \\ "ERROR-REF" \end{stack} \\  \begin{stack} "REFERENCE-FUNCTION" \\ "UDF" \end{stack} \begin{rep} \begin{stack} \\ <Formula> \end{stack} \\  '$,$' \end{rep} '$)$'\end{stack}
		\\"FILE" '$!$' "DDECALL"
		\end{stack}
		\end{syntdiag}
	\end{grammar}
%\end{figure}

\subsection{Precedence and ambiguity}
\label{sec:ambiguity}

\begin{table}
\small
\begin{tabular}{ll}
	\toprule
	Precedence & Operator(s) \\
	(higher is greater) & \\
	\midrule
	1 & = \textless \  \textgreater \  \textless= \  \textgreater= \  \textless\textgreater          \\
	2 & \&  \\
	3 & + - (binary) \\
	4 & $\ast$ / \\
	5 & \textasciicircum \\
	6 & \% \\
	7 & + - (unary) \\
	8 & , \\
	9 & \texttt{\char32} \\
	10 & : \\
	\bottomrule
\end{tabular}
\caption{Operator precedence in formulas}
\label{table:operatorprec}
\end{table}

\footnotetext{This is contrary to most other languages, where the exponentiation operator is right-associative. \\
In Excel \f{2\textasciicircum 1\textasciicircum 2} will be $(2^1)^2 = 4$, while in most other languages it will be $2^{1^2} = 2$}

The production rules are ambiguous, which means they cannot be directly used in a parser generator based on the LALR algorithm like Irony.

To resolve ambiguity with operators, e.g. whether to parse \f{=1+2*3} as \f{=(1+2)*3} or \f{=1+(2*3)}, operator precedence and associativity rules are defined.
These can be found in Table \ref{table:operatorprec}.

However, even with precedence and associativity rules the grammar is still not fully un-ambigious.
This is due to trade-offs on parsing references, see Section \ref{tradeoff:references}, and parsing unions (see section \ref{subsec:desing:unions}).
Ambiguity exists between the following production rules:
\begin{enumerate}
	\item \begin{grammar}<Reference> ::= `(' <Reference> `)'\end{grammar}
	\item \begin{grammar}<Union> ::= `(' <Reference> \{ `,' <Reference> \} `)'\end{grammar}
	\item \begin{grammar}<Formula> ::= `(' <Formula> `)'\end{grammar}
\end{enumerate}

A formula like \texttt{=(A1)} can be interpreted as either a bracketed reference, a union of one reference, or a reference within a bracketed formula.

In a LALR parser, which Irony produces, this ambiguity manifests in a state where, on a \texttt{')'} token, shifting on rule 1 and reducing on either rule 2 or 3 are possibilities, causing a shift-reduce conflict.
This was solved by instructing the parser generator to shift on rule 1 in case of this conflict, because this always is a correct interpretation and thus results in correct ASTs.

\newpage

\noindent
\begin{figure}[h!]
	\hspace*{0.003\textwidth}
	\input{implementation/refactoring-process-print}
	\caption{Section \ref{sec:printing} details AST pretty-printing, converting an AST to a string formula.}
\end{figure}

\section{Printing formula AST}
\label{sec:printing}

Printing a formula AST is the reverse operation of parsing, and is quite straightforward.
It can be done by describing for each tree node type how it can be translated back into a string, most of the time this is the exact reverse of the parser production rule.
Nodes with children need to know the printed form of their children, this can be satisfied by starting at the root of the tree and calling the print function recursively for each child.
A slightly simplified and compacted version of the XLParser code responsible for printing can be found in Appendix \ref{lst:xlparserprint}.

\section{Trade-offs}

The grammar presented in this chapter contains some trade-offs, partly due to the Excel language itself, partly due to design decisions.
These are detailed in this section.

\subsection{References}
\label{tradeoff:references}

References play an important role in the spreadsheet paradigm and therefore in the formula language.
In particular reference expressions, expressions which evaluate to a reference, are a subset of expressions and several operators and functions only accept reference expressions.
For example the formula \f{=SUM(IF(\ldots):A1)} is valid, while \f{=SUM((1+1):A1)} is not, because \f{IF} can return a reference while \f{+} cannot and the \f{:} operator only operates on references.

This is not unique to reference expressions, for example the operator \f{+} only operates on numeric values, making the expression \f{="a"+1} invalid.
What does make reference expressions special is how Excel treats them.
A formula which uses a non-reference expression where a reference expression is required, like the previously mentioned \f{=SUM((1+1):A1)}, will result in a parse error which means excel will not accept this formula from the user.
Meanwhile \f{="a"+1} will only result in a runtime error, an error value, but will be still parsed and evaluated.

For XLParser we had three options: do not concern ourselves with invalidly type expressions, incorporate the reference expression rules into the grammar, or implement a type system similar to how this would be done in a full compiler and reject invalidly typed expressions.
The first option is by far the simplest, but would result in a lot of invalid formulas being accepted, the second option would result in a more complicated grammar and might not even be possible, while the third option would result in an additional layer on top of the parser generator.

Because references are of great interest when analyzing formulas and already had additional grammar rules, the second option seemed to be achievable and acceptable and this is the route XLParser took and successfully implemented.
An additional downside of this approach turned out to be some additional ambiguity, as explained in Section \ref{sec:ambiguity}.

\subsection{Unions}
\label{subsec:desing:unions}

The comma serves both as an union operator and a function argument separator.
This proves challenging to correctly implement in a LALR(1) grammar.

A straightforward implementation would use production rules similar to this:
\begin{grammar}
<Union> ::= <Reference> `,' <Reference>

<Arguments> ::= <Argument> \{ `,' <Arguments> \}
\end{grammar}

However, this will cause a reduce-reduce conflict because the parser will have a state wherein it can reduce to both a \synt{Union} or \synt{Argument} on a \texttt{,} token.
Unfortunately there is no correct choice: in a formula like \texttt{=SUM(A1,1)} the parser must reduce on the \synt{Argument} nonterminal, while in a formula like \texttt{=A1,A1} the parser must reduce to the \synt{Union} nonterminal.
With the above production rules a LALR(1) parser could not correctly parse the language.

The presented grammar only parsers unions in between parentheses, e.g. \texttt{=SMALL((A1,A2),1)}.
This is a trade-off between a lower compatibility (design goal 1) and an easier implementation (design goal 3).
This lower compatibility is deemed acceptable, because unions are only extremely rarely used.
In the evaluation as described in Chapter \ref{chapter:evaluation} unions were only encountered in 0.002\% of formulas.

Additionally formulas that this grammar does not parse often result in an error value after evaluation in Excel.
For example \texttt{=A1,A1} does parse in Excel, but produces the error \texttt{\#VALUE!} on evaluation.

Implementing the straightforward rules above, while desirable, is not possible without using a more powerful grammar class.

\chapter{Refactoring spreadsheets}
\label{chapter:implementingrefactorings}

\noindent
\begin{figure}[h!]
\hspace*{0.003\textwidth}
\input{implementation/refactoring-process-impl}
\caption{This chapter details the AST to AST transformations that implement the refactorings.}
\end{figure}

Refactoring a spreadsheet involves changing the worksheets, cells and formulas in a workbook in such a way that the desired change is performed.
Excel provides an API to change the worksheet and cells, and most other elements of a workbook.
When it is desired to refactor formulas this means the original formula string must be changed into a new formula string.
This is usually implemented by parsing the formula, performing the desired transformations on the AST and then printing the AST back to a string form \cite{fowler1999refactoring}.
The inner workings of the parser are described in Chapter \ref{chapter:parsing}.
This Chapter covers how the AST is transformed for each refactoring.

The refactorings were implemented in the BumbleBee Excel Add-In, and presented to the user through a context menu as seen in Figure X.
This context-menu automatically determines if a refactoring can be performed on the specific selected cell(s) and disables inapplicable refactorings.
\todo{Screenshot van het bumblebee refactoring context-menu}

It must be noted that all refactorings have a major deficiency: they cannot be undone. \fw
The reason for this is a technical limitation imposed by Excel: the Excel undo-redo stack is not available to Excel Add-Ins.
Instead as soon as a Excel Add-In changes the Excel spreadsheet file, in fact as soon as it interacts with the internal document model even if it does not change anything, the Excel undo-redo stack gets cleared.
While there are several non-trivial workarounds for this involving a manually creating an undo stack, the author deemed them outside of the scope of this thesis.
In an industrial-strength application this functionality would be essential.
As long as Excel keeps this limitation this will always be a severe limitation to any tool that automatically changes spreadsheet files for the user, thus Microsoft could modify Excel to facilitate these and similar Add-Ins. \fw

\section{Refactorings provided by excel}

Several useful refactorings one could think of are already provided by Excel, although Excel does not call them refactorings.

\section{\rf{Extract formula}}

The goal of the \rf{extract formula} refactoring is to move part of a formula expression, a sub-formula, to another cell, which has the following potential use cases:

\begin{enumerate}
\itemsep0em
\item Remove "magic numbers" or other constants and make them easy to adjust.
\item Make a formula easier to understand by splitting it into more smaller components.
\item Reduce duplication in a formula by extracting common sub-formulas into another cell.
\end{enumerate}

\subsection{User interface}

\begin{figure}
%\begin{minipage}[c][8cm][c]{0.5\textwidth}
%\centering
%\vspace*{\fill}
%\includegraphics[height=3cm]{implementation/extractformula/21}
%\subcaption{User selects cells to be refactored}
%\label{fig:extractformulaexample2a}
%
%\includegraphics[height=3cm]{implementation/extractformula/23}
%\addtocounter{subfigure}{1}
%\subcaption{Refactoring has been performed}
%\label{fig:extractformulaexample2c}
%\end{minipage}
%\begin{minipage}[c][8cm][t]{0.5\textwidth}
%\vspace*{\fill}
%\centering
%\includegraphics[height=7cm]{implementation/extractformula/22}
%\addtocounter{subfigure}{-2}
%\subcaption{User selects subformula to be extracted}
%\label{fig:extractformulaexample2b}
%\end{minipage}
\centering
\includegraphics{{implementation/extractformula/UI\string_Extractformula\string_arrows\string_pdf}}
\caption{An example application of \rf{Extract Formula}.}
\label{fig:extractformulaexample2}
\end{figure}

The refactoring requires the user to select cell(s) to be refactored, type in the subformula to be extracted and select where the extraction should occur to.
Figure \ref{fig:extractformulaexample2} shows the process as experienced by the user.
The user first selects the formulas to be extracted (Figure \ref{fig:extractformulaexample2}a) and clicks the Extract Formula entry in the refactoring context menu (not shown).
A side-panel pops out which allows the user to enter the sub-formula to be extracted and where it should be extracted to (Figure \ref{fig:extractformulaexample2}b) and presses the Extract Formula button.
In the example the \f{50\%} subformula was extracted to the left, and Figure \ref{fig:extractformulaexample2}c shows the situation after the user has named the new column.

The UI could be improved in two ways: the user could select the subformula to extract or the options for extraction could be provided.
The usability of both options is unclear, for example in large formulas the user might be overwhelmed by the options presented and both were not implemented.
The author leaves this a future work.

\subsection{Implementation}

This section describes the details of the refactoring implementation, which consists of 2 parts: an AST tranformation fo the formula, and a 
The first part operates solely on the formula and refactors it to the desired form, the second part handles actual placement of the formula in the cells and the moving if necessary.

\subsubsection{AST Transformation}
\label{subsec:astreplacementtransformation}

The AST transformation takes the original AST, the AST to replace and the replacement AST.
Then the original AST is traversed and every occurrence of the AST to replace is replaced by the replacement AST, yielding the new AST, this is illustrated in Figure \ref{fig:extractformulaASTtransformations}.
The C\# code for the AST replacement can be found in Listing \ref{lst:astreplace}.

This transformation is somewhat similar to the BumbleBee 1 formula transformation rules.
However, it is complimentary rather than identical as can be seen by comparing Figure \ref{fig:extractformulaASTtransformations} and Figure \ref{fig:bbv1transformationrule}.
We could have tried to written this transformation as a rule "\f{=I2 + [a] * I2}".
However, BumbleBee 1 would have searched for the "outer" formula, keeping the \f{[a]} $\gets$ \f{50\%} available for the replacement rule.
In contrast, this transformation searches for the \f{[a]} "inner" formula, and replaces it with something different.

\subsubsection{Spreadsheet refactoring}

The AST replacement is performed on the original formula, and the new formula is assigned to the cell that needs to be refactored.
If multiple cells are refactored at once, the AST replacement is performed on all of them.
Formulas with the same original R1C1 formula will have the same new R1C1 formula, so the AST replacement is only performed once per unique R1C1 formula.

If the target of the extraction is a single cell, that cell gets assigned the subformula that will be extracted, otherwise if the user wants to extract in a direction, new cells are created in the appropriate direction and all will get assigned the subformula that will be extracted.

The C\# code for the spreadsheet refactoring can be found in Listing \ref{lst:extractformula}.

\begin{figure}
	\centering
	{
\begin{subfigure}[t]{0.12\textwidth}
\centering
\begin{tikzpicture}[-latex ,auto ,node distance =1.3 cm and 0.5cm ,on grid , semithick,
,
state/.style ={ circle ,top color =white ,
draw, minimum width =0.75 cm}]
\node[state] (Percent) {\tiny \%};
\node[state] (Constant50) [below =of Percent] {\tiny 50};

\path (Percent) edge node {} (Constant50);
\end{tikzpicture}
\caption*{${\scriptscriptstyle AST_{search}}$}
\end{subfigure}
}
{
\begin{subfigure}[t]{0.24\textwidth}
\centering
\begin{tikzpicture}[-latex ,auto ,node distance =1.3 cm and 0.8cm ,on grid , semithick,
,
state/.style ={ circle ,top color =white ,
draw, minimum width =0.75 cm}
,
redstate/.style ={ circle , fill=red,
draw, minimum width =0.75 cm}]
\node[state] (OpPlus) {$+$};
\node[state] (OpMult) [below right =of OpPlus] {$\ast$};

\node[state] (RefI2) [below left =of OpPlus] {\tiny Ref};
\node[state] (ConstI2) [below =of RefI2] {\tiny \f{I2}};

\path (RefI2) edge node {} (ConstI2);

\node[state] (RefI22) [below right = 1.3cm and 0.52cm of OpMult] {\tiny Ref};
\node[state] (ConstI22) [below =of RefI22] {\tiny \f{I2}};

\path (RefI22) edge node {} (ConstI22);

\node[redstate] (Percent) [below left = 1.3cm and 0.42cm of OpMult] {\tiny \%};
\node[redstate] (Constant50) [below =of Percent] {\tiny 50};

\path (Percent) edge node {} (Constant50);

\path (OpPlus) edge node {} (OpMult);
\path (OpPlus) edge node {} (RefI2);
\path (OpMult) edge node {} (RefI22);
\path (OpMult) edge node {} (Percent);
\end{tikzpicture}
\caption*{${\scriptscriptstyle AST_{or}}$}
\end{subfigure}
}
{
\begin{subfigure}[t]{0.12\textwidth}
\centering
\begin{tikzpicture}[-latex ,auto ,node distance =1.3 cm and 0.5cm ,on grid , semithick,
,
state/.style ={ circle ,top color =white ,
draw, minimum width =0.75 cm}]
\node[state] (RefJ2) {\tiny Ref};
\node[state] (ConstJ2) [below =of RefJ2] {\tiny \f{J2}};

\path (RefJ2) edge node {} (ConstJ2);
\end{tikzpicture}
\caption*{${\scriptscriptstyle AST_{repl}}$}
\end{subfigure}
}
{
\begin{subfigure}[b]{0.1\textwidth}
\centering
\vspace*{4em}
$\xrightarrow[Returns]{}$
\vspace*{2em}
\end{subfigure}
}
{
\begin{subfigure}[t]{0.24\textwidth}
\centering
\begin{tikzpicture}[-latex ,auto ,node distance =1.3 cm and 0.8cm ,on grid , semithick,
,
state/.style ={ circle ,top color =white ,
draw, minimum width =0.75 cm}
,
greenstate/.style ={ circle , fill=green,
draw, minimum width =0.75 cm}]
\node[state] (OpPlus) {$+$};
\node[state] (OpMult) [below right =of OpPlus] {$\ast$};

\node[state] (RefI2) [below left =of OpPlus] {\tiny Ref};
\node[state] (ConstI2) [below =of RefI2] {\tiny \f{I2}};

\path (RefI2) edge node {} (ConstI2);

\node[state] (RefI22) [below right = 1.3cm and 0.52cm of OpMult] {\tiny Ref};
\node[state] (ConstI22) [below =of RefI22] {\tiny \f{I2}};

\path (RefI22) edge node {} (ConstI22);

\node[greenstate] (RefJ2) [below left = 1.3cm and 0.42cm of OpMult] {\tiny Ref};
\node[greenstate] (ConstJ2) [below =of RefJ2] {\tiny \f{J2}};

\path (RefJ2) edge node {} (ConstJ2);

\path (OpPlus) edge node {} (OpMult);
\path (OpPlus) edge node {} (RefI2);
\path (OpMult) edge node {} (RefI22);
\path (OpMult) edge node {} (Percent);
\end{tikzpicture}
\caption*{${\scriptscriptstyle AST_{new}}$}
\end{subfigure}
}
	\caption{AST transformation to implement extract formula refactoring}
	\label{fig:extractformulaASTtransformations}
\end{figure}

\begin{figure}
	\centering
	{
\begin{subfigure}[t]{0.12\textwidth}
\centering
\tiny\f{I2 + [a] * I2}
\caption*{{\tiny Origin rule}}
\end{subfigure}
}
{
\begin{subfigure}[t]{0.24\textwidth}
\centering
\begin{tikzpicture}[-latex ,auto ,node distance =1.3 cm and 0.8cm ,on grid , semithick,
,
state/.style ={ circle ,top color =white ,
draw, minimum width =0.75 cm}
,
redstate/.style ={ circle , fill=red,
draw, minimum width =0.75 cm}]
\node[redstate] (OpPlus) {$+$};
\node[redstate] (OpMult) [below right =of OpPlus] {$\ast$};

\node[redstate] (RefI2) [below left =of OpPlus] {\tiny Ref};
\node[redstate] (ConstI2) [below =of RefI2] {\tiny \f{I2}};

\path (RefI2) edge node {} (ConstI2);

\node[redstate] (RefI22) [below right = 1.3cm and 0.52cm of OpMult] {\tiny Ref};
\node[redstate] (ConstI22) [below =of RefI22] {\tiny \f{I2}};

\path (RefI22) edge node {} (ConstI22);

\node[state] (Percent) [below left = 1.3cm and 0.42cm of OpMult] {\tiny \%};
\node[state] (Constant50) [below =of Percent] {\tiny 50};

\path (Percent) edge node {} (Constant50);

\path (OpPlus) edge node {} (OpMult);
\path (OpPlus) edge node {} (RefI2);
\path (OpMult) edge node {} (RefI22);
\path (OpMult) edge node {} (Percent);
\end{tikzpicture}
\caption*{{\tiny Matching AST}}
\end{subfigure}
}
{
\begin{subfigure}[t]{0.14\textwidth}
\centering
\tiny\f{\ldots\ [a] \ldots}
\caption*{{\tiny Target rule}}
\end{subfigure}
}
{
\begin{subfigure}[b]{0.1\textwidth}
\centering
\vspace*{4em}
$\xrightarrow[Returns]{}$
\vspace*{2em}
\end{subfigure}
}
{
\begin{subfigure}[t]{0.24\textwidth}
	\centering
	\begin{tikzpicture}[-latex ,auto ,node distance =1.3 cm and 1.0cm ,on grid , semithick,
	,
	state/.style ={ circle ,top color =white ,
		draw, minimum width =0.75 cm}
	,
	redstate/.style ={ circle , fill=red,
		draw, minimum width =0.75 cm}
	,
	greenstate/.style ={ circle , fill=green,
		draw, minimum width =0.75 cm}]
	\node[greenstate] (OpPlus) {\tiny\ldots};
	\node[greenstate] (OpMult) [below right =of OpPlus] {\tiny\ldots};
	
	\node[greenstate] (RefI2) [below left =of OpPlus] {\tiny\ldots};
	
	%\node[state] (Percent) [below left = 1.3cm and 0.42cm of OpMult] {\tiny \%};
	\node[state] (Percent) [below = of OpPlus] {\tiny \%};
	\node[state] (Constant50) [below =of Percent] {\tiny 50};
	
	\path (Percent) edge node {} (Constant50);
	
	\path (OpPlus) edge node {} (OpMult);
	\path (OpPlus) edge node {} (RefI2);
	\path (OpPlus) edge node {} (Percent);
	\end{tikzpicture}
	\caption*{{\tiny New AST}}
\end{subfigure}
}
	\caption{Bumblebee v1 transformation rule}
	\label{fig:bbv1transformationrule}
\end{figure}

\newpage

\subsection{Detection of applicability}

Extract formula is always applicable to a formula cell, as even a very simple formula like \f{=A1} still has a component that can be extracted.
In this case if \f{=A1} would be extracted to \f{B1} the original cell would become \f{=B1}.
This could be repeated endlessly, similarly to how one could always extract a method that only consists of a call to another method.
Whether it is a good thing to perform this refactoring is dubious, but BumbleBee relies on the user to make this assessment.

\subsection{Improvements over RefBook \rf{Extract row or column} and \rf{Extract Literal}}
\label{subsubsec:improvementsextractformula}

Two specialized versions of this refactorings where previously described by Bamade and Dig \cite{badame2012refactoring} and implemented in their RefBook tool.
Refbooks \rf{Extract Row or column} and \rf{Extract Literal} refactorings can both be performed by \rf{Extract formula}.

The author has chosen to not keep the \rf{extract row or column} refactoring name because it does not fully describe the refactoring, a full row or column does not necessarily have to be extracted, and to keep the name in line with refactoring names in other domains.

The RefBook \rf{extract literal} refactoring can put a constant value into a cell and replace the occurrences of it with references to that cell, this can also be achieved with the BumbleBee \rf{extract formula} refactoring.
In addition this is possible for any constant expression, an expression without references, instead of only for constants.

The BumbleBee \rf{Extract Formula} refactoring has several advantages over Badame's implementation of \rf{Extract row or column}.
Firstly RefBook does not handle operator precedence.
This can be very problematic for this refactoring and in the authors opinion should prevent is from being implemented, because one of the prime properties of a refactoring should be that it does not change the program results.
Note that the RefBook authors were aware of this deficiency, and left this as future work.
This future work has been performed by the thesis author.

Secondly RefBook can only handle a single row or column, which has to have exactly the same R1C1 formula. It can only extract the subformula to a column right of or row up of the original range.
BumbleBee can handle arbitrarily shaped ranges, with the only requirement that the subformula to be extracted occurs in all selected formulas.
Furthermore in addition to extracting to a cell neighboring the original formula cell (up, down, left or right) it can also extract the subformula to a single shared cell location.
This is very useful to remove duplication and makes the refactoring more universal by merging the \rf{extract literal} refactoring into it.

\section{\rf{Inline Formula}}

The goal of the \rf{Inline Formula} refactoring is to replace all references to a cell with its contents and delete the original cell, and is therefore the inverse of the \rf{Extract Formula} refactoring.
The main potential use case for this formula is when the contents of a cell are clearer or just as clear as a cell reference.

While single cell references (e.g. \f{=SUM(A1,A2,A3)}) can always be inlined, if a cell is referenced as part of a range (e.g. \f{=SUM(A1:A3)}) it can not always be inlined.
If in the previous example \f{A1} would contain \f{20}, the first formula would turn out fine: \f{=SUM(20,A2,A3)}, while the formula \f{=SUM(20:A3)} is invalid.
It might be possible to handle inlining into ranges using array formulas, but the extra complexity this would introduce in the formulas never outweighs the benefit of inlining in the authors opinion.
Some formulas might be able to be rewritten, e.g. \f{=SUM(A1:A3)} could become \f{=SUM(20,A2:A3)}, but this does not work in every case (e.g. \f{A1} cannot be inlined into \f{=A1:A5 3:3}) and thus such behavior has a higher chance to introduce errors and confuse users.
For these reasons the implementation does not perform the refactoring if the cell is referenced as part of a range.

For similar reasons, this refactoring cannot be performed on cells which are part of a named range consisting of more than one cell.

\subsection{User interface}

The refactoring is a one-click refactoring that is activated from the cell context menu, no additional user input in normally needed.

If one of the selected cells is referenced as part of a range, the user gets the choice to either abort the refactoring or continue replacing all references where the cell isn't part of a range.

\begin{figure}
	\centering
	{
\begin{subfigure}[t]{0.12\textwidth}
\centering
\begin{tikzpicture}[-latex ,auto ,node distance =1.3 cm and 0.5cm ,on grid , semithick,
,
state/.style ={ circle ,top color =white ,
	draw, minimum width =0.75 cm}]
\node[state] (RefJ2) {\tiny Ref};
\node[state] (ConstJ2) [below =of RefJ2] {\tiny \f{J2}};

\path (RefJ2) edge node {} (ConstJ2);
\end{tikzpicture}
\caption*{{\tiny AST to replace}}
\end{subfigure}
}
{
\begin{subfigure}[t]{0.24\textwidth}
\centering
\begin{tikzpicture}[-latex ,auto ,node distance =1.3 cm and 0.8cm ,on grid , semithick,
,
state/.style ={ circle ,top color =white ,
	draw, minimum width =0.75 cm}
,
redstate/.style ={ circle , fill=red,
	draw, minimum width =0.75 cm}
,
greenstate/.style ={ circle , fill=green,
	draw, minimum width =0.75 cm}]
\node[state] (OpPlus) {$+$};
\node[state] (OpMult) [below right =of OpPlus] {$\ast$};

\node[state] (RefI2) [below left =of OpPlus] {\tiny Ref};
\node[state] (ConstI2) [below =of RefI2] {\tiny \f{I2}};

\path (RefI2) edge node {} (ConstI2);

\node[state] (RefI22) [below right = 1.3cm and 0.52cm of OpMult] {\tiny Ref};
\node[state] (ConstI22) [below =of RefI22] {\tiny \f{I2}};

\path (RefI22) edge node {} (ConstI22);

\node[redstate] (RefJ2) [below left = 1.3cm and 0.42cm of OpMult] {\tiny Ref};
\node[redstate] (ConstJ2) [below =of RefJ2] {\tiny \f{J2}};

\path (RefJ2) edge node {} (ConstJ2);

\path (OpPlus) edge node {} (OpMult);
\path (OpPlus) edge node {} (RefI2);
\path (OpMult) edge node {} (RefI22);
\path (OpMult) edge node {} (RefJ2);
\end{tikzpicture}
\caption*{{\tiny Original AST}}
\end{subfigure}
}
{
\begin{subfigure}[t]{0.14\textwidth}
\centering
\begin{tikzpicture}[-latex ,auto ,node distance =1.3 cm and 0.5cm ,on grid , semithick,
,
state/.style ={ circle ,top color =white ,
	draw, minimum width =0.75 cm}]
\node[state] (Percent) {\tiny \%};
\node[state] (Constant50) [below =of Percent] {\tiny 50};

\path (Percent) edge node {} (Constant50);
\end{tikzpicture}
\caption*{{\tiny Replacement AST}}
\end{subfigure}
}
{
\begin{subfigure}[b]{0.1\textwidth}
\centering
\vspace*{4em}
$\xrightarrow[Returns]{}$
\vspace*{2em}
\end{subfigure}
}
{
\begin{subfigure}[t]{0.24\textwidth}
\centering
\begin{tikzpicture}[-latex ,auto ,node distance =1.3 cm and 0.8cm ,on grid , semithick,
,
state/.style ={ circle ,top color =white ,
	draw, minimum width =0.75 cm}
,
redstate/.style ={ circle , fill=red,
	draw, minimum width =0.75 cm}
,
greenstate/.style ={ circle , fill=green,
	draw, minimum width =0.75 cm}]
\node[state] (OpPlus) {$+$};
\node[state] (OpMult) [below right =of OpPlus] {$\ast$};

\node[state] (RefI2) [below left =of OpPlus] {\tiny Ref};
\node[state] (ConstI2) [below =of RefI2] {\tiny \f{I2}};

\path (RefI2) edge node {} (ConstI2);

\node[state] (RefI22) [below right = 1.3cm and 0.52cm of OpMult] {\tiny Ref};
\node[state] (ConstI22) [below =of RefI22] {\tiny \f{I2}};

\path (RefI22) edge node {} (ConstI22);

\node[greenstate] (Percent) [below left = 1.3cm and 0.42cm of OpMult] {\tiny \%};
\node[greenstate] (Constant50) [below =of Percent] {\tiny 50};

\path (Percent) edge node {} (Constant50);

\path (OpPlus) edge node {} (OpMult);
\path (OpPlus) edge node {} (RefI2);
\path (OpMult) edge node {} (RefI22);
\path (OpMult) edge node {} (Percent);
\end{tikzpicture}
\caption*{{\tiny New AST}}
\end{subfigure}
}
	\caption{\rf{Inline Formula} reverse AST transformation of the \rf{Extract Formula} AST transformation of Figure \ref{fig:extractformulaASTtransformations}.}
	\label{fig:inlineformulaAST}
\end{figure}

\subsection{Implementation}

If a formula cell $D$ contains a reference to cell $P$, $D$ is called a dependent of $P$, and $P$ is called a precedent of $D$.
This refactoring works by first collecting all dependents of the to be inlined cell.
This information is provided by Excel, although it could be manually constructed by parsing all formulas and building a dependency graph.

In every dependent a reference to the to be inlined cell is replaced by its contents, using the same AST transformation used by \rf{Extract Formula} as described in Section \ref{subsec:astreplacementtransformation}, with a reference to the cell as the AST to replace and the cell contents as the replacement AST.
This is also illustrated in Figure \ref{fig:inlineformulaAST}, which shows the \rf{Inline Formula} inverse action of the transformation performed in Figure \ref{fig:extractformulaASTtransformations}.
If the refactoring is successful, the original cell is deleted.

The refactoring can be performed on multiple cells at the same time.
To achieve this the above process is simply repeated.

The C\# code for this spreadsheet refactoring can be found in Listing \ref{lst:inlineformula}.

\subsection{Detection of applicability}

As described in the introduction of this section, \rf{Inline Formula} is applicable to all cells which have dependents but are not referenced as part of ranges.
For speed purposed however, the BumbleBee refactoring context menu only check whether the cell has any dependents.
Doing the full check would introduce significant delay every time the user would right click.

\section{\rf{Group References}}

In the spreadsheet formula language, some built-in functions have the ability to accept a variable number of arguments, most prominently \f{SUM}, the most commonly used function \cite{hermans2014enron}.
These functions also accept ranges, and thus the formulas \f{=SUM(A1,A2,A3,A4)} and \f{=SUM(A1:A4)} are equivalent.
The \rf{Group References} was defined but not implemented by Hermans et. al \cite{hermans2014detecting} and assumes spreadsheet users prefer range usage like \f{=SUM(A1:A4)} over individual cell references like \f{SUM(A1,A2,A3,A4)}, and merges multiple adjacent cell references into a single range reference.

\subsection{User Interface}

\todo{Hmm, deze stuktjes worden erg repetaties/oninformatief. Misschien bij introductie zeggen dat dit de standaard manier is en alleen waar nodig een subsectie over maken.}

The refactoring is a one-click refactoring that is activated from the cell context menu, no additional user input is needed.

\subsection{Grouping algorithm}

In order to find the best grouping we have to solve the following problem: given a sheet with a certain set of cells selected, what are the ranges that select exactly those cells and do so with a minimum amount of ranges?

It turns out that this is a NP-hard problem, because it has a straightforward translation to a NP-hard version of the Polygon Covering problem, specifically covering a rectilinear polygon (the selected cells) with axis-parallel rectangles (ranges), allowing for holes.
This allows us to use an approximation algorithm or heuristic.
A $O(\sqrt{\log{n}})$ approximation algorithm for this specific problem has been found by Kumar and Ramesh \cite{kumar2003covering}, but implementing this would take a non-trivial amount of effort.
This makes using a simple heuristic attractive since it will probably be good enough for this purpose since the majority of refactorings will consist of simple cases.

However, rather than implementing a heuristic ourselves, this is currently delegated to Excel which contains this functionality.
The algorithm Excel uses for this is unknown.

\subsection{Implementation}

The implementation traverses the formula AST, and for every function with a variable number of arguments it encounters groups it references by excluding all non-references (e.g. constants) and sending these to Excel to be grouped.
The function arguments then are replaced by grouped references and the AST is printed back to the formula cell.
References are processed separately depending on their absolute markers, e.g. \f{A1},\f{\$A1}, \f{A\$1} and \f{\$A\$1}, because grouping references with different markers cannot be done without changing the meaning of the formula.

If multiple cells are selected, the refactoring is repeated for every one.

The C\# code for this refactoring can be found in Listing \ref{lst:groupreferences}

\subsection{Detection of applicability}

This refactoring will be available to the user if the formula contains 2 or more references.

\section{\rf{Replace Awkward Formula}}

In the spreadsheet formula language, some binary operators have an equivalent aggregate function which accepts any number of arguments: \f{+} correspons to \f{SUM}, \f{*} to \f{PRODUCT} and \f{\&} to \f{CONCATENATE}.

Thus a formula like \f{=A1+A2+A3+A4} can be rewritten to \f{=SUM(A1,A2,A3,A4)}, which is what the \rf{Replace Awkward Formula} refactoring, defined by Badame and Dig \cite{badame2012refactoring}, does.
The refactoring is especially useful when combined with the \rf{group references} refactoring, which rewrites it to \f{=SUM(A1:A4)}.


\subsection{User Interface}

In the user interface, this refactoring is combined with the \rf{Introduce Conditional Aggregate} as a \emph{Introduce (Conditional) Aggregate} function.
This refactoring is a one-click refactoring that is activated from the cell context menu, no additional user input is needed.
In the user interface it will always transparently be followed by the \rf{Group References} refactoring.

\subsection{Implementation}

\subsection{Detecting applicability}

This refactoring will only be offered to the user on a formula cell consisting of one of the applicable operators (\f{+},\f{*} or \f{\&})

\section{\rf{Introduce Conditional Aggregate}}

The formula language also contains the built-in conditional aggregate functions \f{SUMIF}, \f{AVERAGEIF} and \f{COUNTIF}.
These functions take two mandatory arguments: a range and a condition.
The function then performs its operation on every item that meets the condition.
In \f{SUMIF} and \f{AVERAGIF} this is often combined with the optional third argument, another range.
If this argument is supplied the first range is tested 

\subsection{User Interface}

\section{\rf{Fixate References}}

\todo{Unsure of ik deze nog wil implementeren, maar lijkt zeer laaghangend fruit.}

The Fixate References is similar to the \rf{Make Cell Constant} refactoring described by Badame \cite{badame2012refactoring}.
Maar beter want: user kan zelf selecteren welke cellen wel/niet absolute.

"Fixate references" beschrijft 100x beter wat de refactoring doet dan "Make cell constant". "Make cell constant" zou eerder impliceren dat je het resultaat van de cell berekening neemt en dat opslaat i.p.v. de formula. Wat trouwens ook weer een potentiele refactoring is, maar misschien een beetje een anti-pattern in excel.

\section{\rf{Introduce name}}

\todo{Kandidaat voor implementatie, meer laaghanged fruit.}

Cell (of range?) een naam geven, overal waar die naar gereferenced wordt vervangen door cell naam.
Of gelijk bij extract formula intregrereren!

\chapter{Evaluation}
\label{chapter:evaluation}

\chapter{Conclusion}

\chapter{Future Work}

\chapter{Discussion}
\label{chapter:discussion}

\todo{Bibliography uniform}

\bibliographystyle{unsrt}
\bibliography{thesis}

\appendix

\chapter{Code listings}
\chapter{Code listings}

\lstset{style=sharpc}
\begin{lstlisting}[float,caption={XLParser Print method (simplified)}, label={lst:xlparserprint}]
public static string Print(this Node node) {
	// Print token values
	if(node is Terminal) return node.Token.Text;
	
	// Select is C#'s map function
	var ch = node.ChildNodes.Select(Print).ToList();
	
	switch(node.Type()) {
		case "ArrayFormula":
			return "{=" + ch[0] + "}";
		case "FunctionCall":
			if(node.IsBinaryOperation()) {
				return ch[0] + " " + ch[1] + " " + ch[2];
			}
			if(node.isNamedFunction()) {
				return String.Join("", ch) + ")";
			}
			// some more conditions
			break;
		// More cases for every node type
	}
}
\end{lstlisting}

\lstset{style=sharpc}
\begin{lstlisting}[float,caption={Formula AST replacement (simplified)}, label={lst:astreplace}]

/* Context contains the workbook and worksheet of a node */

public static Node Replace(Node subject, Node search, Node replace, Context csub, Context csearch, Context crepl) {
	// Check if the subject matches search
	if(Equals(subject, search, csub, csearch)) {
		// We can return the replacement.
		// Moveto handles changing reference prefixes if necesarry 
		return MoveTo(replace, crepl, csub);
	}
	
	// No match, if we are at a leaf node, simply return the leaf node
	if (subject.ChildNodes.Count == 0) return subject;
	
	// Otherwise continue the replacement on the child nodes
	return new Node() {
		Type = subject.Type(),
		// Select is C#'s map
		ChildNodes = subject.ChildNodes.Select(child => Replace(child, search, replace, csub, csearch, crepl))
	};
}

public static bool Equals(Node p1, Node p2, Context c1, Context c2) {
	
	// RemoveNonEqualityAffectingNodes removes things like brackets,
	// which do not affect the equality of nodes
	p1 = RemoveNonEqualityAffectingNodes(p1);
	p2 = RemoveNonEqualityAffectingNodes(p2);
	
	// Qualify adds workbook and worksheet prefix to all references, so that
	// equality isn't affected by whether or not these are supplied in the original formula
	p1 = c1.Qualify(p1);
	p2 = c2.Qualify(p2);
	
	return p1.Type() == p2.Type()
		// Compare the token values if these are tokens
	    && (p1 is Terminal && p1.Token.ValueString == p2.Token.ValueString)
	    // Compare child count
	    && p1.ChildNodes.Count == p2.ChildNodes.Count
	    // Check if all children are equal
	    && p1.ChildNodes.Zip(p2.ChildNodes).All((ch1, ch2) => Equals(ch1, ch2, c1, c2));
}
\end{lstlisting}

\lstset{style=sharpc}
\begin{lstlisting}[float,caption={Extract formula refactoring (simplified)}, label={lst:extractformula}]
public void ExtractFormula(Range applyto, Location moveto, Node subformula) {
	
	/** Check if all applyto cells contain subformula **/
	
	/** Check if target cell is empty **/
	
	/** Check if subformula contains any non-absolute references **/
	
	// Set the target cell to the subformula
	moveto.Formula = subformula.Print();
	// and get its parsed address reference
	var replacementAST = moveto.Address().Parse();
	
	// Apply the refactoring once per unique R1C1 formula
	foreach (var uniqueR1C1 in applyto.Cells.GroupBy()(c => c.FormulaR1C1)) {
		var prototype = uniqueR1C1.First();
		var AST_or = prototype.Parse();
		
		prototype.Formula = Replace(AST_or, subformula, replacementAST, /*...*/).Print();
		
		foreach(var cell in uniqueR1C1) {
			cell.FormulaR1C1 = prototype.FormulaR1C1;
		}
	}
}

public void ExtractFormula(Range applyto, Direction dir, Node subformula) {
	
	/** Check if all applyto cells contain subformula **/
	
	/** Insert new cells in the appropriate direction **/
	
	/** Set all new cells to contain the subformula formula **/
	
	// Apply the refactoring once per unique R1C1 formula
	foreach (var uniqueR1C1 in applyto.Cells.GroupBy()(c => c.FormulaR1C1)) {
		Cell prototype = uniqueR1C1.First();
		Node AST_or = prototype.Parse();
		Node replacementAST = prototype.Offset[dir].Address().Parse();
		
		prototype.Formula = Replace(AST_or, subformula, replacementAST, /*...*/).Print();

		foreach(var cell in uniqueR1C1) {
			cell.FormulaR1C1 = prototype.FormulaR1C1;
		}
	}
}
\end{lstlisting}

\lstset{style=sharpc}
\begin{lstlisting}[float,caption={Inline Formula Refactoring (simplified)}, label={lst:inlineformula}]
public void InlineFormula(Cell toInline) {
	// Dependendants are gotten from Excel
	var dependents = toInline.dependents;
	if(dependents.Count == 0) {
		// Abort, no dependants
	}
	
	Node AstToRepl = Parse(toInline.Address);
	Node AstReplacement = Parse(toInline.Formula);
	
	// Check if this cell is part of any named ranges with more than one cell
	if(toInline.Names.Any(name => name.Cells.Count > 1)) {
		// Abort
	}
	
	// AST representation of the names
	var names = toInline.Names.Select(Parse);
	
	foreach(var dependent in dependents) {
		Node AstOriginal = Parse(dependent.Formula);
		
		// Abort if to be inlined cell is references as part of a range
		if(CellContainedInRanges(toInlineAddress, AstOriginal)) {
			// Abort
		}
		
		// Replace references to the cell with the value
		var AstNew = Replace(AstOriginal, AstToRepl, AstReplacement, /*...*/);
		foreach(var name in names) {
			AstNew = Replace(AstNew, name, AstReplacement, /*...*/);
		}
		
		dependent.Formula = AstNew.Print();
	}
	
	toInline.Delete();
	
}
\end{lstlisting}

\lstset{style=sharpc}
\begin{lstlisting}[float,caption={Introduce Cell Name (simplified)}, label={lst:introducecellname}]
public void IntroduceName(Range toName, string name) {
		// Check if name already exists
		if(toName.Workbook.Names.Contains(name)) {
			// Abort
		}
		
		toName.Name = name;
		
		var dependents = toInline.dependents;
		if(dependents.Count == 0) {
			// Abort, no dependants
		}
		
		var AstToRepl = Parse(toName.Address);
		var AstReplacement = Parse(name);
		
		foreach(var dependent in dependents) {
			var AstOriginal = Parse(dependent.Formula);
			
			var AstNew = Replace(AstOriginal, AstToRepl, AstReplacement, /*...*/);
			
			dependent.Formula = AstNew.Print();
		}
		
}
\end{lstlisting}

\lstset{style=sharpc}
\begin{lstlisting}[float,caption={Group References Refactoring (simplified)}, label={lst:groupreferences}]
public Node GroupReferences(Node formula) {
	// Get all nodes representing varargs functions
	var targets = formula.AllNodes().Where(IsVarargsFunction);
	
	foreach(Node function in targets) {
		// split arguments that can be grouped from those than can't
		var toGroupArguments = function.arguments
			.Where(node => node.IsCellOrRange());
		var ignoredArguments = function.arguments
			.Where(node => !node.isCellOrRange());
		
		var grouped = new List<Node>();
		
		// Several characteristics define whether references can be grouped
		foreach(var sheetGroup in SplitByWorksheet(toGroupArguments)) {
			foreach(var toGroup in SplitByAbsoluteMarkers(sheetGroup)) {
				grouped.Add(GroupUsingExcel(toGroup));
			}
		}
		
		function.arguments = ignoredArguments.Concat(grouped);
	}
	
	return formula;
}

\end{lstlisting}

\lstset{style=sharpc}
\begin{lstlisting}[float,caption={Introduce Aggregate (simplified)}, label={lst:introduceaggregate}]
public Node IntroduceAggregate(Node formula) {

	// Precondition: formula is a function
	if(!formula.IsFunction()) {
		// Abort
	}
	
	// Precondition: formula is an operator that has an aggregate equivalent
	var op = formula.GetFunctionName();
	if(!AggregateEquivalents.ContainsKey(op)) {
		// Abort
	}
	
	var arguments = new List<Node>();
	var current = formula;
	
	// Gather arguments while the right subtree remains the same operator
	while(current.GetFunctionName() == op){
		arguments.Add(current.LeftArgument);
		current = current.RightArgument;
	}
	arguments.Add(current.RightArgument);
	
	return new Function(AggregateEquivalents[op], arguments);
}

private static Dictionary<string,string> AggregateEquivalents =
	new Dictionary<string,string>() {
		{ "+", "SUM" },
		{ "*", "PRODUCT" },
		{ "&", "CONCATENATE" }
	};

\end{lstlisting}

\lstset{style=sharpc}
\begin{lstlisting}[float,caption={Introduce Conditional Aggregate (simplified)}, label={lst:introduceconditionalaggregate}]
public void IntroduceConditionalAggregate(Cell subject) {

	Node function = Parse(subject.Formula);
	
	// Check if we can perform the refactoring
	if(!IsTargetFunction(function)) {
		if(function.IsFunction() && function.FunctionName == "+") {
			// Rewrite + to SUM
			function = IntroduceAggregate(Ast);
		} else {
			// Abort
		}
	}
	
	// Check if all arguments are references to a single cell
	var arguments = function.arguments;
	if(argument.Any(arg => !IsSingleCellReference(arg))) {
		// Abort
	}
	
	// Logic for all argument in a single row is ommited for brevity,
	// it is identical but transposed.
	
	// Check if all arguments are in a single column
	var summedColumn = arguments.First().Select(cell => cell.Column);
	if(arguments.Select(arg => arg.Column).Any(col => col != summedColumn)) {
		// Abort
	}
	
	var summedRows = argument.Select(cell => cell.Row).ToList();
	
	// Get all non-empty columns in the worksheet
	var columns = GetNonEmptyColumns(subject.Worksheet);
	
	// Find a functional determiner
	// See the listing on the next page for the code of FindDeterminerColumn
	var determiner = FindDeterminerColumn(columns, summedRows);

	if(determiner == null) {
		// Abort, there are no candidate columns
	}
	
	// Create the SUMIF(Column, Value, SummedColumn)
	var AstNew = new Function(function.FunctionName + "IF", new List<Node>() {
			new ColumnRange(determiner.Item1),
			new AstString(determiner.Item2),
			new ColumnRange(summedColumn)
		}
	);
	
	subject.Formula = AstNew.Print();
}
\end{lstlisting}

\begin{lstlisting}[float,caption={FindDeterminerColumn method}, label={lst:finddeterminercolumn}]
private static Tuple<string,string> FindDeterminerColumn(IEnumerable<Column> columns, IEnumerable<Row> summedRows) {
	// Return the first column that is a determiner of the summed column
	return columns.Select(column => {
		// Check if there is a value that is the same in all
		// the corresponding rows of the column
		string candidateValue = column[summedRows.First()].Value;
		
		if(summedRows.Any(row => column[row.ID].Value != candidateValue)) {
			return null;
		}
		
		// Value is the same in all corresponding rows
		// Now check if it is different in all other rows
		if(column.rows.Except(summedRows)
			.Any(cell => cell.Value == candidateValue)) {
			return null;
		}
		
		// We found a candidate column
		return Tuple.Create(column.ID, candidateValue);
	})
	.Where(found => found != null)
	.FirstOrDefault();
}


\end{lstlisting}


%// Find a functional determiner column if there is one
%foreach(var column in columns) {
%	string candidateValue = null;
%	// Check if there is a value that is the same in all rows
%	foreach(var row in summedRows) {
%		if(candidateValue == null) {
%			candidateValue = column[row]; 
%		} elseif (column[row] != candidateValue) {
%		candidateValue = null;
%		break;
%	}
%}
%if(candidateValue != null) {
%	// Check if that value does not occur in another row
%	if(!column.rows.Except(summedRows).Any(val => val == candidateValue)) {
%		// We found a column that can be used
%		FDcolumn = column.ID;
%		FDvalue = candidateValue;
%		break;
%	}
%}
%}

\chapter{A Grammar for Spreadsheet Formulas Evaluated on Two Large Datasets}
\label{appendix:xlparser}

The following is a verbatim copy of the papers as it was published in the proceedings of SCAM 2015.

\clearpage

\includepdf[pages={1-10}]{appendix/scam-xlparser.pdf}

\clearpage

\vspace*{\fill}

\centering
\large{End of appendix \ref{appendix:xlparser}}

\vspace*{\fill}

\end{document}

